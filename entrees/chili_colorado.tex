\section{Chili Colorado}\label{chili_colorado_goat}\index{chili!Chili~Colorado}\index{pork!Chili~Colorado}


\begin{recipestats}[
	servings=4 people,
	preptime=30 \minute,
	bakingtime=2 \hour,
	source=\citeauthor{rMartinez2015}~\cite{rMartinez2015},
	]
\end{recipestats}


\ragmarpar{%
	Your chilies are fresh if they are pliable.
	Do not use if they are brittle.
}
\begin{recipeabstract}
	``Chili Colorado. It's a traditional Mexican dish of beef or pork stewed in a red chili sauce, chili 'colored red,' not chili from the state of Colorado'' \cite{rMartinez2015}.
	The dried chilies lend a sweetness to the dish not possible with chili powder.
\end{recipeabstract}


\begin{ingredientcolumns}
	\begin{ingredientblock}
		\ingredient[5]{ancho}\\
		\ingredient[2]{pasilla}\\
		\ingredient[2]{guajillo}\\
		\ingredient[8][C]{stock, chicken}\\
		\ingredient[2][lb]{pork shoulder}
	\end{ingredientblock}
	\begin{ingredientblock}
		\ingredient[8][]{garlic cloves}\\
		\ingredient[2][]{bay leaves}\\
		\ingredient[1][\Tablespoon]{cumin, ground}\\
		\ingredient[1][\teaspoon]{sage, fresh}\\
		\ingredient[2][\teaspoon]{oregano, Mexican}
	\end{ingredientblock}
\end{ingredientcolumns}


\begin{preparation}
\item Measure the spices, chop the sage, and smash the garlic.

\item Cut pork into $\approx1$ $inch$ cubes, toss with salt, pepper.

\ragmarpar{Toss pork with a bit of flour as well to thicken the chili further.}
\item Brown the pork.
	Heat a neutral oil almost to smoking point in a $ \geqq 3.5$ $quart$ pot.
	Reduce to medium high.
	Brown in batches so as to not overcrowd, de-glazing if necessary to prevent burning.

\item Add spices, stir for about a minute.

\item Add 5 $C$ stock, simmer uncovered for 1 \hour.

\item Meanwhile, re-hydrate chilies.
	Remove stems, seeds, veins from chilies and roughly chop.
	Add to large bowl, add 3 $C$ boiling stock, cover with plastic wrap.
	Wait 30 \minute, then blend all.

\item Add blended chilies to the soup at the end of the first simmer.
	Simmer for 45 \minute~uncovered.

\item Season with salt pepper to taste.
\end{preparation}


\begin{variation}
\item Marjoram can be a substitute for the Mexican oregano if necessary, but not Mediterranean oregano.

\item Serve with tortillas, and the carrots from \nameref{mexican_beef_bowl}~\ref{mexican_beef_bowl}.  \citeauthor{rMartinez2015} recommends rice, beans a la charra, and tortillas.
\end{variation}


\begin{experiments}
\item Should we cook the chili sauce in some oil, like in enchiladas?
\end{experiments}

\recipeend%
