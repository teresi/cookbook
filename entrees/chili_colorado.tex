\begin{recipe}[
	preparationtime = 30 minutes,
	bakingtime = 2 hours,
	portion = 4 people,
	source = Rick Martinez; Bon Apetit,
	]
	{Chili Colorado}
	\graph{
		smallpicturewidth = 0.3\textwidth,%
		bigpicturewidth = 0.6\textwidth,%
	}
	%%%%%%%%%%%%%%%%%%%%%%%%%%%%%%%%%%%%%%%%%%%%%%%%%%%%%%
	\ingredients{
		\unit[5]{} & ancho\\
		\unit[2]{} & pasilla\\
		\unit[2]{} & guajillo\\
		\unit[8]{C} & stock, chicken\\
		\unit[2]{lb} & pork shoulder\\
		\unit[6...9]{} & garlic cloves\\
		\unit[2]{} & bay leaves\\
		\unit[1]{Tbs} & cumin, ground\\
		\unit[2]{tsp} & sage, fresh\\
		\unit[2]{tsp} & oregano, Mexican
	}
	
	
	%%%%%%%%%%%%%%%%%%%%%%%%%%%%%%%%%%%%%%%%%%%%%%%%%%%%%%
	\preparation{
		\step Measure the spices, chop the sage, and crush the garlic.
		\step Prepare the pork. Cut into $\approx1\;inch$ cubes, toss with salt, pepper.
		\step Brown the pork. Heat a neutral oil almost to smoking point in a $ \geqq 3.5\;quart$ pot. Reduce to medium high. Brown in batches so as to not overcrowd, de-glazing if necessary to prevent burning.
		\step Add spices, stir for about a minute.
		\step Add $5\;Cups$ stock, simmer uncovered for $1\;hour$.
		\step Meanwhile,re-hydrate the chilies. Remove stems, seeds, veins from chilies and roughly chop. Add to large bowl, add $3\;Cups$ boiling stock, cover with plastic wrap. Wait $30\;minutes$, then blend it all.
		\step Add blended chilies to the soup at the end of the first simmer. Simmer for $45\;minutes$ uncovered.
		\step Season with salt pepper to taste.
	}
	%%%%%%%%%%%%%%%%%%%%%%%%%%%%%%%%%%%%%%%%%%%%%%%%%%%%%%
	
	\hint{
		\begin{itemize}
			\item "Chili Colorado" means "chili colored red" rather than from the state of Colorado.
			\item Toss pork with a bit of flour as well to thicken the chili further.
			\item Marjoram can be a substitute for the Mexican oregano if necessary, but not Mediterranean oregano.
			\item Serve with tortillas, and the carrots from 'Mexican Beef Bowl' \ref{mexican_beef_bowl}. Rick Martinez recommends rice, beans a la charra, and tortillas.
		\end{itemize}
	}
	%%%%%%%%%%%%%%%%%%%%%%%%%%%%%%%%%%%%%%%%%%%%%%%%%%%%%%
	\setRecipeLengths{
		preparationwidth = 0.60\textwidth,
		ingredientswidth = 0.35\textwidth,
		pictureheight = 6cm,
		bigpicturewidth = 0.6\textwidth,
		smallpicturewidth = 0.35\textwidth
	}
	%%%%%%%%%%%%%%%%%%%%%%%%%%%%%%%%%%%%%%%%%%%%%%%%%%%%%%
	\setRecipeSizes{
		recipename = \fontsize{25pt}{30pt},
		ing = \normalsize,
		inghead = \normalsize,
		prep = \normalsize,
		prephead = \normalsize,
		hint = \normalsize,
		hinthead = \Large
	}
	%%%%%%%%%%%%%%%%%%%%%%%%%%%%%%%%%%%%%%%%%%%%%%%%%%%%%%
	%\setRecipenameFont{
	%%pbsi
	%%fau
	%%fwb
	%%fjd % default when using the option handwritten
	%cmr % default
	%}{T1}{m}{n}
	%%%%%%%%%%%%%%%%%%%%%%%%%%%%%%%%%%%%%%%%%%%%%%%%%%%%%%
	\setHeadlines{
		inghead = Ingredients,
		prephead = Preparation,
		hinthead = Hint,
		calory = Cal,
		continuationhead = Continuation,
		continuationfoot = Continuation on next page
	}
	%%%%%%%%%%%%%%%%%%%%%%%%%%%%%%%%%%%%%%%%%%%%%%%%%%%%%%
	%\setBackgroundPicture
	%[%
	%x = 2cm,
	%y = -1cm,
	%width=\paperwidth-3cm,
	%height,
	%orientation=pagecenter
	%]{pic/bg_transparent} % filepath
\end{recipe}

%\clearpage
%\thispagestyle{empty}
% \begin{tikzpicture}[remember picture,overlay]
% inelegant way of getting a good image:
% use [keepaspectratio], trim to an aspect ratio close to the page, then remove [keepaspectraio] to get full page
%   \node at (current page.center) {\includegraphics[width=\pdfpagewidth,height=\pdfpageheight,clip,trim={47px 10px 47px 10px}]{tricolor}};
% IMAGE: http://www.homemadeitaliancooking.com/italian-rainbow-cookies/
%\end{tikzpicture}