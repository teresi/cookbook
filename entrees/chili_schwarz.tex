\section{Chili Schwarz}
\begin{recipestats}[
	servings=4 people,
	preptime=1 hour,
	bakingtime=1 hour,
	]
\end{recipestats}

\begin{recipeabstract}
	This was an experiment on \nameref{chili_colorado_goat} \ref{chili_colorado_goat} to use more black chilies and reduce cook time.
\end{recipeabstract}

\begin{ingredientcolumns}
	\begin{ingredientblock}
		\ingredient[4]{ancho}\\
		\ingredient[4]{pasilla}\\
		\ingredient[1]{guajillo}\\
		\ingredient[8][C]{stock, beef}\\
		\ingredient[\approx \nicefrac{3}{2}][lb]{ground beef}\\
		\ingredient[1]{onion, red}\\
		\ingredient[2][]{pablano}
	\end{ingredientblock}
	\begin{ingredientblock}
		\ingredient[3][]{carrots}\\
		\ingredient[3][]{garlic cloves}\\
		\ingredient[3][]{black garlic cloves}\\
		\ingredient[2][]{bay leaves}\\
		\ingredient[1][\Tablespoon]{cumin, ground}\\
		\ingredient[2][\teaspoon]{oregano, Mexican}
	\end{ingredientblock}
\end{ingredientcolumns}

\ragmarpar{Leftovers may need some chili d\`{e} arbol to bring back the heat}

\begin{preparation}
\item Preheat oven to $400\;F^\circ$. Remove seeds / veins / stems of chilies. Squash garlic and cut off the ends. Toast chilies for $\approx 1...2$ on a sheet (make certain not to burn them).
\item Add chilies \& garlic to a bowl, cover w/ boiling water. Keep submerged, cover w/ plastic wrap, steep for $20...30\;min$.
\item Meanwhile begin browning the beef. Achieve a strong color and deglaze often. Recommend 1/2 in a dutch oven, 1/2 in a pan, drain and reserve fat from pan. Deglaze with a toasty beer for a bit of flair.
\item Julienne the onion, lightly saut\`{e} in pan w/ pepper. Meanwhile slice the Pablano, roughly chop the carrots, set aside.
\item When the chilies are done remove from the water into a food processor. Add black garlic. Blend well with a bit of the stock for $\approx5\;min$.
\item Add cumin to beef, stir $\approx 1\;min$, add chili sauce.
\item Add onion, broth / stock, bay leaf, oregano. Bring to simmer. Simmer for $\approx30...60\;min$.
\item Meanwhile, saut\`{e} the carrot, add to chili, repeat with Pablano. Season with salt.
\end{preparation}

\begin{variation}
	\item There is not a clear answer on whether or not to use the liquid used to re-hydrate the chilies. Prefer to toss it if it tastes bitter.
\end{variation}

\begin{experiments}
	\item Still working out which chilies to use. Maybe try some mulato or morita?
\end{experiments}

\recipeend
