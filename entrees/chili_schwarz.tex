\begin{recipe}[
	preparationtime = 1 hour,
	bakingtime = 1 hour,
	portion = 4 people,
	]
	{Chili Schwarz}
	\graph {
		smallpicturewidth = 0.3\textwidth,%
		bigpicturewidth = 0.6\textwidth,%
	}

	\ingredients {
		\unit[3]{} & Ancho\\
		\unit[3]{} & Pasilla\\
		\unit[1]{} & Guajillo\\
		\unit[8]{C} & stock / broth\\
		\unit[1...2]{lb} & ground beef\\
		\unit[1]{} & onion, red\\
		\unit[1...2]{} & Pablano\\
		\unit[3]{} & carrots\\
		\unit[3]{} & garlic cloves\\
		\unit[3]{} & black garlic cloves\\
		\unit[2]{} & bay leaves\\
		\unit[1]{Tbs} & cumin, ground\\
		\unit[2]{tsp} & oregano, Mexican
	}
	
	\preparation
	{
		\step Preheat oven to $400\;F^\circ$. Remove seeds / veins / stems of chilies. Squash garlic and cut off the ends. Toast chilies for $\approx 1...2$ on a sheet (make certain not to burn them).
		\step Add chilies \& garlic to a bowl, cover w/ boiling water. Keep submerged, cover w/ plastic wrap, steep for $20...30\;min$.
		\step Meanwhile begin browning the beef. Achieve a strong color and deglaze often. Recommend 1/2 in a dutch oven, 1/2 in a pan, drain and reserve fat from pan. Deglaze with a toasty beer for a bit of flair.
		\step Julienne the onion, lightly saut\`{e} in pan. Meanwhile slice the Pablano, roughly chop the carrots, set aside.
		\step When the chilies are done remove from the water into a food processor. Add black garlic. Blend well for $\approx5\;min$.
		\step Add cumin to beef, stir $\approx 1\;min$, add chili sauce.
		\step Add onion, broth / stock, bay leaf, oregano. Bring to simmer. Simmer for $\approx30...60\;min$.
		\step Meanwhile, saut\`{e} the carrot, add to chili, repeat with Pablano. Season with salt.
	}
	
	\hint{
		\begin{itemize}
			\item Consider adding ground Arbol to leftovers which will mellow out.
		\end{itemize}
	}
	%%%%%%%%%%%%%%%%%%%%%%%%%%%%%%%%%%%%%%%%%%%%%%%%%%%%%%
	\setRecipeLengths{
		preparationwidth = 0.60\textwidth,
		ingredientswidth = 0.35\textwidth,
		pictureheight = 6cm,
		bigpicturewidth = 0.6\textwidth,
		smallpicturewidth = 0.35\textwidth
	}
	%%%%%%%%%%%%%%%%%%%%%%%%%%%%%%%%%%%%%%%%%%%%%%%%%%%%%%
	\setRecipeSizes{
		recipename = \fontsize{25pt}{30pt},
		ing = \normalsize,
		inghead = \normalsize,
		prep = \normalsize,
		prephead = \normalsize,
		hint = \normalsize,
		hinthead = \Large
	}
	%%%%%%%%%%%%%%%%%%%%%%%%%%%%%%%%%%%%%%%%%%%%%%%%%%%%%%
	%\setRecipenameFont{
	%%pbsi
	%%fau
	%%fwb
	%%fjd % default when using the option handwritten
	%cmr % default
	%}{T1}{m}{n}
	%%%%%%%%%%%%%%%%%%%%%%%%%%%%%%%%%%%%%%%%%%%%%%%%%%%%%%
	\setHeadlines{
		inghead = Ingredients,
		prephead = Preparation,
		hinthead = Hint,
		calory = Cal,
		continuationhead = Continuation,
		continuationfoot = Continuation on next page
	}
	%%%%%%%%%%%%%%%%%%%%%%%%%%%%%%%%%%%%%%%%%%%%%%%%%%%%%%
	%\setBackgroundPicture
	%[%
	%x = 2cm,
	%y = -1cm,
	%width=\paperwidth-3cm,
	%height,
	%orientation=pagecenter
	%]{pic/bg_transparent} % filepath
\end{recipe}

%\clearpage
%\thispagestyle{empty}
% \begin{tikzpicture}[remember picture,overlay]
% inelegant way of getting a good image:
% use [keepaspectratio], trim to an aspect ratio close to the page, then remove [keepaspectraio] to get full page
%   \node at (current page.center) {\includegraphics[width=\pdfpagewidth,height=\pdfpageheight,clip,trim={47px 10px 47px 10px}]{tricolor}};
% IMAGE: http://www.homemadeitaliancooking.com/italian-rainbow-cookies/
%\end{tikzpicture}
