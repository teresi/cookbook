\begin{recipe}[
	preparationtime = 1 hour,
	bakingtime = 1 hour,
	portion = 4 people,
	]
	{Chili Schwarz}
	\graph {
		smallpicturewidth = 0.3\textwidth,%
		bigpicturewidth = 0.6\textwidth,%
	}

	\ingredients {
		\unit[2]{} & Ancho\\
		\unit[2]{} & Pasilla\\
		\unit[2]{} & Guajillo\\
		\unit[8]{C} & stock / broth\\
		\unit[1]{lb} & ground beef\\
		\unit[1]{} & onion, red\\
		\unit[1]{} & Pablano\\
		\unit[3]{} & carrots\\
		\unit[4]{} & garlic cloves\\
		\unit[2]{} & black garlic cloves\\
		\unit[2]{} & bay leaves\\
		\unit[2]{tsp} & cumin, ground\\
		\unit[2]{tsp} & oregano, Mexican\\
		\unit[1]{Tbs} & butter
	}
	
	\preparation
	{
		\step Bring $\approx 4\;Cups$ water in an iron skillet / pan to a boil. Reduce to very low.
		\step Meanwhile, remove seeds \& veins of chilies, add to skillet. Add 1 black garlic clove, add 2 squashed cloves. Simmer $30...40\; min$, stir occasionally, keep submerged.
		\step Meanwhile begin browning the beef, making sure to deglaze often and working in batches if necessary.
		\step Julienne the onion, lightly saut\`{e} in another pan. Slice the Pablano, roughly chop the carrots, set aside.
		\step When the chilies / garlic are done, remove from the water into a food processor. Blend well for $\approx5\;min$.
		\step Add cumin to beef, stir $\approx 1\;min$, add chili paste.
		\step Add onion, broth / stock, bay leaf, oregano, chopped black garlic clove. Bring to simmer. Simmer for $\approx30...60\;min$.
		\step Meanwhile, saut\`{e} the carrot, add to chili, repeat with Pablano.
		\step Squash 2 garlic cloves, mince, lightly saut\`{e} in butter. Add to chili. Season with salt.
	}
	
	\hint{
		\begin{itemize}
			\item Consider adding ground Arbol to leftovers which will mellow out.
			\item Optionally toast chilies $1..2\;min$ at $400^\circ F$ prior to rehydrating.
		\end{itemize}
	}
	%%%%%%%%%%%%%%%%%%%%%%%%%%%%%%%%%%%%%%%%%%%%%%%%%%%%%%
	\setRecipeLengths{
		preparationwidth = 0.60\textwidth,
		ingredientswidth = 0.35\textwidth,
		pictureheight = 6cm,
		bigpicturewidth = 0.6\textwidth,
		smallpicturewidth = 0.35\textwidth
	}
	%%%%%%%%%%%%%%%%%%%%%%%%%%%%%%%%%%%%%%%%%%%%%%%%%%%%%%
	\setRecipeSizes{
		recipename = \fontsize{25pt}{30pt},
		ing = \normalsize,
		inghead = \normalsize,
		prep = \normalsize,
		prephead = \normalsize,
		hint = \normalsize,
		hinthead = \Large
	}
	%%%%%%%%%%%%%%%%%%%%%%%%%%%%%%%%%%%%%%%%%%%%%%%%%%%%%%
	%\setRecipenameFont{
	%%pbsi
	%%fau
	%%fwb
	%%fjd % default when using the option handwritten
	%cmr % default
	%}{T1}{m}{n}
	%%%%%%%%%%%%%%%%%%%%%%%%%%%%%%%%%%%%%%%%%%%%%%%%%%%%%%
	\setHeadlines{
		inghead = Ingredients,
		prephead = Preparation,
		hinthead = Hint,
		calory = Cal,
		continuationhead = Continuation,
		continuationfoot = Continuation on next page
	}
	%%%%%%%%%%%%%%%%%%%%%%%%%%%%%%%%%%%%%%%%%%%%%%%%%%%%%%
	%\setBackgroundPicture
	%[%
	%x = 2cm,
	%y = -1cm,
	%width=\paperwidth-3cm,
	%height,
	%orientation=pagecenter
	%]{pic/bg_transparent} % filepath
\end{recipe}

%\clearpage
%\thispagestyle{empty}
% \begin{tikzpicture}[remember picture,overlay]
% inelegant way of getting a good image:
% use [keepaspectratio], trim to an aspect ratio close to the page, then remove [keepaspectraio] to get full page
%   \node at (current page.center) {\includegraphics[width=\pdfpagewidth,height=\pdfpageheight,clip,trim={47px 10px 47px 10px}]{tricolor}};
% IMAGE: http://www.homemadeitaliancooking.com/italian-rainbow-cookies/
%\end{tikzpicture}
