\section[Spinach Quiche]{Spinach Artichoke Quiche}
\begin{recipestats}[
	servings=4 people,
	preptime=45 minutes,
	bakingtime=45 minutes,
	source=\citefield{joyofcooking2006}{title} \cite{joyofcooking2006},
	]
\end{recipestats}

\begin{recipeabstract}
	A classic dish.
	We like to serve this as a side along with cardamom rice and have leftovers for breakfast.
\end{recipeabstract}

\begin{ingredientcolumns}
	\begin{ingredientblock}
		\ingredient[1][]{pie pastry}\\
		\ingredient[5][large]{eggs}\\
		\ingredient[1][C]{cream}\\
		\ingredient[2][C]{spinach}\\
		\ingredient[\nicefrac{3}{2}][C]{Asiago}\\
		\ingredient[6][\fluidounce]{artichokes}
	\end{ingredientblock}
	\begin{ingredientblock}
		\ingredient[1][]{onion, white}\\
		\ingredient[2][Tbs]{butter}\\
		\ingredient[2][tsp]{garlic}\\
		\ingredient[\onefourth][tsp]{salt}\\
		\ingredient[\onehalf][tsp]{pepper, white}
	\end{ingredientblock}
\end{ingredientcolumns}

\begin{preparation}
\item Remove ingredients from the refrigerator.
\item Slice onion, begin cooking over medium-low heat with a little oil while stirring occasionally.
\item Measure and set aside remaining ingredients. Roughly chop artichokes \& spinach, shred cheese, separate 1 egg yolk, add remaining eggs to cream, grind spices.
\item Butter pie pan, add pastry, partially pre-bake (see \ref{basicFlakyPiePastry}). Immediately brush with egg yolk when the pastry is removed from oven.
\item Meanwhile prepare filling. Whisk egg, cream, spices. Add spinach to onions. Add garlic and remaining butter in the last minute or so.
\item Layer into pie: cheese, vegetables, egg mixture. Optionally garnish w/ cayenne $\|$ chili flakes $\|$ paprika for some pizzazz.
\item Bake at $375\; F^\circ$ for $30\; min$, broil briefly, rest $10\; min$.
\end{preparation}


\begin{variation}
\item This recipe has a lot of leeway. Try marjoram, cayenne, or a different pepper corn for example. Asiago / artichoke / spinach, or, cheddar / spinach both work well.
\item Other milk varieties are available but heavy cream will make the best texture IMHO.
\end{variation}


\recipeend
