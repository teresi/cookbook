\section{Hoola Poola}


\begin{recipestats}[
	servings=1 person,
	preptime=10 \minute,
	bakingtime=20 \minute,
	source=Teresi Family,
	]
\end{recipestats}


\begin{recipeabstract}
	This was created for a cost-effective lunch.
	The ingredients keep a long time and can mostly be prepared before hand.
	The name is a play on the Giesken's balogna \& eggs recipe Hunka Punka. % FUTURE add ref to Hunka Punka
\end{recipeabstract}


\begin{ingredientcolumns}
	\begin{ingredientblock}
		\ingredient[4][\ounce]{Spam}\\
		\ingredient[2][C]{collards}\\
		\ingredient[1][C]{chickpeas, cooked}\\
		\ingredient[1][C]{rice, cooked}
	\end{ingredientblock}
	\begin{ingredientblock}
		\ingredient[\approx \onehalf][\teaspoon]{Sa\'zon}\\
		\ingredient[1][\Tablespoon]{butter}\\
		\ingredient[\onehalf][\Tablespoon]{garlic, minced}
	\end{ingredientblock}
\end{ingredientcolumns}


\begin{preparation}
\item Dice and add Spam to frying pan over medium heat.
	Optionally add mushrooms, or other customization.
	Stir occasionally.

\item Meanwhile prepare the collards.
	Discard woody stems, slice leaves ($\approx1x1"$), let soak in water w/ a dash of vinegar and salt.
	Set aside remaining ingredients.

\item Once Spam is toasted, drain majority of water from collards, add to pan, add Sa\'zon.
	Stir often and cook until collards begin to get crispy.

\item Once collards are toasted, reduce heat a bit, expose center of pan by pushing mixture aside and add butter and garlic.
	Cook to a light toast ($\approx[30\dots 60]$ \second).

\item Serve with rice.
	Top with a dash of Sa\'zon and yell ``HOOLA POOLA!'' for a bit of pizzazz.
\end{preparation}


\recipeend
