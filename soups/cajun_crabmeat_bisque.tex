\begin{recipe}[
    preparationtime = 45 $min$,
    bakingtime = 45 $min$,
    portion = 5 people,
    source = Art of the Palate; 2006 (Pamela Teresi)
]
{Cajun Crabmeat Bisque}
\graph{
    %small = strawberry,
    smallpicturewidth = 0.3\textwidth,
    %big = strawberrycake,
    bigpicturewidth = 0.6\textwidth,
}
%%%%%%%%%%%%%%%%%%%%%%%%%%%%%%%%%%%%%%%%%%%%%%%%%%%%%%
\ingredients{
\unit[3/4]{C} & butter\\
\unit[3/4]{C} & flour\\
\unit[3]{Tbs} & tomato paste\\
\unit[1 1/2]{C} & yellow onions, chopped\\
\unit[1]{C} & celery\\
\unit[1/2]{C} & scallions\\
\unit[4]{cloves} & garlic\\
\unit[2/3]{C} & green pepper\\
\unit[3]{Tbs} & parsley\\
\unit[2]{quarts} & stock, chicken\\
\unit[1]{Tbs} & Worcestershire\\
\unit[1]{} & bay leaf\\
\unit[1]{tsp} & thyme, dried\\
\unit[1]{tsp} & salt\\
\unit[1/8]{tsp} & black pepper\\
\unit[1/8]{tsp} & cayenne pepper\\
\unit[1/2]{tsp} & ketchup\\
\unit[1]{lb} & crabmeat
}
%%%%%%%%%%%%%%%%%%%%%%%%%%%%%%%%%%%%%%%%%%%%%%%%%%%%%%
\preparation{

\step Make roux: melt butter, gradually combine flour, stirring constantly on medium heat, $20...30\; min$ or until golden brown.

\step Add tomato, vegetables finely diced, sweat.

\step Add stock gradually. Add spices, crab meat.

\step Simmer 40 minutes, covered, stirring occasionally.
}
%%%%%%%%%%%%%%%%%%%%%%%%%%%%%%%%%%%%%%%%%%%%%%%%%%%%%%
\hint{
\begin{itemize}
\item Serve with hot sauce \& yeast rolls.
\item Vary amount of stock, crab, cayenne to taste. Adding a quart of stock \& 8 $oz$ crab can work.
\end{itemize}
}
%%%%%%%%%%%%%%%%%%%%%%%%%%%%%%%%%%%%%%%%%%%%%%%%%%%%%%
\setRecipeLengths{
preparationwidth = 0.60\textwidth,
ingredientswidth = 0.35\textwidth,
pictureheight = 6cm,
bigpicturewidth = 0.6\textwidth,
smallpicturewidth = 0.35\textwidth
}
%%%%%%%%%%%%%%%%%%%%%%%%%%%%%%%%%%%%%%%%%%%%%%%%%%%%%%
\setRecipeSizes{
recipename = \fontsize{25pt}{30pt},
ing = \normalsize,
inghead = \normalsize,
prep = \normalsize,
prephead = \normalsize,
hint = \normalsize,
hinthead = \Large
}
%%%%%%%%%%%%%%%%%%%%%%%%%%%%%%%%%%%%%%%%%%%%%%%%%%%%%%
%\setRecipenameFont{
%%pbsi
%%fau
%%fwb
%%fjd % default when using the option handwritten
%cmr % default
%}{T1}{m}{n}
%%%%%%%%%%%%%%%%%%%%%%%%%%%%%%%%%%%%%%%%%%%%%%%%%%%%%%
\setHeadlines{
inghead = Ingredients,
prephead = Preparation,
hinthead = Hint,
calory = Cal,
continuationhead = Continuation,
continuationfoot = Continuation on next page
}
%%%%%%%%%%%%%%%%%%%%%%%%%%%%%%%%%%%%%%%%%%%%%%%%%%%%%%
%\setBackgroundPicture
%[%
%x = 2cm,
%y = -1cm,
%width=\paperwidth-3cm,
%height,
%orientation=pagecenter
%]{pic/bg_transparent} % filepath
\end{recipe}

%\clearpage
%\thispagestyle{empty}
% \begin{tikzpicture}[remember picture,overlay]
   % inelegant way of getting a good image:
   % use [keepaspectratio], trim to an aspect ratio close to the page, then remove [keepaspectraio] to get full page
%   \node at (current page.center) {\includegraphics[width=\pdfpagewidth,height=\pdfpageheight,clip,trim={47px 10px 47px 10px}]{tricolor}};
   % IMAGE: http://www.homemadeitaliancooking.com/italian-rainbow-cookies/
%\end{tikzpicture}
