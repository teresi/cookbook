\begin{recipe}[
    preparationtime = 60 minutes,
    bakingtime = 2 weeks,
    portion = 12 fluid ounces,
    source = \citefield{fieryferments2017}{booktitle}\cite{fieryferments2017}; \textit{Pickl-It},
]
{Base Brined Hot Sauce}
\graph{
    %small = strawberry,
    smallpicturewidth = 0.3\textwidth,
    %big = strawberrycake,
    bigpicturewidth = 0.6\textwidth,
}

%%%%%%%%%%%%%%%%%%%%%%%%%%%%%%%%%%%%%%%%%%%%%%%%%%%%%%
\ingredients{
\unit[1]{} & Mason jar, 32 oz\\
\unit[1]{} & Pickling weight\\
\unit[1]{} & Air lock\\
\unit[]{} & Star San\\
\\
\unit[]{} & water, un-chlorinated\\
\unit[]{} & salt, un-iodized\\
\unit[]{} & peppers, fresh $\|$ dried\\
\unit[]{} & spices\\
\\
\unit[1/4]{tsp} & Achiote
}
%%%%%%%%%%%%%%%%%%%%%%%%%%%%%%%%%%%%%%%%%%%%%%%%%%%%%%
\preparation{
\label{base_lacto_brine_hotsauce}
\step Sanitize equipement with Star San or equivalent.
\step Prepare a $[5...5.3]\%$ brine by weight, $\approx3/4$ volume of jar.
\step Prepare dried peppers by removing stems, veins, and seeds. Keep flesh largely intact by slicing stem off and cutting once down length-wise, then unrolling the pepper.
\step Prepare fresh peppers by removing stems / seeds and cutting into large portions.
\step Add ingredients to jar. Prevent ingredients from floating up by starting with smaller pieces, then top off with a large dried pepper or cabbage leaf tucked down the sides of the jar. Reserve Achiote for post fermentation.
\step Add a fermentation weight to top of ingredients.
\step Add brine mixture, leaving $\approx1cm$ of head space. Remove ingredients floating on top of brine.
\step Add lid and air lock, making sure the lock vent is not covered in brine.
\step Ferment for $[1...2]$ weeks or up to many months, then place in fridge until ready to pur\`{e}e. Make certain that everything is covered in brine.
\step Separate brine and ingredients, pur\`{e}e with $\approx1/2\;Cup$ brine. Strain if desired; recommended if dried peppers are used. Optionally blend Achiote for color.
\step Add vinegar or brine to sauce to desired consistency, refridgerate.
}
\vspace{-4cm}
%%%%%%%%%%%%%%%%%%%%%%%%%%%%%%%%%%%%%%%%%%%%%%%%%%%%%%

\hint{
\begin{itemize}
\item This base recipe is the technique for whole brined peppers $c.f.$ mashed peppers. Peppers are fermented whole and then pur\`{e}ed rather than mashed prior to fermentation, SEE "Mixed-Media Basic Mash" of Fiery Ferments.
\item Prevent spoilage by keeping ingredients submerged and by not removing the lid (keep it anaerobic). SEE Fiery Ferments for troubleshooting. In general: you should throw out the batch if you see any fuzzy mold.
\item The heat level of the sauce lowers drastically over fermentation. Consider compensating for example with Thai, Pequin, or Arbol.
\end{itemize}
}
%%%%%%%%%%%%%%%%%%%%%%%%%%%%%%%%%%%%%%%%%%%%%%%%%%%%%%
\setRecipeLengths{
preparationwidth = 0.60\textwidth,
ingredientswidth = 0.35\textwidth,
pictureheight = 6cm,
bigpicturewidth = 0.6\textwidth,
smallpicturewidth = 0.35\textwidth
}
%%%%%%%%%%%%%%%%%%%%%%%%%%%%%%%%%%%%%%%%%%%%%%%%%%%%%%
\setRecipeSizes{
recipename = \fontsize{25pt}{30pt},
ing = \normalsize,
inghead = \normalsize,
prep = \normalsize,
prephead = \normalsize,
hint = \normalsize,
hinthead = \Large
}
%%%%%%%%%%%%%%%%%%%%%%%%%%%%%%%%%%%%%%%%%%%%%%%%%%%%%%
%\setRecipenameFont{
%%pbsi
%%fau
%%fwb
%%fjd % default when using the option handwritten
%cmr % default
%}{T1}{m}{n}
%%%%%%%%%%%%%%%%%%%%%%%%%%%%%%%%%%%%%%%%%%%%%%%%%%%%%%
\setHeadlines{
inghead = Ingredients,
prephead = Preparation,
hinthead = Hint,
calory = Cal,
continuationhead = Continuation,
continuationfoot = Continuation on next page
}
%%%%%%%%%%%%%%%%%%%%%%%%%%%%%%%%%%%%%%%%%%%%%%%%%%%%%%
%\setBackgroundPicture
%[%
%x = 2cm,
%y = -1cm,
%width=\paperwidth-3cm,
%height,
%orientation=pagecenter
%]{pic/bg_transparent} % filepath
\end{recipe}

%\clearpage
%\thispagestyle{empty}
% \begin{tikzpicture}[remember picture,overlay]
   % inelegant way of getting a good image:
   % use [keepaspectratio], trim to an aspect ratio close to the page, then remove [keepaspectraio] to get full page
%   \node at (current page.center) {\includegraphics[width=\pdfpagewidth,height=\pdfpageheight,clip,trim={47px 10px 47px 10px}]{tricolor}};
   % IMAGE: http://www.homemadeitaliancooking.com/italian-rainbow-cookies/
%\end{tikzpicture}