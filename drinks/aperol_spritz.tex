\section[Aperol Spritz]{Aperol Spritz\ragStars{1}}
\index{Aperol!Aperol Spritz}
\index{champagne!Aperol Spritz}

\begin{recipestats}[
	servings=1,
	preptime=5 \minute,
	original=\citefield{cocktailSeminars2021}{title} \cite{cocktailSeminars2021},
]
\end{recipestats}


\ragmarpar{Prosecco may be best to provide sweetness.}
\begin{recipeabstract}
	A spritz is composed of carbonated wine, bitter liquer, and soda.
	Any Amaro can be used, which is an Italian herbal liqueur.
	The Aperol here gives it a bitter citrus flavor, similar to but more subdued than Campari.
	One could be compare this spritz to an IPA.
\end{recipeabstract}


\begin{ingredientcolumns}[1]
	\begin{ingredientblock}
		\ingredient[3][\fluidounce]{sparkling wine}\\
		\ingredient[2][\fluidounce]{Aperol}\\
		\ingredient[1][\fluidounce]{soda water}
	\end{ingredientblock}
\end{ingredientcolumns}
\ragmarpar{One traditonally eats the orange slice after the spritz.}


\begin{preparation}
\item Fill large wine glass halfway with ice.
\item Add ingredients, stir, garnish with an orange slice.
\end{preparation}


\recipeend
