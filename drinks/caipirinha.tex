\section[Caipirinha]{Caipirinha\ragStars{3}}
\index{lime!Caipirinha}
\index{cacha\c{c}a!Caipirinha}


\begin{recipestats}[
	servings=2,
	preptime=5 \minute,
	source=Kevin \& Marina,
]
\end{recipestats}
\ragmarpar{Marina makes this to taste, these ratios are guidelines.}


\begin{recipeabstract}
	Brazil's national drink.
	Cacha\c{c}a is a spirit distilled from sugarcane juice typically used for mixed drinks.
	It has a slight fruity / vegetal flavor that is distinct.
	It is not considered a rum which is typically made with molasses.
\end{recipeabstract}


\begin{ingredientcolumns}[1]
	\begin{ingredientblock}
		\ingredient[2]{limes}\\
		\ingredient[2][\fluidounce]{cacha\c{c}a}\\
		\ingredient[2][\Tablespoon]{sugar}
	\end{ingredientblock}
\end{ingredientcolumns}


\begin{preparation}
\item Roll limes on counter w/ your palm to help release the juice.

\ragmarpar{Domino `quick dissolve' sugar is best.}
\item Slice limes into eigths, add to pitcher, add sugar.

\item Muddle the limes for a few minutes.

\item Fill with $\approx 1$ $C$ ice, add cacha\c{c}a, stir.

\item Pour into a highball with ice.
	You may want to leave most of the limes in the pitcher for making more batches.
\end{preparation}


\begin{variation}
\item Strawberry or passion fruit are popular.
	Muddle other fruit with the limes.
\end{variation}


\recipeend
