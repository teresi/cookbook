\section[Hemingway Daiquiri]{\ragStars{3}Hemingway Daiquiri, or,\\ La Floridita Daiquiri}
\index{grapefruit!Hemingway Daiquiri}
\index{rum!Hemingway Daiquiri}


\begin{recipestats}[
	servings=1,
	preptime=5 \minute,
	source=Mike \& Jane,
	original=\citefield{howToDrinkHemingway2017}{shorttitle} \cite{howToDrinkHemingway2017},
]
\end{recipestats}


\begin{recipeabstract}
	A low sugar variation of the Daiquiri originally created for Earnest Hemingway.
	The maraschino evens out the tart citrus.
\end{recipeabstract}


\begin{ingredientcolumns}[1]
	\begin{ingredientblock}
		\ingredient[2][\fluidounce]{rum, white}\\
		\ingredient[\threefourth][\fluidounce]{lime juice}
	\end{ingredientblock}
	\begin{ingredientblock}
		\ingredient[\onehalf][\fluidounce]{grapefruit juice}\\
		\ingredient[\onehalf][\fluidounce]{maraschino liqueur}
	\end{ingredientblock}
\end{ingredientcolumns}
\ragmarpar{Although out of style, a bit of sugar helps to balance this out.}


\begin{preparation}
\item Add ice to shaker, add ingredients, shake and serve in a coupe.
\end{preparation}


\begin{variation}
\item Some recipes swap the volumes for grapefruit / lime juice.
\end{variation}
\recipeend
