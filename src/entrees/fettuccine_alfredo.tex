\section{Christmas Alfredo}\index{pasta!Christmas~Alfredo}\index{Christmas!Christmas~Alfredo}


\begin{recipestats}[
	servings=2 people,
	preptime=15 \minute,
	bakingtime=20 \minute,
	source=Mike \& Jane,
	original=\citetitle{newCookBook2014}~\cite[p.~426]{newCookBook2014},
	]
\end{recipestats}


\begin{recipeabstract}
	A Teresi Christmas tradition.
	Celebrate being young and being alive.
	Enjoy with some bubbly.
\end{recipeabstract}
\ragmarpar{Mike likes to add mushrooms.}


\begin{ingredientcolumns}
	\begin{ingredientblock}
		\ingredient[10][\ounce]{fettuccine, dry}\\
		\ingredient[6][\ounce]{shrimp}\\
		\ingredient[3][\ounce]{Asiago}\\
		\ingredient[\approx~3][cloves]{garlic}
	\end{ingredientblock}
	\begin{ingredientblock}
		\ingredient[3][\Tablespoon]{butter, unsalted}\\
		\ingredient[12][\fluidounce]{cream, heavy}\\
		\ingredient[\onehalf][\teaspoon]{salt}\\
		\ingredient[\onefourth][\teaspoon]{pepper}
	\end{ingredientblock}
\end{ingredientcolumns}


\begin{preparation}
\item Reserve a large pot, skillet and an iron skillet.

\ragmarpar{Parmesan works well, Pecorino Romano does not melt, Fontina is too mild.}
\item Salt some water and begin to boil

\item Add the iron skillet to the oven and preheat to $400~\Fahrenheit$.

\item Measure ingredients, grate the cheese, peel the garlic.

\item Cook pasta to al dente, then toss with oil, set aside.

\item Meanwhile, defrost the shrimp in cold running water $\approx~5..8~\minute$.
	Toss with olive oil and a dash of cumin.

\item Melt the butter, saut\'{e} crushed garlic in a large saucepan $\approx1$ \minute~on medium-high.

\item Move the shrimp to the skillet and cook $\approx~8..10~\minute$.

\ragmarpar{Use a bit of the pasta water to thicken the sauce.}
\item Add cream, salt, pepper to sauce. Bring to boil then reduce heat, simmer uncovered $\approx3$ \minute~or until it begins to thicken.

\item Add pasta to sauce, toss to combine.

\item Plate and serve with shrimp on the side.
\end{preparation}

\begin{variation}
\item Try \reffull{sous_vide_shrimp}, it takes a little preparation but the timing is much easier.

\end{variation}

\recipeend%
