\section[Runza]{Runza}\index{beef!Kraut~Runza}


\begin{recipestats}[
    servings=16,
    preptime=4~\hour,
    bakingtime=30~\minute,
    inactivetime=overnight,
    source=Teresi Family,
    original=Mrs. Steinmiller,
]
\end{recipestats}


\begin{recipeabstract}
    We've updated the Steinmiller runza slightly to flesh out the method.
    Jane would visit our house to join our tradition when we first started dating.
    I asked her to work on the most important job of grinding the pepper and she is now known as the Pepper Boss.
    The key really is to have high quality and freshly ground pepper.
    Do not use pre-ground pepper!
    Make the filling the day before to save time.
\end{recipeabstract}

% space to encourage a page break without forcing the abstract to the bottom of page
% depends on prior recipes
\bigskip
\bigskip

\begin{ingredientcolumns}
    \begin{ingredientblock}[filling]
        \ingredient[2][\pound]{cabbage}\\
        \ingredient[1~\onehalf][\pound]{ground beef}\\
        \ingredient[1~\onehalf][\pound]{white onion}\\
        \ingredient[1][\Tablespoon]{bacon fat}\\
        \ingredient[1][\Tablespoon]{black pepper}\\
        \ingredient[\onehalf][\Tablespoon]{white pepper}
    \end{ingredientblock}

    \bigskip
    \begin{ingredientblock}[dough]
        \ingredient[3][large]{egg}\\
        \ingredient[\onehalf][C]{water}\\
        \ingredient[\threefourth][C]{sour cream}\\
        \ingredient[\threefourth][C]{butter}\\
        \ingredient[\onefourth][C]{sugar}\\
        \ingredient[4~\onehalf][C]{flour}\\
        \ingredient[2~\onefourth][\teaspoon]{yeast, active}

    \end{ingredientblock}
\end{ingredientcolumns}


\begin{preparation}
\item Measure the filling ingredients.
\ragmarpar{Make the filling the day before.}
\item Reserve a large soup pot and a large saut\'{e} pan.
\item Chop the onion and cabbage.
\item Set large pot to medium low, saut\'{e} pan to medium.
\item Sweat the cabbage in the large pot w/ some bacon fat.
\item Saut\'{e} the onion, working in batches, w/ some bacon fat.
    Transfer to the large pot when lightly browned.
\ragmarpar{You should be able to see the pepper flakes!}
\item Brown the beef in the saut\'{e} pan, deglaze.
    Move to the pot.
\item Salt and pepper it to taste.
\item Let cool and refridgerate until you're ready for the dough.
\item Measure the dough ingredients.
\item Reserve several baking sheets and a bread machine.
\ragmarpar{Try a bit \\of nutmeg in the filling.}
\item Add the dough ingredients in the order listed to the machine.
    Set to a basic dough program.
\item Move dough to a lightly floured surface when ready.
\item Cut dough and roll into balls $\approx 2$ $\ounce$ each.
\item Let rise 10 $\minute$, preheat oven to 375 $\Fahrenheit$.
\item Grease the baking sheets, reheat some filling.
\ragmarpar{Roll the edges more thinly than the center.}
\item Make an egg wash with an egg yolk and a splash of milk.
\item Roll the dough balls $\approx 6$ $\inch$ in diameter.
\item Add filling to dough and wrap into a bun, sealing it well.
\item Move buns to baking sheet, let rise 10 $\minute$, or so.
\item Brush on egg wash, grind some salt on top.
\ragmarpar{About the color of a medium amber ale.}
\item Bake 25 $\minute$ or until well browned
\item Let cool. Serve with onion rings and beer.
\end{preparation}


\recipeend%
