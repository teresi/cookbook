\section{Venison Tenderloin}\index{venison!Venison~Tenderloin}


\begin{recipestats}[
	servings=2,
	preptime=5 \minute,
	bakingtime=$[10 \ldots 15]~\minute$,
	source=David Henry,
]
\end{recipestats}


\begin{recipeabstract}
	A super simple and wonderfully tasty way to prepare venison tenderloin or back straps that if done right will make anyone who thinks that they don't like venison change their mind.
	Best served with a side of steamed vegetables.
	My personal favorite pairing is spinach and carrots.
\end{recipeabstract}



\begin{ingredientcolumns}[1]
	\begin{ingredientblock}
		\ingredient[1][\pound]{venison tenderloin or backstrap}\\
		\ingredient{steak seasoning}\\
		\ingredient{vegetable oil}
	\end{ingredientblock}
\end{ingredientcolumns}


\ragmarpar{McCormick Montreal Steak seasoning is best.}
\begin{preparation}
\item Slice venison to $\approx~\threefourth''$, season heavily on both sides.

\item Place a cast iron skillet on the stove with oil and bring to medium high heat.

\ragmarpar{Doe meat preferred; younder deer tastes better.}
\item Use one slice of venison to test the temperature, it should sizzle immediately when placed into pan.
	Cook $\approx2~\minute$ per side and remove.
	Cut open center, should be red but warm, outside nicely seared.
	Towards the end of the second side the red juice should pool on the top.
	Adjust time / temperature based on the result of the first slice.

\ragmarpar{Both fresh | thawed work well.}
\item Cook remainder using the same technique, adjusting for different thickness of the slices.

\item Serve immediately after cooking.
	Best if eaten within 5 \minute~after removing from pan.
\end{preparation}


\recipeend%
