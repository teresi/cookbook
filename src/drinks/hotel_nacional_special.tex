\section[Hotel Nacional]{Hotel Nacional Special\ragStars{2}}\index{apricot!Hotel Nacional}\index{rum!Hotel Nacional}


\begin{recipestats}[
	servings=1,
	preptime=5 \minute,
	original=\citeauthor{potionsCaribbeanBerry}~\cite{potionsCaribbeanBerry},
]
\end{recipestats}


\ragmarpar{Use fresh pineapple.}
\begin{recipeabstract}
	Named after a bar, as often is the case, Hotel Nacional de Cuba.
	Similar to a Daiquiri but includes apricot liquer.
\end{recipeabstract}


\ragmarpar{Use lemon instead of lime if using canned pineapple\cite{hotelNacionalColdGlass}.}


\begin{ingredientcolumns}
	\begin{ingredientblock}
		\ingredient[1][\fluidounce]{white rum}\\
		\ingredient[1][\fluidounce]{pineapple juice}
	\end{ingredientblock}

	\begin{ingredientblock}
		\ingredient[\onefourth][\fluidounce]{apricot liqueur}\\
		\ingredient[\onefourth][\fluidounce]{lime juice}
	\end{ingredientblock}
\end{ingredientcolumns}


\begin{preparation}
\item Shake w/ ice and strain into frozen coupe.
\item Garnish w/ lime slice.
\end{preparation}


\begin{experiments}
\item Reformulations abound, need to checkout Smuggler's Cove's.
\end{experiments}


\recipeend%
