\section[SC Hibiscus Infusion]{SC Spiced Hibiscus Infusion}\label{hibiscus_infusion}\index{hibiscus!Hibiscus~Infusion}


\begin{recipestats}[
	servings=16~\fluidounce,
	preptime=5~\minute,
	inactivetime=72~\hour,
	original=\citeauthor{smugglersCove}~\cite[p.~331]{smugglersCove},
]
\end{recipestats}


\ragmarpar{See \cite[p.~197]{smugglersCove} for rum recommendations.}
\begin{recipeabstract}
	A deep magenta tincture of hibiscus and spices.
	Hibiscus has a sour and fruity flavor and known for it's use in Sorrel Punch and Agua de Jamaica.
	Used to make \reffull{hibiscus_liqueur}.
\end{recipeabstract}


\begin{ingredientcolumns}[1]
	\begin{ingredientblock}
		\ingredient[500][ml]{blended lightly aged rum}\\
		\ingredient[\threeeighth][C]{dried hibiscus}\\
		\ingredient[5][whole]{cloves}\\
		\ingredient[2][discs]{ginger}
	\end{ingredientblock}
\end{ingredientcolumns}


\begin{preparation}
\ragmarpar{Use Mount Gay Eclipse.}
\item Slice off quarter sized discs of ginger.
\item Macerate rum, cloves, ginger, for 24 $\hour$ at room temperature.
\item Strain out spices and discard.
\item Macerate with hibiscus for 48 $\hour$.
\item Strain out hibiscus, discard.
\item Transfer to a bottle, keep up to 6 $month$.
\end{preparation}


\recipeend%
