\section[Tom Collins]{Tom Collins\ragStars{1}}
\index{lemon!Tom Collins}
\index{gin!Tom Collins}


\begin{recipestats}[
	servings=1,
	preptime=5 \minute,
	source=Mike \& Jane,
	original=\citefield{heresHow1941}{title} \cite{heresHow1941},
]
\end{recipestats}


\begin{recipeabstract}
	A perfect summer drink, bright and refreshing.
	``Made orginally with Tom Gin, hence the name.'' \cite{heresHow1941}.
\end{recipeabstract}


\begin{ingredientcolumns}
	\begin{ingredientblock}
		\ingredient[1 \onehalf][\fluidounce]{gin, dry}\\
		\ingredient[1][\fluidounce]{lemon juice}
	\end{ingredientblock}
	\begin{ingredientblock}
		\ingredient[2][\teaspoon]{sugar, powdered}\\
		\ingredient[3][\fluidounce]{club soda}
	\end{ingredientblock}
\end{ingredientcolumns}


\begin{preparation}
\item Add to highball, fill halfway with ice, stir, fill will soda.
\end{preparation}


\begin{variation}
\item A ``Sandy Collins'' is done with Scotch, ``Rum Collins'' with rum.
\end{variation}

\begin{experiments}
\item Need to select the right gin, and dial in the sugar / soda.
\end{experiments}


\recipeend
