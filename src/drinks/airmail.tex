\section[Airmail]{Airmail\ragStars{3}}
\index{champagne!Airmail}
\index{rum!Airmail}


\begin{recipestats}[
	servings=1,
	preptime=5~\minute,
	original=\citefield{cocktailSeminars2021}{title} \cite{cocktailSeminars2021},
]
\end{recipestats}


\ragmarpar{Jane and Mike like the Airmail for it's buoyant bubbles.}

\begin{recipeabstract}
	Complex, sweet, and effervescent.
	\citeauthor{cocktailSeminars2021} (the author) recommends the dry shake technique which unlocks the ``aromatic power'' of the honey and produces a light foam.
\end{recipeabstract}


\begin{ingredientcolumns}
	\begin{ingredientblock}
		\ingredient[1\onehalf][\fluidounce]{rum, aged}\\
		\ingredient[\threefourth][\fluidounce]{lime juice}
	\end{ingredientblock}
	\begin{ingredientblock}
		\ingredient[\onehalf][\fluidounce]{honey}\\
		\ingredient[1\onehalf][\fluidounce]{sparkling wine}
	\end{ingredientblock}
\end{ingredientcolumns}
\ragmarpar{Rum $[5 \ldots 12]$ years old is best\cite{cocktailSeminars2021}.}

\begin{preparation}
\item Shake rum, lime, honey, without ice.
\item Shake again with ice.
\item Pour into a coupe and top w/ wine.
\item Garnish with a citrus peel.
\end{preparation}


\begin{experiments}
\item Is Champagne, Processo, or Cava best?
	Processo may be sweeter but less carbonated.
\end{experiments}


\recipeend
