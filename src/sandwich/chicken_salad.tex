\section[Chicken Salad]{Curry Chicken Salad}\index{chicken!Chicken~Salad}


\ragmarpar{Croissant is pronounced ``kwa-son''.}
\begin{recipestats}[
	servings=2,
	preptime=15~\minute,
	source=Mike \& Jane,
	original=Pamela Teresi,
]
\end{recipestats}


\begin{recipeabstract}
	Chicken salad on a croissant.
	The mild Japanese curry powder provides a nice contrast to the sweet \& savory ingredients.
\end{recipeabstract}


\begin{ingredientcolumns}
	\begin{ingredientblock}
		\ingredient[4]{croissant}\\
		\ingredient[2][C]{chicken, chopped}\\
		\ingredient[\onehalf][C]{mayo}\\
		\ingredient[\onehalf][C]{celery, minced}\\
		\ingredient[\onefourth][C]{relish}\\
		\ingredient[\onefourth][C]{onion, minced}
	\end{ingredientblock}

	\begin{ingredientblock}
		\ingredient[\onefourth][C]{peanuts, chopped}\\
		\ingredient[1][\Tablespoon]{S\&B Curry powder}\\
		\ingredient[1][\teaspoon]{salt}\\
		\ingredient[1][\teaspoon]{white pepper}\\
		\ingredient[1][dash]{honey}\\
		\ingredient[1][dash]{olive oil}
	\end{ingredientblock}
\end{ingredientcolumns}


\ragmarpar{Mom recommends sweet relish.}
\begin{preparation}
	\item Mince and combine the celery, onion.
	\item Chop and add the cooked chicken, peanuts.
	\item Mix in condiments, spices.
	\item Slice croissant in half, toast in the oven briefly.
	\item Top croissant with chicken salad, serve.
\end{preparation}


\recipeend%
