\section[Green Beans]{Garlic Green Beans}\index{beans!Garlic~Green~Beans}


\begin{recipestats}[
	servings=2,
	preptime=10~\minute,
	bakingtime=5~\minute,
	source=Mike Teresi,
	original=Pamela Teresi
]
\end{recipestats}

\begin{recipeabstract}
	An old standby.
	The fresh garlic and pepper are the key.
\end{recipeabstract}


\ragmarpar{Don't overcook! We likes it fresh and snappy.}
\begin{ingredientcolumns}
	\begin{ingredientblock}
		\ingredient[1][\pound]{green beans}\\
		\ingredient[3][clove]{garlic}
	\end{ingredientblock}
	\begin{ingredientblock}
		\ingredient[1][\Tablespoon]{olive oil}\\
		\ingredient[2][\teaspoon]{black pepper}\\
		\ingredient[2][\teaspoon]{salt}
	\end{ingredientblock}
\end{ingredientcolumns}


\begin{preparation}
\item Heat $\approx 1 \onehalf$ quart water on high to a boil.
\item Meanwhile, rinse beans, cut off ends, remove paper from cloves.
\item Once boiled, add salt and beans, stir.
	Boil for 4 \minute.
\ragmarpar{About 8 turns of pepper should do it.}
\item Drain and return to burner.
	Combine with olive oil, crushed garlic, and pepper.
\item Saut\'{e} briefly while stirring.
\end{preparation}


\recipeend%
