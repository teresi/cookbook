\section[Quick Pickles]{Basic Quick Pickles}\label{quick_pickles}\label{basic_quick_pickles}


\begin{recipestats}[
	servings=16 \fluidounce,
	preptime=30 \minute,
	bakingtime=10 \minute,
	source=Teresi Family,
]
\end{recipestats}


\begin{recipeabstract}
	These pickles are ``quick'' since they don't require canning.
	They are stored in the fridge as a result.
	We like to make jalepe\~{n}os and red onions, both with plenty of garlic.
\end{recipeabstract}


\ragmarpar{Prefer un-iodized salt to prevent discoloration.}
\begin{ingredientcolumns}
	\begin{ingredientblock}
		\ingredient[1][16 \ounce]{canning jar}\\
		\ingredient[\threefourth][C]{vinegar}\\
		\ingredient[\onehalf][C]{water}\\
		\ingredient[1][\Tablespoon]{sugar}\\
		\ingredient[1][\Tablespoon]{salt, kosher}\\
		\ingredient[1][]{onion, red}\\
		\ingredient[1][]{jalepe\~{n}o}\\
		\ingredient[2][]{carrots}\\
		\ingredient[2][]{radishes}
	\end{ingredientblock}

	\bigskip
	\begin{ingredientblock}[spices]
		\ingredient[4][cloves]{garlic}\\
		\ingredient[\onehalf][\teaspoon]{oregano}\\
		\ingredient[\onehalf][\teaspoon]{cumin seeds}\\
		\ingredient[\onehalf][\teaspoon]{mustard seeds}\\
		\ingredient[\onehalf][\teaspoon]{peppercorns}
	\end{ingredientblock}
\end{ingredientcolumns}


\ragmarpar{We prefer black or white peppercorns over pink.}
\begin{preparation}
	\item Begin heating the water, vinegar, sugar, salt on low.
	\item Smash garlic, peel, cut off ends, slice.
		Add spices to the jar.
	\item Slice vegetables and pack into the jar.
	\item Bring brine to a simmer.
	\item Pour into jars leaving about a centimeter of head space, loosely tighten lid.
		Let cool on the counter at least until lukewarm, tighten.
	\item Put in refrigerator and let cool overnight.
		The general rule of thumb is to keep about a month but your mileage may vary.
\end{preparation}


\begin{variation}
	\item Other vinegar, spices, vegetables, etc. are available.
\end{variation}


\begin{experiments}
	\item Large spices that float are not recommended, such as coriander or Brazilian pepper corns.
		It can be unpleasant to bite into and is difficult to separate from the vegetables.
		Mustard seeds and dried herbs do not have this issue.
	\item Still need more experiments to find the right measurements for the spices.
		I am not convinced they make much a difference.
\end{experiments}

\recipeend%
