\section[Ancho Rub]{Ancho Chicken Rub\ragStars{3}}


\begin{recipestats}[
	servings=10 \fluidounce,
	bakingtime=10 \minute,
	preptime=30 \minute,
	inactivetime=$[3 \ldots 12]~\hour$,
	source=Mike \& Jane,
]
\end{recipestats}


\begin{recipeabstract}
	A smokey chicken rub perfect for barbeque.
	The garlic helps hold the rub together, and the chili comes through strong.
\end{recipeabstract}
\ragmarpar{Good for about 1~\onefourth~\pound\ chicken thighs.}


\begin{ingredientcolumns}
	\begin{ingredientblock}[rub]
		\ingredient[\onefourth][C]{chili powder}\\
		\ingredient[5][cloves]{garlic}\\
		\ingredient[1][\Tablespoon]{kosher salt}\\
		\ingredient[1][\Tablespoon]{brown sugar}\\
		\ingredient[1][\Tablespoon]{cumin}\\
		\ingredient[1][\Tablespoon]{onion powder}\\
		\ingredient[\onehalf][\Tablespoon]{achiote powder}
	\end{ingredientblock}
	\begin{ingredientblock}[chili powder]
		\ingredient[5]{Ancho}\\
		\ingredient[2]{Pasilla}\\
		\ingredient[2]{Gaujillo}
	\end{ingredientblock}
\end{ingredientcolumns}


\begin{preparation}
\ragmarpar{Chilies are done when they turn crispy after cooling.}
\item Rinse chilies and let dry.
	Slice open, remove stems, seeds, veins.

\ragmarpar{The red chilies will toast faster than the black.}
\item Toast the chilis in a 325~\Fahrenheit~$\approx[5..10]~\minute$.
	Remove and cool.

\item Grind chilies to a powder.

\item Combine the rub ingredients in a food processor.
	Refrigerate rub and use within a week.
\end{preparation}


\begin{experiments}
\item Still optimizing the chili combination.
	Probably don't need smoked chilis when smoking the chicken.

\item Need to weigh out the garlic for consistency.
\end{experiments}

\recipeend%
