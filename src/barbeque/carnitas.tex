\section[Carnitas]{Carnitas, or Little Meats}%\index{pork!Carnitas}


\begin{recipestats}[
    servings=6,
    preptime=45~\minute,
    bakingtime=6~\hour,
    source=Mike \& Jane,
    original=Rick~\cite{baylessCarnitas} \& Rick~\cite{martinezCarnitas},
]
\end{recipestats}


\ragmarpar{Mexican oregano is best.}
\begin{recipeabstract}
    Pork confit.
    Pork shoulder is simmered, shredded, then fried.
    Bayless provides a few methods to recreate the dish in lieu of a copper cauldron filled with lard.
    The slow cooker method is adapted here, which allows for some spice as frying in lard would otherwise `obliterate' it.
    Spices are usually either just salt and pepper or cinnamon, citrus, oregano, and the like.
    Martinez recommends Canela cinnamon which has a sweeter flavor than Cassia.
    People often include ingredients to encourage browning such as cola, citrus, condensed milk, or sugar.
\end{recipeabstract}
\ragmarpar{Lightly crack the pepper with the bottom of a pot.}


\begin{ingredientcolumns}
    \begin{ingredientblock}[pot]
        \ingredient[4~\onehalf][\pound]{pork shoulder}\\
        \ingredient[1][\Tablespoon]{salt}\\
        \ingredient[1][C]{lard}\\
        \ingredient[\onehalf][]{white onion}\\
        \ingredient[1][]{orange}\\
        \ingredient[12][\fluidounce]{Negra Modelo}\\
        \ingredient[1][C]{stock}\\
        \ingredient[2][\Tablespoon]{brown sugar}
    \end{ingredientblock}
    \begin{ingredientblock}[sachet]
        \ingredient[2][\teaspoon]{oregano}\\
        \ingredient[2][\teaspoon]{peppercorns}\\
        \ingredient[1][\teaspoon]{chili flakes}\\
        \ingredient[2][leaf]{bay}\\
        \ingredient[2][stick]{canela}\\
        \ingredient[1][head]{garlic}
    \end{ingredientblock}
\end{ingredientcolumns}
\ragmarpar{The Ricks recommend Pico de Gallo or Guacamole.}


\begin{preparation}
\item Reserve a large dutch oven \& iron skillet, preheat to $250$ $\Fahrenheit$.
\item Debone pork shoulder, cut into large chunks, salt, set aside.
\item Cut the onion and orange in half, remove seeds.
\item Cut the garlic head in half, tie up spices in cheese cloth.
\item Melt the sugar in the stock.
\item Squeeze oranges, add juice to stock, keep the peel.
\item Melt the lard in the pot, lightly brown the pork in batches.
\item Turn off heat, add onion, peel, pork, beer, stock, sachet.
\item Roast covered $\approx5$ $\hour$, stirring occasionally, to 207 $\Fahrenheit$.
\item Lower to $225$ $\Fahrenheit$, roast $\ge$ 30 $\minute$ or until ready to eat.
\item Remove the orange peel.
\item Transfer pieces to a cutting board, chop.
\item Broil or fry the pork in the skillet with some of the juice until it browns, serve.
\end{preparation}


\recipeend%
