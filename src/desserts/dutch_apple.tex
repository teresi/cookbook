\section{Dutch Apple Pie}\index{apple!Dutch~Apple~Pie}


\begin{recipestats}[
	servings=10,
	preptime=45~\minute,
	bakingtime=45~\minute,
	original=Jean Nelson,
]
\end{recipestats}


\ragmarpar{About 6 apples.}
\begin{recipeabstract}
    A Giesken Thanksgiving tradition.
    Baked in the special ceramic oversized pie dish.
\end{recipeabstract}


\ragmarpar{Use a lot of apples, pile them up high.}
\begin{ingredientcolumns}
	\begin{ingredientblock}[filling]
		\ingredient[1][9~\inch]{pie crust}\\
		\ingredient[5~..~6][C]{apples}\\
		\ingredient[\onefourth~..~\threefourth][C]{sugar}\\
		\ingredient[\onehalf][\teaspoon]{cinnamon}\\
		\ingredient[\onefourth][\teaspoon]{nutmeg}
	\end{ingredientblock}

	\bigskip
	\begin{ingredientblock}[topping]
		\ingredient[\threefourth][C]{flour}\\
		\ingredient[\onehalf][\teaspoon]{cinnamon}\\
		\ingredient[\onehalf][C]{sugar}\\
		\ingredient[\onehalf][C]{butter}
	\end{ingredientblock}
\end{ingredientcolumns}


\ragmarpar{Use a mix of apples, e.g. Granny Smith, Honeycrisp...}
\begin{preparation}
\item Preheat oven to 425 $\Fahrenheit$.
\item Peel and slice apples into a big bowl.
\item Mix sugar and spices for the filling, mix into apples.
\item Add raw crust to pie dish, add filling.
\item Mix dry ingredients for topping, spread over top of pie.
\item Cut up butter and spread over top.
\item Bake for 45 $\minute$.
\end{preparation}


\recipeend%
