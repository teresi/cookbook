\section{Basic Flaky Pie Pastry}\label{basicFlakyPiePastry}\index{pastry!Basic~Flaky~Pastry}


\begin{recipestats}[
	servings=1 pastry,
	preptime=45 \minute,
	inactivetime=1 \hour,
	source=\citefield{pie2004}{title}~\cite{pie2004},
	]
\end{recipestats}


\begin{recipeabstract}
	An all purpose pie pastry.
	Very useful to make in large batches, freezing up to a month.
\end{recipeabstract}


\begin{ingredientcolumns}
	\begin{ingredientblock}
		\ingredient[1 \onehalf][C]{flour, all purpose}\\
		\ingredient[1 \onehalf][\teaspoon]{sugar}\\
		\ingredient[\onehalf][\teaspoon]{salt}
	\end{ingredientblock}
	\begin{ingredientblock}
		\ingredient[\onefourth][C]{butter, unsalted}\\
		\ingredient[\onefourth][C]{shortening}\\
		\ingredient[\onefourth][C]{water}
	\end{ingredientblock}
\end{ingredientcolumns}


\begin{preparation}
\item Cut fat into small pieces ($\approx 3/8''$ cubes), place in freezer briefly along with water until cold.

\item Mix flour, sugar, salt, butter, in a large bowl.
	Blend using a pastry cutter $\|$ fork $\|$ fingers, until the butter is pea sized.
	Blend the shortening similarly.

\item Add half the water and toss with fork.
	Add water $\approx 1.5 \dots 2$ \Tablespoon~at a time, and pull all the flour into the dough.
	Continue until the dough can be packed together.

\item Pack dough into a ball, knead once or twice.
	Flatten onto a floured surface into $\approx \threefourth''$ disks.
	Wrap in plastic and refrigerate at least 1 \hour~or overnight.

\ragmarpar{Freezing before rolling results in cracking}
\item Roll onto wax paper, invert onto pan \& shape.
	Freeze 15 \minute.

\item For a pre-baked crust: preheat to 400 \Fahrenheit, press aluminum foil on top of pastry and fill with pie weights.
	Bake 15 \minute, remove foil \& weights, prick holes into pastry base with fork to prevent bubbles, lower to 375 \Fahrenheit.

\item For a partially pre-baked crust: bake $[10..12]$ \minute~at 375 \Fahrenheit.

\item For a fully pre-baked crust: bake $[15..17]$ \minute~at 375 \Fahrenheit.
\end{preparation}


\recipeend%
