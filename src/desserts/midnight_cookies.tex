\section{Frankie Cookies}\index{chocolate!Frankie~Cookies}\index{mint!Frankie~Cookies}\index{cookies!Frankie~Cookies}


\begin{recipestats}[
	servings=20 cookies,
	preptime=30 \minute,
	bakingtime=11 \minute,
	source=Mike \& Jane,
	original=\citetitle{entenmanns2011}~\cite{entenmanns2011},
]
\end{recipestats}


\begin{recipeabstract}
	A chocolate cookie with a peppermint pattie center.
	Simple and quick.
	Aliased for our black and white cats, Felix and Frankie.
\end{recipeabstract}


\ragmarpar{Each peppermint pattie is $\approx 8~grams$.}
\begin{ingredientcolumns}
	\begin{ingredientblock}
		\ingredient[\onehalf][C]{butter, salted}\\
		\ingredient[1][C]{sugar}\\
		\ingredient[1][large]{egg}\\
		\ingredient[\onehalf][\teaspoon]{vanilla}
	\end{ingredientblock}
	\begin{ingredientblock}
		\ingredient[1][C]{flour}\\
		\ingredient[\onethird][C]{cocoa powder}\\
		\ingredient[\onehalf][\teaspoon]{baking soda}\\
		\ingredient[12][thin]{peppermint patties}
	\end{ingredientblock}
\end{ingredientcolumns}


\begin{preparation}
\item Remove butter \& eggs from refrigerator.
	Soften the butter.
	Reserve two cookie sheets and wipe with Crisco.

\item Begin to preheat oven to 350 \Fahrenheit.
	Place one more empty cookie sheet on bottom rack, move second rack to middle.

\item Add butter, sugar to a mixing bowl.
	Mix flour, cocoa, baking soda to another bowl.

\ragmarpar{Make sure the edges are covered to prevent leaks.}
\item Cream butter \& sugar on medium high $\approx 2$ \minute.
	Add egg \& vanilla, mix $\approx 1$ \minute.
	Add dry ingredients, mix $\approx 2$ \minute.

\item Assemble the cookies in batches.
	Take two portions $\approx 1$ \Tablespoon~each and enclose a peppermint pattie.
	Top with sprinkles.
	Arrange on the sheet, refrigerate $\approx 15$ \minute, then bake 11 \minute.
\ragmarpar{The cookies are done after the pattie melts and spreads out.}
\end{preparation}


\begin{variation}
\item The cookies work well on their own without the peppermint, too.
\end{variation}

\begin{experiments}
\item Dutch process cocoa would be interesting, but would require a change in leavening as it's not as acidic.
\end{experiments}

\recipeend%
