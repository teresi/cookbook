\section[Red No. 3]{Red No. 3}


\begin{recipestats}[
	servings=12 \fluidounce,
	preptime=1 \hour,
	bakingtime=4 \week,
	source=Teresi Family,
	]
\end{recipestats}


\begin{ingredientcolumns}[1]
	\begin{ingredientblock}
		\ingredient[8]{fresno}\\
		\ingredient[6]{cherry}\\
		\ingredient[6]{thai}\\
		\ingredient[3]{guajillo}\\
		\ingredient[3]{arbol}
	\end{ingredientblock}
	\begin{ingredientblock}
		\ingredient[1]{shallot}\\
		\ingredient[1][\teaspoon]{Indian green pepper}\\
		\ingredient[\onefourth]{oil}\\
		\ingredient[\onefourth][\teaspoon]{achiote}
	\end{ingredientblock}
\end{ingredientcolumns}


\begin{preparation}
\item Follow~\ref{base_lacto_brine_hotsauce}~\nameref{base_lacto_brine_hotsauce} with a $5.3\%$ brine.

\item Ferment 2 \week, refridgerate 2 \week, blend with \onehalf~$C$ brine and white vinegar each, strain.
\end{preparation}


\begin{experiments}
	\item Look into using xantham gum to keep the emulsion in suspension longer
	\item Consider rinsing / re-hydrating the dried chilies
	\item Still needs more heat
	\item Consider adding some cabbage to help fermentation
\end{experiments}


\recipeend%
