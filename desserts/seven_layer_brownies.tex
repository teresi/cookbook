\begin{recipe}[
source = Rand Pearson
]
{7-Layer Brownies}

\ingredients {
\unit[1]{box} & Betty Crocker brownie/cookie combo mix\\
\unit[4]{Tbs} & butter\\
\unit[4\dots 6]{} & Heath bars\\
\unit[]{} & marshmallows\\
\unit[]{} & graham crackers\\
\unit[]{} & chocolate chips\\
\unit[]{} & peanut butter\\
}

\preparation {
\step Mix up the brownie and cookie mix as instructed on the back of the box.

\step Place brownie mix in a baking dish as instructed.

\step Corsely chop Heath bar and spead evenly over brownie batter.

\step Place cookie dough mix on top of Heath bar layer.

\step Crush graham crackers and spread over cookie dough.

\step Melt half a stick of butter and pour over graham crackers.

\step If using chocolate chips spread a layer over graham crackers. If using peanut butter melt it and pour over graham cracker.

\step Split marxhmallows lengthwise and arrange a asolid layer over the top of everything. You might also use mini marshamallows to skip cutting them.

\step Bake as instructed on brownie box. Time may increase due to the extra layers. Brownies are done when a toothpick comes out clean.
}

\hint{
\begin{itemize}
\item Keep in mind that there really isn't a recipe since I just made it all up as I went along. What you should take from that is this: Feel free to experiment; any problems can be overcome with enough butter and sugar. I'm pretty sure this is also true in life.
\item In the future I was considering leaving the marshmallows off until I take the brownies out of the oven, then adding them ad hitting them with a torch.
\end{itemize}
}

\end{recipe}
