\section{Basic Quick Pickles}

\begin{recipestats}[
	servings=16 \fluidounce,
	preptime=15 minutes,
	bakingtime=10 minutes (+1 day chill),
%	source=,  % TODO add dad's source
]
\end{recipestats}

\begin{recipeabstract}
	These pickles are ``quick'' since they don't require canning.
	They are stored in the fridge as a result.
	We like to make jalepe\~{n}os and red onions, both with plenty of garlic.
\end{recipeabstract}

\begin{ingredientcolumns}
	\begin{ingredientblock}
		\ingredient[1][16 \ounce]{canning jar}\\
		\ingredient[1][C]{water}\\
		\ingredient[1][C]{vinegar}\\
		\ingredient[2][\Tablespoon]{sugar}\\
		\ingredient[2][\Tablespoon]{salt}
	\end{ingredientblock}
	\begin{ingredientblock}
		\ingredient[4][cloves]{garlic}\\
		\ingredient[\onehalf][\teaspoon]{oregano}\\
		\ingredient[\onehalf][\teaspoon]{cumin seeds}\\
		\ingredient[\onehalf][\teaspoon]{peppercorns}
	\end{ingredientblock}
\end{ingredientcolumns}

\ragmarpar{we prefer black or white peppercorns over pink}
\begin{preparation}
	\item Smash garlic, peel, and cut off the woody end. Add spices to the bottom of the jar.
	\item Slice vegetables and pack into the jar.
	\item Simmer the water, vinegar, salt, sugar. Let cool $\approx5$ $min$.
	\item Pour into jars leaving about a centimeter of head space and tighten lid loosely. Let cool on the counter at least until lukewarm, tighten lid.
	\item Put in refrigerator and let cool overnight. The general rule of thumb is to keep about a month but your mileage may vary.
\end{preparation}

\begin{variation}
	\item Your choice of vegetables, e.g. red onions, jalepe\~{n}os, radishes, cucumbers.
\end{variation}

\recipeend