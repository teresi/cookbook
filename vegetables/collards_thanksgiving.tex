\begin{recipe}[
    preparationtime = 30 minutes,
    bakingtime = 2...6 hours,
    portion = 4...6 people,
]
{Collard Greens, Thanksgiving}
\graph{
    %small = strawberry,
    smallpicturewidth = 0.3\textwidth,
    %big = strawberrycake,
    bigpicturewidth = 0.6\textwidth,
}
%%%%%%%%%%%%%%%%%%%%%%%%%%%%%%%%%%%%%%%%%%%%%%%%%%%%%%
\ingredients{
\unit[1]{lb} & Collard Greens\\
\unit[2]{} & turkey necks, smoked\\
\unit[6]{C} & water\\
\unit[2]{Tbs.} & butter\\
\unit[2]{Tbs.} & vinegar\\
\unit[1]{tsp.} & sugar\\
\unit[1]{tsp.} & pepper\\
\unit[1]{tsp.} & hot sauce
}
%%%%%%%%%%%%%%%%%%%%%%%%%%%%%%%%%%%%%%%%%%%%%%%%%%%%%%
\preparation{
\step Make stock with turkey necks: cover with cold water, cook over high heat, bring to simmer, skim off foam from surface, reduce to $\sim 180 \; F^\circ$. Simmer for $1...4 \; hr$.

\step Prepare collards greens ($\sim 15 \; min$ prior to removing turkey necks). Clean \& rinse collard greens, cut off the main stem, dice $\|$ tear greens (into $\sim 1x1"$ rectangles).

\step Remove turkey necks and sieve out undesired particulates.

\step Add greens, simmer $\sim 1 \; hr$ uncovered.

\step Meanwhile, remove and dice meat from turkey necks according to your preference. Add to pot.

\step Add butter, vinegar, spices, serve.


}
%%%%%%%%%%%%%%%%%%%%%%%%%%%%%%%%%%%%%%%%%%%%%%%%%%%%%%
\hint{
\begin{itemize}
\item Substitute a ham hock for the turkey necks for a more traditional dish.
\item Consider adding garlic, bay leaf, and/or bacon after the stock is complete.
\end{itemize}
}
%%%%%%%%%%%%%%%%%%%%%%%%%%%%%%%%%%%%%%%%%%%%%%%%%%%%%%
\setRecipeLengths{
preparationwidth = 0.60\textwidth,
ingredientswidth = 0.35\textwidth,
pictureheight = 6cm,
bigpicturewidth = 0.6\textwidth,
smallpicturewidth = 0.35\textwidth
}
%%%%%%%%%%%%%%%%%%%%%%%%%%%%%%%%%%%%%%%%%%%%%%%%%%%%%%
\setRecipeSizes{
recipename = \fontsize{25pt}{30pt},
ing = \normalsize,
inghead = \normalsize,
prep = \normalsize,
prephead = \normalsize,
hint = \normalsize,
hinthead = \Large
}
%%%%%%%%%%%%%%%%%%%%%%%%%%%%%%%%%%%%%%%%%%%%%%%%%%%%%%
%\setRecipenameFont{
%%pbsi
%%fau
%%fwb
%%fjd % default when using the option handwritten
%cmr % default
%}{T1}{m}{n}
%%%%%%%%%%%%%%%%%%%%%%%%%%%%%%%%%%%%%%%%%%%%%%%%%%%%%%
\setHeadlines{
inghead = Ingredients,
prephead = Preparation,
hinthead = Hint,
calory = Cal,
continuationhead = Continuation,
continuationfoot = Continuation on next page
}
%%%%%%%%%%%%%%%%%%%%%%%%%%%%%%%%%%%%%%%%%%%%%%%%%%%%%%
%\setBackgroundPicture
%[%
%x = 2cm,
%y = -1cm,
%width=\paperwidth-3cm,
%height,
%orientation=pagecenter
%]{pic/bg_transparent} % filepath
\end{recipe}

%\clearpage
%\thispagestyle{empty}
% \begin{tikzpicture}[remember picture,overlay]
   % inelegant way of getting a good image:
   % use [keepaspectratio], trim to an aspect ratio close to the page, then remove [keepaspectraio] to get full page
%   \node at (current page.center) {\includegraphics[width=\pdfpagewidth,height=\pdfpageheight,clip,trim={47px 10px 47px 10px}]{tricolor}};
   % IMAGE: http://www.homemadeitaliancooking.com/italian-rainbow-cookies/
%\end{tikzpicture}