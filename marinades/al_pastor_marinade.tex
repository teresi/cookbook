\section[Marinade Al Pastor]{Marinade Al Pastor~\vegan}\label{marinade_al_pastor}
\index{chili!Marinade~Al~Pastor}
\index{pineapple!Marinade~Al~Pastor}


\begin{recipestats}[
	servings=3 \pound~meat,
	preptime=15 \minute,
	inactivetime=$[3 \ldots 12]~\hour$,
	source=Mike \& Jane,
	original=\citeauthor{rMartinezTacos2015} \cite{rMartinezTacos2015},
]
\end{recipestats}


\begin{recipeabstract}
	Shepherd's style tacos features a mix of Mexican \& Lebanese spices.
	The marinade plays a large role in the dish as the meat is sliced very thin.
	It has a strong red color from the achiote paste and chile.
\end{recipeabstract}

\ragmarpar{Morita chilis also work well}
\begin{ingredientcolumns}
	\begin{ingredientblock}
		\ingredient[6]{guajillo}\\
		\ingredient[3]{chili de \`{a}rbol}\\
		\ingredient[2]{ancho}\\
		\ingredient[2][\ounce]{garlic}\\
		\ingredient[3][\ounce]{salt}\\
		\ingredient[1][C]{white vinegar}
	\end{ingredientblock}
	\begin{ingredientblock}
		\ingredient[\onefourth][C]{sugar}\\
		\ingredient[3][\Tablespoon]{achiote paste \ref{achiote_paste_RM}}\\
		\ingredient[10][\ounce]{pineapple, sliced}\\
		\ingredient[1]{orange, juiced}\\
		\ingredient[1]{lime, juiced}\\
		\ingredient[\onehalf]{onion, white}
	\end{ingredientblock}
\end{ingredientcolumns}


\begin{preparation}
\item Remove stems, veins, and seeds.
	Rinse off exterior with water.

\item Place chiles in a large bowl and submerge with boiling water.
	Cover and wait 30 \minute.

\ragmarpar{Half of a 20 $\ounce$ can of pinneapple is perfect.}
\item Meanwhile prepare remaining ingredients \& add to a blender.
	Smash and peel garlic.
	Juice the orange, lime.
	Add pineapple slices and juice.
	Add onion roughly chopped.

\item Add chiles \& 1 $C$ soaking liquid to the blender when done.
	Blend on high for a few minutes.

\item Marinade meat $\approx[3\ldots12]~\hour$.
\end{preparation}


\begin{variation}
\item There is a lot of leeway for chile selection.
	Guajillo seems to be common, as well as chipotle in adobo.
	The morita / chipotle provide a smoky flavor.
\end{variation}


\recipeend
