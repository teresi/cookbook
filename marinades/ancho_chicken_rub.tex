\section[Ancho Rub]{Ancho Chicken Rub~\vegan}


\begin{recipestats}[
	servings=10 \fluidounce,
	preptime=30 \minute,
	source=Mike \& Jane,
]
\end{recipestats}


\begin{recipeabstract}
	A smokey chicken rub perfect for barbeque.
	The garlic helps hold the rub together, and the chili comes through strong.
\end{recipeabstract}
\ragmarpar{Good for about 2 \onehalf~\pound~chicken thighs.}


\begin{ingredientcolumns}
	\begin{ingredientblock}[rub]
		\ingredient[5][cloves]{garlic}\\
		\ingredient[\onefourth][C]{chili powder}\\
		\ingredient[2][\Tablespoon]{salt}\\
		\ingredient[1][\Tablespoon]{brown sugar}\\
		\ingredient[1][\Tablespoon]{cumin}\\
		\ingredient[1][\Tablespoon]{onion powder}\\
		\ingredient[\onehalf][\Tablespoon]{achiote}
	\end{ingredientblock}
	\begin{ingredientblock}[chili powder]
		\ingredient[5]{Ancho}\\
		\ingredient[5]{Morita}\\
		\ingredient[5]{Pasilla}\\
		\ingredient[3]{Gaujillo}
	\end{ingredientblock}
\end{ingredientcolumns}


\begin{preparation}
\ragmarpar{Chilis are done when they turn crispy after cooling.}
\item Rinse and pat dry the chilis.
	Toast the chilis in a 325~\Fahrenheit $\approx5~\minute$.
	Remove and let cool.

\item Break chilis apart, grind in spice grinder to a powder.

\item Combine the rub ingredients in a food processsor.
	Refridgerate rub and use within a week.
\end{preparation}


\begin{experiments}
\item Still optimizing the chili combination.
	Should only really need half of the chilis used.
	Morita might not be necessary since I usually smoke when barbequing.

\item Need to weigh out the chilis for consistency.

\item Need to get a better or faster way to process the chilis.
	Maybe rehydrate them instead and use as a marinade?
\end{experiments}

\recipeend
