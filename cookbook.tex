% family cookbook using 'xcookybooky'
% authors: Michael Teresi

\documentclass[]{article}

\usepackage[T1]{fontenc}
\usepackage[latin1]{inputenc}
\usepackage{lmodern}
\usepackage[ngerman, english]{babel}
\usepackage{tikz} % for creating the lines for the hint
\usepackage{graphicx}
\usepackage{xcolor}
%\usepackage[clock, misc, weather]{ifsym} % Symbolpackage e.g. \Interval, \Wecker etc
\usepackage{cookingsymbols} % Cookings symbols e.g. \Oven, \Dish etc
\usepackage{wrapfig} % To wrap the tabular with the ingredients
\usepackage{ifthen}
\usepackage{xkeyval}
\usepackage{lettrine} % numbering the preparation steps
\usepackage{fancyhdr}
\usepackage{units}
\usepackage{eso-pic} % for background pictures
\usepackage{picture} % for modfifying the position of the bg pictures
\usepackage{tabularx} % line breaks in tabular
\usepackage{tikz}
\usepackage{nicefrac}
\usepackage{amssymb}
\frenchspacing

\usepackage[nowarnings]{xcookybooky}

\usepackage{blindtext}    % only needed for generating test text

\DeclareRobustCommand{\textcelcius}{\ensuremath{^{\circ}\mathrm{C}}}
\DeclareRobustCommand{\textfahrenheiht}{\ensuremath{^{\circ}\mathrm{C}}}

\setcounter{secnumdepth}{2}
\renewcommand*{\recipesection}[2][]
{%
    \subsection[#1]{#2}
}
\renewcommand{\subsectionmark}[1]
{% no implementation to display the section name instead
}

\usepackage{hyperref}    % must be the last package
\hypersetup{%
    pdfauthor            = {Michael Teresi},
    pdftitle             = {Teresi Family Cookbook},
    pdfsubject           = {Recipes},
    pdfkeywords          = {recipes, recipe, cookbook, Teresi},
    pdfstartview         = {FitV},
    pdfview              = {FitH},
    pdfpagemode          = {UseNone}, % Options; UseNone, UseOutlines
    bookmarksopen        = {true},
    pdfpagetransition    = {Glitter},
    colorlinks           = {true},
    linkcolor            = {magenta},
    urlcolor             = {blue},
    citecolor            = {cyan},
    filecolor            = {black},
}

\hbadness=10000	% Ignore underfull boxes

\graphicspath{{./desserts/img/}}

\begin{document}

\title{Teresi Family Cookbook}
\author{Michael Teresi}
\maketitle

\tableofcontents

% FEATURE make macro or find package to auto-crop an roi to make a full page image w/o a border
%	probably use `adjustbox` package
% FEATURE make collage (matrix) of different recipes, make the images links
% FEATURE redifine the ingredients so it can be a 3 column table with the middle as the unit
% TODO bibliography
% TODO standardize Tablespoon --> Tbs, teaspoon --> tps
% TODO standardize minutes / min --> minutes

\clearpage
\section{Soups}
\begin{recipe}[
    preparationtime = 10 minutes,
    bakingtime = 4 hours 20 minutes,
    portion = 3 quarts (4...6 people),
    source = Pamela Teresi
]
{Turkey Gumbo}
\graph{
    %small = strawberry,
    smallpicturewidth = 0.3\textwidth,
    %big = strawberrycake,
    bigpicturewidth = 0.6\textwidth,
}
%%%%%%%%%%%%%%%%%%%%%%%%%%%%%%%%%%%%%%%%%%%%%%%%%%%%%%
\ingredients{
\unit[1]{lb} & sausage, smoked\\
\unit[2]{C} & turkey\\
\unit[1/4]{C} & vegetable oil\\
\unit[1/4]{C} & flour\\
\unit[5]{stalks} & celery\\
\unit[2]{medium} & onions\\
\unit[2]{C} & turkey\\
\unit[4]{C} & stock $\|$ broth\\
\unit[1]{tsp} & gumbo fil\'{e}\\
\unit[1/2]{tsp} & salt\\
\unit[2 1/2]{C} & cooked rice
}
%%%%%%%%%%%%%%%%%%%%%%%%%%%%%%%%%%%%%%%%%%%%%%%%%%%%%%
\preparation{
\step Brown sausage, dice vegetables \& turkey, set aside.

\step Make roux: combine flour \& oil, stirring constantly on medium heat, $10...15\; min$ or until copper in color.

\step Add celery \& onions to roux, cook $5\; minutes$.

\step Add turkey, sausage, stock. Simmer $\sim 4\; hours$.

\step Meanwhile, prepare rice to serve when gumbo is complete.

\step Add gumbo fil\'{e}, salt, pepper. Serve over rice.
}
%%%%%%%%%%%%%%%%%%%%%%%%%%%%%%%%%%%%%%%%%%%%%%%%%%%%%%
\hint{
\begin{itemize}
\item Andouille sausage works particularly well.
\item Serve with hot sauce \& yeast rolls.
\item Traditionally served the day after Thanksgiving.
\end{itemize}
}
%%%%%%%%%%%%%%%%%%%%%%%%%%%%%%%%%%%%%%%%%%%%%%%%%%%%%%
\setRecipeLengths{
preparationwidth = 0.60\textwidth,
ingredientswidth = 0.35\textwidth,
pictureheight = 6cm,
bigpicturewidth = 0.6\textwidth,
smallpicturewidth = 0.35\textwidth
}
%%%%%%%%%%%%%%%%%%%%%%%%%%%%%%%%%%%%%%%%%%%%%%%%%%%%%%
\setRecipeSizes{
recipename = \fontsize{25pt}{30pt},
ing = \normalsize,
inghead = \normalsize,
prep = \normalsize,
prephead = \normalsize,
hint = \normalsize,
hinthead = \Large
}
%%%%%%%%%%%%%%%%%%%%%%%%%%%%%%%%%%%%%%%%%%%%%%%%%%%%%%
%\setRecipenameFont{
%%pbsi
%%fau
%%fwb
%%fjd % default when using the option handwritten
%cmr % default
%}{T1}{m}{n}
%%%%%%%%%%%%%%%%%%%%%%%%%%%%%%%%%%%%%%%%%%%%%%%%%%%%%%
\setHeadlines{
inghead = Ingredients,
prephead = Preparation,
hinthead = Hint,
calory = Cal,
continuationhead = Continuation,
continuationfoot = Continuation on next page
}
%%%%%%%%%%%%%%%%%%%%%%%%%%%%%%%%%%%%%%%%%%%%%%%%%%%%%%
%\setBackgroundPicture
%[%
%x = 2cm,
%y = -1cm,
%width=\paperwidth-3cm,
%height,
%orientation=pagecenter
%]{pic/bg_transparent} % filepath
\end{recipe}

%\clearpage
%\thispagestyle{empty}
% \begin{tikzpicture}[remember picture,overlay]
   % inelegant way of getting a good image:
   % use [keepaspectratio], trim to an aspect ratio close to the page, then remove [keepaspectraio] to get full page
%   \node at (current page.center) {\includegraphics[width=\pdfpagewidth,height=\pdfpageheight,clip,trim={47px 10px 47px 10px}]{tricolor}};
   % IMAGE: http://www.homemadeitaliancooking.com/italian-rainbow-cookies/
%\end{tikzpicture}
\section[Crab Bisque]{Cajun Crabmeat Bisque}
\begin{recipestats}[
	servings=5 people,
	preptime=45 minutes,
	bakingtime=45 minutes,
	source=Art of the Palate 2006 (Pamela Teresi),
	]
\end{recipestats}

\begin{ingredientcolumns}
	\begin{ingredientblock}
		\ingredient[\nicefrac{3}{4}][C]{butter}\\
		\ingredient[\nicefrac{3}{4}][C]{flour}\\
		\ingredient[3][Tbs]{tomato paste}\\
		\ingredient[\nicefrac{3}{2}][C]{onion, yellow}\\
		\ingredient[1][C]{celery}\\
		\ingredient[\nicefrac{1}{2}][C]{scallions}\\
		\ingredient[4][cloves]{garlic}\\
		\ingredient[\nicefrac{2}{3}][C]{green pepoer}\\
		\ingredient[3][Tbs]{parsley}
	\end{ingredientblock}
	\begin{ingredientblock}
		\ingredient[2][quarts]{stock, chicken}\\
		\ingredient[1][Tbs]{Worcestershire}\\
		\ingredient[1][]{bay leaf}\\
		\ingredient[1][tsp]{thyme, dried}\\
		\ingredient[1][tsp]{salt}\\
		\ingredient[\nicefrac{1}{8}][tsp]{pepper, black}\\
		\ingredient[\nicefrac{1}{8}][tsp]{cayenne}\\
		\ingredient[\nicefrac{1}{2}][tsp]{ketchup}\\
		\ingredient[1][lb]{crabmeat}
	\end{ingredientblock}
\end{ingredientcolumns}

%%%%%%%%%%%%%%%%%%%%%%%%%%%%%%%%%%%%%%%%%%%%%%%%%%%%%%
\begin{preparation}
\item Make roux: melt butter, gradually combine flour, stirring constantly on medium heat, $20...30\; min$ or until golden brown.

\item Add tomato, vegetables finely diced, sweat.

\item Add stock gradually. Add spices, crab meat.

\item Simmer 40 minutes, covered, stirring occasionally. \ragmarpar{Serve w/ hot sauce \& yeast rolls}
\end{preparation}

\begin{experiments}
\item Vary the amount of stock, crab; adding a quart of stock \& 8 $oz$ crab can work.
\end{experiments}

\recipeend

\clearpage
\section{Vegetables}
 \section[Collard Greens]{Thanksgiving Collard Greens}

\begin{recipestats}[
	servings=[4 \ldots 6] people,
	preptime=30 \minute,
	bakingtime=[2 \ldots 6] \hour,
	]
\end{recipestats}

\begin{recipeabstract}
	These are similar to southern-style collards in that they are simmered for a long time with meat.
	Turkey was substituted here for ham as an experiment for thanksgiving, partly due to availability.
	The stock can also be used to make gravy and can be very flavorful.
\end{recipeabstract}


\ragmarpar{smoked turkey necks works best}
\begin{ingredientcolumns}
	\begin{ingredientblock}
		\ingredient[1][lb]{collards}\\
		\ingredient[2][]{turkey necks}\\
		\ingredient[6][C]{water}\\
		\ingredient[2][Tbs]{butter}
	\end{ingredientblock}
	\begin{ingredientblock}
		\ingredient[2][Tbs]{vinegar}\\
		\ingredient[1][tsp]{sugar}\\
		\ingredient[1][tsp]{pepper}\\
		\ingredient[1][tsp]{hot sauce}
	\end{ingredientblock}
\end{ingredientcolumns}


\begin{preparation}
\item Make stock with turkey necks: cover with cold water, cook over high heat, bring to simmer, skim off foam from surface, reduce to $\approx 180$ \Fahrenheit. Simmer for $1 \ldots 4$ \hour.

\item Prepare collards greens ($\approx 15$ \minute~prior to removing turkey necks).
	Clean \& rinse collard greens, cut off the main stem, dice $\|$ tear greens (into $\approx 1x1"$ rectangles).

\item Remove turkey necks and sieve out undesired particulates.

\item Add greens, simmer $\approx 1$ \hour uncovered.

\item Meanwhile, remove and dice meat from turkey necks according to your preference. Add to pot.

\item Add butter, vinegar, spices, serve.
\end{preparation}

\begin{experiments}
	\item This was more of a quick experiment, it would be good to find a more traditional recipe.
	\item The stock was good but removing the meat was quite labor intensive for Thanksgiving day. This recipe might benefit from making the stock ahead of time and giving it a bit more spices \& vegetables.
\end{experiments}


\recipeend


\clearpage
\section{Entr\`{e}es}
\section[Beef Bowl]{Mexican Beef Bowl}\label{mexican_beef_bowl}
\begin{recipestats}[
	servings=4 people,
	preptime=30 minutes,
	bakingtime=30 minutes,
	source=\citeauthor{blueApronBeefBowl} \cite{blueApronBeefBowl}
	]
\end{recipestats}

\begin{ingredientcolumns}
	\begin{ingredientblock}
		\ingredient[1][lb]{ground beef}\\
		\ingredient[\onehalf][lb]{tomatoes, cherry}\\
		\ingredient[\onehalf][lb]{carrots}\\
		\ingredient[1]{lime}\\
		\ingredient[\onefourth][C]{jalepe\~{n}os, pickled}\\
		\ingredient[\onefourth][C]{mayonnaise}\\
		\ingredient[\approx 8][oz.]{cheese, cotija}\\
		\ingredient[\approx 1][C]{rice}\\
		\ingredient[\approx 3][\Tablespoon]{spice blend}
	\end{ingredientblock}
	\begin{ingredientblock}[spice blend]
		\ingredient[2][\Tablespoon]{guajillo}\\
		\ingredient[1][\Tablespoon]{ancho}\\
		\ingredient[1][\Tablespoon]{paprika, smoked}\\
		\ingredient[1][\Tablespoon]{cumin}\\
		\ingredient[1][\Tablespoon]{marjoram}\\
		\ingredient[\onehalf][\Tablespoon]{garlic powder}\\
		\ingredient[\onehalf][\Tablespoon]{salt}
	\end{ingredientblock}
\end{ingredientcolumns}

\ragmarpar{for dried chilis: remove seeds, toast $\approx3\;min$ \@ $350\; \Fahrenheit$, grind}

\begin{preparation}
\item Prepare ground spice blend; ahead of time if preferred.

\item Begin to brown the beef in large saut\'{e} pan, stirring occasionally. Reach a strong color at end of recipe.

\item Dice carrots, toss with oil and spices. Preheat oven to $450 \Fahrenheit$.

\item Begin cooking rice.

\item Roast carrots in oven on a cookie sheet, $12...14\;min$.

\item Dice tomatoes, finely dice jalepe\~{n}os, combine with juice of 1/2 of lime, pickled jalepe\~{n}o juice to taste.

\item Add spices to beef ($\approx 3/4\; Tbs$) at end, cook $\approx1\; min$.\\Add $1/4\;C$ water, cook $2...3\;min$.

\item Combine mayonnaise and juice of 1/2 lime.\\Add pickled jalepe\~{n}o juice to taste.

\item Serve by layering rice, beef, vegetables, pickled jalepe\~{n}os, cheese, mayonnaise.
\end{preparation}


\begin{variation}
\item Substitute riced cauliflower in place of rice. Cut into small pieces, chop in food processor with garlic, optionally pan fry.
\item Substitute Chorizo in place of beef, reduce spices.
\item Cotija cheese is worth the effort to find it, but mozzarella works.
\item For convenience, regular pre-ground chili powder can replace the guajillo. Similarly ancho powder is readily available.
\end{variation}

\recipeend

\section{Fettuccine Alfredo}\index{pasta!Fettucine Alfredo}


\begin{recipestats}[
	servings=2 people,
	preptime=15 \minute,
	bakingtime=20 \minute,
	source=Mike \& Jane,
	original=\citetitle{newCookBook2014}~\cite{newCookBook2014},
	]
\end{recipestats}

\begin{ingredientcolumns}
	\begin{ingredientblock}
		\ingredient[8][\ounce]{fettuccine, dry}\\
		\ingredient[4][\ounce]{mushrooms}\\
		\ingredient[2][\ounce]{Asiago}\\
		\ingredient[\approx~3][cloves]{garlic}
	\end{ingredientblock}
	\begin{ingredientblock}
		\ingredient[3][\Tablespoon]{butter, unsalted}\\
		\ingredient[1][C]{cream, heavy}\\
		\ingredient[\onehalf][\teaspoon]{salt}\\
		\ingredient[\oneeighth][\teaspoon]{pepper}
	\end{ingredientblock}
\end{ingredientcolumns}


\begin{preparation}
\ragmarpar{Stir mushrooms occasionally, cook to well done.}
\item Slice mushrooms, begin to saut\`{e} on medium w/ dash of oil.

\item Cook pasta in salted water to al dente, then toss with oil.
	Meanwhile grate chese, crush garlic.

\item Saut\`{e} crushed garlic in butter in a large saucepan $\approx1$ \minute~on medium-high.

\ragmarpar{Use a bit of the pasta water to thicken the sauce.}
\item Add cream, salt, pepper to sauce. Bring to boil then reduce heat, simmer uncovered $\approx3$ \minute~or until it begins to thicken.

\item Remove from heat add cheese \& mushrooms.

\item Add pasta to sauce, toss to combine.
\end{preparation}


\begin{variation}
\ragmarpar{Pecorino Romano does not melt well for use in the sauce.}
\item Try Parmesan instead of Asiago, mushrooms are optional.

\item For shrimp alfredo add $\approx8~\ounce$ prior to removing from heat, cook through and continue.
\end{variation}


\recipeend

\section{Chistmas Rib Roast}
\index{beef!Rib~Roast}


\begin{recipestats}[
	servings=$2\;person \; / \; 1 \; lb$,
	preptime=1 \onefourth~ \hour,
	bakingtime=2 \onefourth~ \hour,
	source=Ralph Nelson (Fa),
	]
\end{recipestats}

\ragmarpar{Fa's rule is that you MUST NOT open the oven under and circumstances.}

\begin{recipeabstract}
	A Giesken Christmas tradition.
\end{recipeabstract}

\begin{ingredientcolumns}[1]
	\begin{ingredientblock}
		\ingredient{standing rib roast}\\
		\ingredient[\approx 1][\Tablespoon]{pepper}\\
		\ingredient[\approx 1][\teaspoon]{salt}
	\end{ingredientblock}
\end{ingredientcolumns}

\begin{preparation}
\item Let stand at room temperature for one hour.

\item Rub black pepper \& salt on roast. Preheat oven to $400$ \Fahrenheit.

\item Place roast on pan fat side up. Do not cover or add water.

\item Bake 15 \minute~ at 400 \Fahrenheit.

\item Lower to 375 \Fahrenheit, bake for 45 \minute.

\item Turn off heat, do not open oven. Bake for about 30 \minute.


\item Turn on oven to 375 \Fahrenheit~$[35 \dots 45]$ \minute~before eating.
\end{preparation}

\recipeend

\begin{recipe}[
	preparationtime = 30 minutes,
	bakingtime = 2 hours,
	portion = 4 people,
	source = Rick Martinez; Bon Apetit,
	]
	{Chili Colorado}
	\graph{
		smallpicturewidth = 0.3\textwidth,%
		bigpicturewidth = 0.6\textwidth,%
	}
	%%%%%%%%%%%%%%%%%%%%%%%%%%%%%%%%%%%%%%%%%%%%%%%%%%%%%%
	\ingredients{
		\unit[5]{} & ancho\\
		\unit[2]{} & pasilla\\
		\unit[2]{} & guajillo\\
		\unit[8]{C} & stock, chicken\\
		\unit[2]{lb} & pork shoulder\\
		\unit[6...9]{} & garlic cloves\\
		\unit[2]{} & bay leaves\\
		\unit[1]{Tbs} & cumin, ground\\
		\unit[2]{tsp} & sage, fresh\\
		\unit[2]{tsp} & oregano, Mexican
	}
	%%%%%%%%%%%%%%%%%%%%%%%%%%%%%%%%%%%%%%%%%%%%%%%%%%%%%%
	\preparation{
		\step Measure the spices, chop the sage, and crush the garlic.
		\step Prepare the pork. Cut into $\approx1\;inch$ cubes, toss with salt, pepper.
		\step Brown the pork. Heat a neutral oil almost to smoking point in a $ \geqq 3.5\;quart$ pot. Reduce to medium high. Brown in batches so as to not overcrowd, de-glazing if necessary to prevent burning.
		\step Add spices, stir for about a minute.
		\step Add $5\;Cups$ stock, simmer uncovered for $1\;hour$.
		\step Meanwhile,re-hydrate the chilies. Remove stems, seeds, veins from chilies and roughly chop. Add to large bowl, add $3\;Cups$ boiling stock, cover with plastic wrap. Wait $30\;minutes$, then blend it all.
		\step Add blended chilies to the soup at the end of the first simmer. Simmer for $45\;minutes$ uncovered.
		\step Season with salt pepper to taste.
	}
	%%%%%%%%%%%%%%%%%%%%%%%%%%%%%%%%%%%%%%%%%%%%%%%%%%%%%%
	
	\hint{
		\begin{itemize}
			\item "Chili Colorado" means "chili colored red" rather than from the state of Colorado.
			\item Toss pork with a bit of flour as well to thicken the chili further.
			\item Marjoram can be a substitute for the Mexican oregano if necessary, but not Mediterranean oregano.
			\item Serve with tortillas, and the carrots from 'Mexican Beef Bowl' \ref{mexican_beef_bowl}. Rick Martinez recommends rice, beans a la charra, and tortillas.
		\end{itemize}
	}
	%%%%%%%%%%%%%%%%%%%%%%%%%%%%%%%%%%%%%%%%%%%%%%%%%%%%%%
	\setRecipeLengths{
		preparationwidth = 0.60\textwidth,
		ingredientswidth = 0.35\textwidth,
		pictureheight = 6cm,
		bigpicturewidth = 0.6\textwidth,
		smallpicturewidth = 0.35\textwidth
	}
	%%%%%%%%%%%%%%%%%%%%%%%%%%%%%%%%%%%%%%%%%%%%%%%%%%%%%%
	\setRecipeSizes{
		recipename = \fontsize{25pt}{30pt},
		ing = \normalsize,
		inghead = \normalsize,
		prep = \normalsize,
		prephead = \normalsize,
		hint = \normalsize,
		hinthead = \Large
	}
	%%%%%%%%%%%%%%%%%%%%%%%%%%%%%%%%%%%%%%%%%%%%%%%%%%%%%%
	%\setRecipenameFont{
	%%pbsi
	%%fau
	%%fwb
	%%fjd % default when using the option handwritten
	%cmr % default
	%}{T1}{m}{n}
	%%%%%%%%%%%%%%%%%%%%%%%%%%%%%%%%%%%%%%%%%%%%%%%%%%%%%%
	\setHeadlines{
		inghead = Ingredients,
		prephead = Preparation,
		hinthead = Hint,
		calory = Cal,
		continuationhead = Continuation,
		continuationfoot = Continuation on next page
	}
	%%%%%%%%%%%%%%%%%%%%%%%%%%%%%%%%%%%%%%%%%%%%%%%%%%%%%%
	%\setBackgroundPicture
	%[%
	%x = 2cm,
	%y = -1cm,
	%width=\paperwidth-3cm,
	%height,
	%orientation=pagecenter
	%]{pic/bg_transparent} % filepath
\end{recipe}

%\clearpage
%\thispagestyle{empty}
% \begin{tikzpicture}[remember picture,overlay]
% inelegant way of getting a good image:
% use [keepaspectratio], trim to an aspect ratio close to the page, then remove [keepaspectraio] to get full page
%   \node at (current page.center) {\includegraphics[width=\pdfpagewidth,height=\pdfpageheight,clip,trim={47px 10px 47px 10px}]{tricolor}};
% IMAGE: http://www.homemadeitaliancooking.com/italian-rainbow-cookies/
%\end{tikzpicture}

\clearpage
\section{Desserts}
TODO collage of images
TODO add a sub header to the recipe title (e.g. a commit msg)
NOTE consider using bold font in key areas of the steps (e.g. the first sentence?)
TODO fix reference/label between recipes, add recipe name to ref (SEE chili colorado)

% background graphic
%\setBackgroundPicture[x, y=-2cm, width=\paperwidth-4cm, height, orientation = pagecenter]
%{pic/background}
\section{Italian Tricolors}

\begin{recipestats}[
	servings=36 cookies,
	preptime=1 hour (1.25 hour chilling),
	bakingtime=10 minutes,
	source=\citefield{goodHousekeeping_2013}{title} \cite{goodHousekeeping_2013},
	]
\end{recipestats}

\begin{recipeabstract}
	A Teresi Christmas tradition.
	The almond, apricot, and chocolate are quite complementary flavors.
	
\end{recipeabstract}

\ragmarpar{fresh almond paste is critical}
\begin{ingredientcolumns}[1]
	\begin{ingredientblock}
		\ingredient[8][\ounce]{almond paste}\\
		\ingredient[\threefourth][C]{butter}\\
		\ingredient[\threefourth][C]{sugar}\\
		\ingredient[\onehalf][\teaspoon]{almond extract}\\
		\ingredient[3][large]{eggs}\\
		\ingredient[1][C]{flour, all purpose}\\
		\ingredient[\onefourth][\teaspoon]{salt}\\
		\ingredient[15][drops]{red food coloring}\\
		\ingredient[15][drops]{green food coloring}\\
		\ingredient[\twothird][C]{apricot preserves}\\
		\ingredient[3][\ounce]{dark chocolate}\\
		\ingredient[2][\teaspoon]{shortening}
	\end{ingredientblock}
\end{ingredientcolumns}

\begin{preparation}
\item Preheat oven to $350$ \Fahrenheit, grease three $8x8"$ pans.
Line bottoms w/ waxed paper, grease and flour the interior.

\ragmarpar{it's ok if a few lumps remain}
\item Blend at medium-high speed: almond paste, butter, sugar, almond extract.
Reduce to medium and add eggs one-at-a-time.
Reduce to low and beat in flour \& salt until just combined.

\item Divide batter into thirds into separate bowls.
Blend green dye into one, red into another.

\item For each mixture. transfer and spread evenly into the pans.

\item Bake on two oven racks $10-12\; min$ rotating between upper/lower halfway through.

\item Cool in pans on wire racks $5\; min$.
Run knife around sides to loosen layers. Invert onto racks and cool completely; removing the paper when done.

\ragmarpar{add more shortening or corn syrup to the chocolate to make it easier to cut}
\item Blend jam in food processor to remove the larger chunks.

\item Assemble layers, green / white / red, with jam between.
Melt on low chocolate / shortening, stirring frequently.
Spread on top then refrigerate $\geq1$ $hour$.

\item Rest at room temperature for $\geq 15\; min$ then trim the edges and cut into squares.
Store cookies in a single layer in a tightly covered container.
Refrigerate $\approx 1\; week$ or freeze $\approx 3\; months$.
\end{preparation}

\begin{variation}
\item Other colors and jams are available. Fourth of July with red / white / blue / cherries works out nicely.
\end{variation}
\recipeend
\section[Felix Cookies]{Felix Cookies, or,\\ \mbox{Schwarz-Wei\ss-Geb\"{a}ck}}


\begin{recipestats}[
	servings=40 cookies,
	preptime=30 \minute~(+chill),
	bakingtime=12 \minute,
	source=\citetitle{luisaWeiss2016} \cite{luisaWeiss2016},
	]
\end{recipestats}


\begin{recipeabstract}
	A shortbread cookie with a black and white checkerboard.
	Works very well with the addition of chocolate.
	Aliased for our black and white cats, Felix and Frankie.
\end{recipeabstract}


\begin{ingredientcolumns}
	\begin{ingredientblock}
		\ingredient[150][\gram]{butter, unsalted}\\
		\ingredient[75][\gram]{sugar, powdered}\\
		\ingredient[\oneeighth][\teaspoon]{salt}\\
		\ingredient[\onefourth][\teaspoon]{vanilla extract}
	\end{ingredientblock}
	\begin{ingredientblock}
		\ingredient[200][\gram]{flour, all purpose}\\
		\ingredient[1]{egg yolk}\\
		\ingredient[2][\Tablespoon]{whole milk}\\
		\ingredient[2\;\onehalf][\Tablespoon]{cocoa powder}
	\end{ingredientblock}
\end{ingredientcolumns}


\begin{preparation}
\ragmarpar{other shapes are available, like 6 petal flowers, spirals, etc.}
\item Cream butter $\approx 1$ \minute, add sugar, salt, vanilla, then cream.
Add flour and mix until just combined.

\item Divide dough in half, mix cocoa into one half.
Form into disks, wrap in plastic wrap, refrigerate $\approx1$ \hour.

\item Mix milk \& egg yolk in a small bowl.

\ragmarpar{don't over bake, try adding another sheet below to shield the radiation}
\item Make 4 square logs, brush sides w/ egg wash, press together and refrigerate $\approx 30$ \minute.

\item Preheat oven, line baking sheets w/ parchment paper.

\item Slice cookies to $\approx 1$ $cm$, bake $12 \dots 15$ \minute.
\end{preparation}


\begin{variation}
\item Dip into dark chocolate.
\end{variation}


\recipeend

\begin{recipe}[
	bakingtime = 30 $min$,
	portion = $\sim60$ squares,
	source = Cookie Swap 2003,
	]
	{Peppermint Fudge}
	\graph{
		%small = strawberry,
		smallpicturewidth = 0.3\textwidth,
		%big = strawberrycake,
		bigpicturewidth = 0.6\textwidth,
	}
	%%%%%%%%%%%%%%%%%%%%%%%%%%%%%%%%%%%%%%%%%%%%%%%%%%%%%%
	\ingredients{
		\unit[4]{C} & sugar \\
		\unit[10]{oz} & evaporated milk\\
		\unit[1]{C} & butter\\
		\unit[2]{C} & chocolate chips\\
		\unit[7]{oz} & marshmallow creme\\
		\unit[1/2]{tsp} & peppermint extract\\
		\unit[2/3]{C} & peppermint candy
	}
	%%%%%%%%%%%%%%%%%%%%%%%%%%%%%%%%%%%%%%%%%%%%%%%%%%%%%%
	\preparation{
		\step Line a $13x9\; inch$ pan with foil and butter the interior.\\Crush peppermint candy.
		\step Combine sugar, milk, butter, in a $3\;quart$ saucepan.\\Bring to boil over medium-high heat, stirring constantly.
		\step Reduce to medium, stir to $10\; minutes$.
		\step Remove from heat, add chocolate chips, marshmallow creme, peppermint extract. Stir until chocolate and creme are melted and mixture is smotth.
		\step Pour into pan, sprinkle peppermint on top, cover, refridgerate until set.
	}
	%%%%%%%%%%%%%%%%%%%%%%%%%%%%%%%%%%%%%%%%%%%%%%%%%%%%%%
	\hint{
		\begin{itemize}
			\item 
		\end{itemize}
	}
	%%%%%%%%%%%%%%%%%%%%%%%%%%%%%%%%%%%%%%%%%%%%%%%%%%%%%%
	\setRecipeLengths{
		preparationwidth = 0.60\textwidth,
		ingredientswidth = 0.35\textwidth,
		pictureheight = 6cm,
		bigpicturewidth = 0.6\textwidth,
		smallpicturewidth = 0.35\textwidth
	}
	%%%%%%%%%%%%%%%%%%%%%%%%%%%%%%%%%%%%%%%%%%%%%%%%%%%%%%
	\setRecipeSizes{
		recipename = \fontsize{25pt}{30pt},
		ing = \normalsize,
		inghead = \normalsize,
		prep = \normalsize,
		prephead = \normalsize,
		hint = \normalsize,
		hinthead = \Large
	}
	%%%%%%%%%%%%%%%%%%%%%%%%%%%%%%%%%%%%%%%%%%%%%%%%%%%%%%
	%\setRecipenameFont{
	%%pbsi
	%%fau
	%%fwb
	%%fjd % default when using the option handwritten
	%cmr % default
	%}{T1}{m}{n}
	%%%%%%%%%%%%%%%%%%%%%%%%%%%%%%%%%%%%%%%%%%%%%%%%%%%%%%
	\setHeadlines{
		inghead = Ingredients,
		prephead = Preparation,
		hinthead = Hint,
		calory = Cal,
		continuationhead = Continuation,
		continuationfoot = Continuation on next page
	}
	%%%%%%%%%%%%%%%%%%%%%%%%%%%%%%%%%%%%%%%%%%%%%%%%%%%%%%
	%\setBackgroundPicture
	%[%
	%x = 2cm,
	%y = -1cm,
	%width=\paperwidth-3cm,
	%height,
	%orientation=pagecenter
	%]{pic/bg_transparent} % filepath
\end{recipe}

%\clearpage
%\thispagestyle{empty}
% \begin{tikzpicture}[remember picture,overlay]
% inelegant way of getting a good image:
% use [keepaspectratio], trim to an aspect ratio close to the page, then remove [keepaspectraio] to get full page
%   \node at (current page.center) {\includegraphics[width=\pdfpagewidth,height=\pdfpageheight,clip,trim={47px 10px 47px 10px}]{tricolor}};
% IMAGE: http://www.homemadeitaliancooking.com/italian-rainbow-cookies/
%\end{tikzpicture}




\end{document} 