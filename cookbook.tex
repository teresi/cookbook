% The Teresi family cookbook
%
% Collection of favorites from family / friends / the interwebz
%
% Design:
% - targets a book with a 6x9" trim
% - uses outer marginnotes and single column
%   (worked out better than both inner/outer notes for 6x9)
%   (worked out better than twocolumn for 6x9)
% - provides environments for formatting
% - targets a recipe format of
%   abstract, ingredients (1 | 2 col), steps, variations
% - images not included in this git repository for space
%
% Future:
% - add temperature command w/ compile time switch between Fahrenheit / Centigrade
% - add aliases for Tablespoon, teaspon, fluid oz, minutes etc
% - add keywords / index
% - add 'contents at a glance' (1 level) before contents
% - add a new package and move environments there?
% - add a compile time toggle for adding images
% - add a matrix of images for multiple recipes
% - add tessalation for page backround artwork
%   (SEE http://tex.my/creating-tiled-background-patterns/)
% - add a flag for changing justification on the margin paragraphs (e.g. use justify, use ragged)
% - add an alias for common measurements? e.g to replace \nicefrac{1}{4} etc.?

% Bugs:
% - there may be a line skip on the ingredient tables (on top) when using multicols
% - marginnote used the wrong side of the page, possibly after the preface was added? had to use marginpar
% - in xcookybooky couldn't use \cite[]{} (vs \cite{}) inside a recipe block? :  \begin{recipe}[source=\cite[]{}]{}

% Recipes:
% - can we get a southern style mac and cheese?
% - can we get a canonical southern style collards?

\documentclass[10pt]{book}
\newcommand\ningredientcol{2}  % variable for number of columns for ingredient list
\usepackage[T1]{fontenc}
\usepackage[latin1]{inputenc}

\usepackage{ragged2e}  % for \justify
\usepackage{marginnote}
\usepackage[%
showframe,%
twoside,%
top=0.75in,%
bottom=0.5in,%
marginparsep=3ex,%
marginparwidth=0.8in,%
inner=0.875in,%
outer=1.25in,%
papersize={6in,9in},%
]{geometry}

%\usepackage[cam,letter,center]{crop} % show trim lines


\usepackage{fancyhdr}
\pagestyle{empty}
\usepackage{lmodern}
\usepackage{lipsum}
\usepackage[ngerman, english]{babel}
\usepackage{graphicx}
\usepackage{xcolor}
\usepackage[clock, misc]{ifsym} % symbols
\usepackage{cookingsymbols}     % symbols e.g. \Oven, \Dish etc
\usepackage{tikzsymbols}        % symbols
\usepackage{fontawesome}        % symbols
\usepackage{xkeyval}
\usepackage{lettrine} % a large first letter/number
\usepackage{units}
%\usepackage{eso-pic} % for background pictures
%\usepackage{picture} % for modfifying the position of the bg pictures
\usepackage{tabularx} % line breaks in tabular
\usepackage{tikz}
\usepackage{nicefrac}
\usepackage{amssymb}
\usepackage{paralist}
\usepackage{multicol}
\usepackage{xparse}  % positional arg defaults
%\usepackage{tikzpagenodes}  % for printing marginnotes on correct side
\usepackage{pgfornament}  % symbols
\usepackage{nameref}      % ref section name etc. w/ \nameref
\usepackage{adforn}
\usepackage{enumitem}
\setlist{leftmargin=*}  % don't indent lists
\usepackage{longtable}
\usepackage{ifthen}

% NOTE what are the right settings for `nowidow` to work?
% adding \noclub to variation env doesn't seem to work
\usepackage[]{nowidow}  % wrecks tables so just load macros (no options)

% NOTE what are the right inputs to have sections moved to new page
% if they are too close to the bottom?
\usepackage[nobottomtitles]{titlesec}
%\titlespacing*{\section}{12pt}{48pt}{4pt}
%\usepackage[defaultlines=4,all]{nowidow}  % nowidow breaks the ingredient tables

\usepackage{sectsty}  % access sectional units
\usepackage{fancyhdr}
\usepackage[backend=bibtex]{biblatex}

\bibliography{bib}

\usepackage{hyperref}    % must be the last package
\hypersetup{%
	pdfauthor            = {Michael Teresi},
	pdftitle             = {Teresi Family Cookbook},
	pdfsubject           = {Recipes},
	pdfkeywords          = {recipes, recipe, cookbook, Teresi},
	pdfstartview         = {FitV},
	pdfview              = {FitH},
	pdfpagemode          = {UseNone}, % Options; UseNone, UseOutlines
	bookmarksopen        = {true},
	pdfpagetransition    = {Glitter},
	colorlinks           = {true},
	linkcolor            = {magenta},
	urlcolor             = {blue},
	citecolor            = {cyan},
	filecolor            = {black},
}

%\hbadness=10000	% Ignore underfull boxes

% FUTURE move to env variable?
\graphicspath{}  % use out of source path

% add an ornament to the section numbers
\makeatletter
\def\@seccntformat#1{\adforn{33} \csname the#1\endcsname\quad}
\makeatother

% show servings, time, etc. w/ icons
\ExplSyntaxOn
% keys
\keys_define:nn { mymodule/recipestats }
{
	servings .tl_set:N = \l_mymodule_servings_tl,
	preptime .tl_set:N = \l_mymodule_preptime_tl,
	bakingtime .tl_set:N = \l_mymodule_bakingtime_tl,
	source .tl_set:N = \l_mymodule_source_tl,
}
\NewDocumentEnvironment{recipestats}{O{}}
{
	\keys_set:nn { mymodule/recipestats } { #1 }
	% BUG a multicols doesn't left justify the icons?
	% FUTURE using a multicols may be preferred so that missing icons aren't just blank cells
	\begin{tabular}{clcl}
		\tl_if_empty:NTF \l_mymodule_servings_tl
		{& &}
		{\Dish & \l_mymodule_servings_tl &}
		
		\tl_if_empty:NTF \l_mymodule_source_tl
		{& \\}
		{\PaperLandscape & \l_mymodule_source_tl \\}
		
		\tl_if_empty:NTF \l_mymodule_preptime_tl
		{& &}
		{\Gloves & \l_mymodule_preptime_tl &}
		
		\tl_if_empty:NTF \l_mymodule_bakingtime_tl
		{& \\}
		{\oven & \l_mymodule_bakingtime_tl \\}
}
{
	\end{tabular}
	\newline  % NOTE is there a better way?
}
\ExplSyntaxOff

% italicize the abstract
\newenvironment{recipeabstract}%
{
	\itshape
}
{

}

% a 3 col tabular for ingredients
% can be used to specify which componenet the ingredients are for
% can be used to have multiple tabulars in columns
% Args:
%    (str): name of ingredient 'section', no name if None
\ExplSyntaxOn % for \tl_if_blank:nTF
\NewDocumentEnvironment{ingredientblock}{O{}}
{
	% default args:
	% https://tex.stackexchange.com/questions/29973/more-than-one-optional-argument-for-newcommand
	\tl_if_blank:nTF {#1}
	{}  % empty
	{\textit{#1}\\}  % not empty
	\begin{tabular}[t]{rl>{\bfseries}l}
}
{
	\end{tabular}
}
\ExplSyntaxOff % for \tl_if_blank:nTF

% wrapper around a multicols
% Args:
%     (int): number of columns, 2 if None
\NewDocumentEnvironment{ingredientcolumns}{O{2}}
{
	\setlength\columnsep{1ex}
	% NOTE is there a better way to fix the spacing?
	% The 1col ingredientcolumns has weird spacing so add a skip
	\ifthenelse{\equal{#1}{1}}{\vskip1ex}{}
	\centering
	\noindent
	\begin{multicols}{#1}
}
{
	\end{multicols}
	\par
	\ifthenelse{\equal{#1}{1}}{\vskip1em}{}
}

% ingredient for a 3 column tabular 
% intended for: quantity & unit & food
% default empty quantity and unit:
% e.g. \food[col1][col2]{col3}
% e.g. \food[quantity][unit]{name}
% e.g. \food{<ingredient name>}
\NewDocumentCommand{\ingredient}{O{} O{} m }{
	\ensuremath{#1} & \ensuremath{#2} & #3
}

\newenvironment{preparation}%
{
	% can also be controlled globally using
	% \setlist[enumerate]{<options>}
	\begin{enumerate}[topsep=0pt,itemsep=0ex,partopsep=0ex,parsep=0.5ex,label=\textbf{\arabic*}.]
	\setcounter{enumi}{-1}
	}
{
	\end{enumerate}
	\par
	\vspace{2ex}  % NOTE is there a more canonical solution?
}

\newenvironment{experiments}%
{
	
	\textbf{EXPERIMENTS}%
	\begin{itemize}[label=\faFlask,topsep=0pt,itemsep=0ex,partopsep=1ex,parsep=1ex]
	}
	{
	\end{itemize}
	\par
	\vspace{2ex}  % NOTE is there a more canonical solution?
}

\newenvironment{variation}%
{
	\textbf{VARIATIONS}%
	\begin{itemize}[label=\adforn{28},topsep=0pt,itemsep=0ex,partopsep=1ex,parsep=1ex]
	}
{
	\end{itemize}
	\par
	\vspace{2ex}  % NOTE is there a more canonical solution?
}

% FUTURE wrap an input call to add this?
\newcommand{\recipeend}[0]{
	{\centering \pgfornament[scale=0.55]{88} \par}
}

% marginpar with ragged justification
% NOTE had to use marginpar b/c marginnote produce notes on wrong side (after adding preface?)
% https://latex.org/forum/viewtopic.php?t=6041
\newcounter{pl}
\newcommand\ragmarpar[1]{%
	\stepcounter{pl}\label{pl-\thepl}%
	\ifthenelse{\isodd{\pageref{pl-\thepl}}}%
	{\marginpar{\RaggedRight #1}}
	{\marginpar{\RaggedLeft #1}}
}


% abbreviations / aliases
\newcommand\onehalf{\nicefrac{1}{2}}
\newcommand\onethird{\nicefrac{1}{3}}
\newcommand\onefourth{\nicefrac{1}{4}}
\newcommand\oneeighth{\nicefrac{1}{8}}

\newcommand\twothird{\nicefrac{2}{3}}

\newcommand\threehalf{\nicefrac{3}{2}}
\newcommand\threefourth{\nicefrac{3}{4}}
\newcommand\threeeighth{\nicefrac{3}{8}}

\newcommand\Tablespoon{Tbs}
\newcommand\teaspoon{tsp}
\newcommand\Celsius{\ensuremath{C\^\circ}}
\newcommand\Fahrenheit{\ensuremath{F^\circ}}
\newcommand\ounce{\ensuremath{oz}}
\newcommand\fluidounce{\ensuremath{fl\;oz}}

\frenchspacing
% stop chapters / titles from having an extra blank page
\let\cleardoublepage=\clearpage

\pagestyle{fancy}
\newcommand{\changefont}{%
	\fontsize{9}{11}\selectfont
}
\fancyhf{}
\fancyhead[LE,RO]{\changefont \slshape \rightmark} %section
\fancyhead[RE,LO]{\changefont \slshape \leftmark} %chapter
\fancyfoot[C]{\changefont \thepage} %footer



%% wrapper around a multicols
%% Args:
%%     (int): number of columns, 2 if None
%\newsavebox\ltmcbox
%\NewDocumentEnvironment{ingredientmulticols}{O{2}}
%{
%	\vbox\bgroup
%	\setlength\columnsep{0ex}
%	\centering
%	\begin{multicols}{#1}
%		\setbox\ltmcbox
%			\makeatletter\col@number\@ne
%			\begin{longtable}{rl>{\bfseries}l}
%}
%{
%	\end{longtable}
%	\unskip
%	\unpenalty
%	\unpenalty
%	\egroup
%	\unvbox\ltmcbox
%	\end{multicols}
%}



% box for a multi column long table
% SEE https://tex.stackexchange.com/questions/161827/balanced-longtables-multicol-and-page-breaks
%\makeatletter
%\newsavebox\ltmcbox
%\newsavebox\xxbox
%\newenvironment{multicolslongtable}[1]{
%	\setbox\ltmcbox\vbox\bgroup
%	\col@number\@ne
%	\begin{longtable}{#1}
%	}
%	{
%	\end{longtable}
%	\unskip
%	\unpenalty
%	\unpenalty\egroup
%	\unvbox\ltmcbox
%}
%\makeatother


% environment to accept table input and produce multicol
%\NewDocumentEnvironment{ingredientmulticols}{O{2}}
%{
%	\savebox\xxbox\bgroup
%	\begin{minipage}{\textwidth}
%		\begin{multicols}{#1}
%			\begin{multicolslongtable}{| l | l | l |}
%			}
%			{
%			\end{multicolslongtable}
%		\end{multicols}
%	\end{minipage}
%	\egroup
%	\begin{multicols}{#1}
%		\centering
%		\usebox\xxbox
%	\end{multicols}
%}


% patch multicols to allow for one column
% SEE https://tex.stackexchange.com/questions/233866/one-column-multicol-environment
\let\multicolmulticols\multicols
\let\endmulticolmulticols\endmulticols
\RenewDocumentEnvironment{multicols}{mO{}}
{%
	\ifnum#1=1
	#2%
	\else % More than 1 column
	\multicolmulticols{#1}[#2]
	\fi
}
{%
	\ifnum#1=1
	\else % More than 1 column
	\endmulticolmulticols
	\fi
}


\begin{document}

\frontmatter
\begin{titlepage}
\onecolumn
\pagestyle{empty}
%% temporary titles
% command to provide stretchy vertical space in proportion
\newcommand\nbvspace[1][3]{\vspace*{\stretch{#1}}}
% allow some slack to avoid under/overfull boxes
\newcommand\nbstretchyspace{\spaceskip0.5em plus 0.25em minus 0.25em}
% To improve spacing on titlepages
\newcommand{\nbtitlestretch}{\spaceskip0.6em}
{
	\centering
	\bfseries
	\nbvspace[1]
	\Huge
	{\nbtitlestretch
		TERESI FAMILY COOKERY
	}\\
	\nbvspace[1]
	\footnotesize
	AN ONGOING COLLECTION OF FAVORITES FROM AROUND THE GLOBE\\
	\nbvspace[1]
	\Large MICHAEL TERESI \\ \small \& COMPANY\\
	\nbvspace[1]
	\hspace{1em}  % FUTURE a more legit way of lining it up?
	\IfFileExists{../cookbook_img/stein2.png}{\includegraphics[width=0.33\textwidth]{../cookbook_img/stein2}}{}
	\vfill
}
\end{titlepage}

\tableofcontents
\chapter*{Preface}
\addcontentsline{toc}{chapter}{Preface}
Welcome to the Teresi family cookbook.
This is a curated list of recipes from family, friends, and other sources.
Consider it an invitation for collaboration.

This is designed as an ongoing project.
It's written in \LaTeX\ to separate content from presentation and versioned in \texttt{Git} to track changes over a long period of time.

Said presentation is included below, see \texttt{recipe\_snippet.tex}.
Future changes to the layout will affect all recipes through the use of environments / commands provided there.

\section[Example Recipe]{Example Recipe Long Title~\ragStars{3}}%\index{ingredient!Example Recipe}
% add \index on the same line as the \section


\begin{recipestats}[
	servings=servings,
	preptime=prep time,
	bakingtime=cook time,
	inactivetime=inactive time,
	source=source,
	original=original source,
]
\end{recipestats}
\ragmarpar{$\largeblackstar$'s for ratings}

\begin{recipeabstract}
	This abstract provides the background.
	Fill out recipe fields above and the steps below.
	Removing an icon entry will leave it blank.
\end{recipeabstract}
\ragmarpar{\faLeaf's~for vegetarian}


\begin{ingredientcolumns}
	\begin{ingredientblock}[first~group]
		\ingredient[qnty][unit]{ingredient}\\
		\ingredient[qnty][unit]{ingredient}
	\end{ingredientblock}
	\begin{ingredientblock}[second~group]
		\ingredient[qnty][unit]{ingredient}\\
		\ingredient[qnty][unit]{ingredient}
	\end{ingredientblock}
\end{ingredientcolumns}


\begin{preparation}
\item First step$\ldots$
\end{preparation}
\ragmarpar{add special hints to margin here}


\begin{variation}
\item First variation$\ldots$
\end{variation}


\begin{experiments}
\item First experiment$\ldots$
\end{experiments}


\recipeend%

\nopagebreak

\mainmatter

\chapter{Breakfast}
\clearpage
\section[Egg Bites]{Egg Bites}
\index{egg!Egg~Bites}


\begin{recipestats}[
	servings=12 jars,
	preptime=25 \minute,
	bakingtime=1 \hour,
	inactivetime=1 \hour,
	source=Mike \& Jane,
	original=\citetitle{anovaEggBites} \cite{anovaEggBites},
	]
\end{recipestats}


\begin{recipeabstract}
	These are perfect for a quick and portable breakfast.
	Jane and I got a sous vide machine for our wedding and tried them as the inaugural recipe since they cook quickly.
	That and Jane's appreciation for Starbucks Egg Bites.
	As it turns out there are many egg bite clones available for all tastes.
\end{recipeabstract}


\begin{ingredientcolumns}
	\begin{ingredientblock}
		\ingredient[12][large]{eggs}\\
		\ingredient[1][C]{Asiago}\\
		\ingredient[8][\ounce]{mushrooms}\\
		\ingredient[6][slices]{bacon}
	\end{ingredientblock}
	\begin{ingredientblock}
		\ingredient[\nicefrac{1}{2}][C]{cream cheese}\\
		\ingredient[\approx 3][\Tablespoon]{butter}\\
		\ingredient[\onefourth][tsp]{salt}\\
		\ingredient[12][4~\fluidounce]{canning jars}
	\end{ingredientblock}
\end{ingredientcolumns}


\ragmarpar{Greasing the jars is important as the eggs can get stuck.}
\begin{preparation}
\item Set the water bath for $172$ \Fahrenheit / $77.8$ \Celsius.

\item Grease jar interiors w/ butter $\|$ crisco etc.

\item Slice mushrooms and begin to saut\`{e} w/ dash of salt. Meanwhile slice bacon in half and add.
	Remove bacon when still slightly chewy, remove mushrooms when well browned.

\item Meanwhile, grate the cheese.

\item Add bacon \& mushrooms to each jar.

\item Whisk eggs, cheeses, salt. Add egg mixture to each jar.

\item Optionally, add a small pat of butter on top to each jar.

\item Add lids to each jar and screw on lightly using only your fingertips. The goal is to allow air to release in order to prevent the jars from shattering in the bath.

\item Add jars to water bath and cook for $1\; hour$. Remove, cool, tighten lids, then refrigerate up to $\approx1 week$.

\item Reheat by:
\begin{inparaenum}[\itshape a)\upshape]
	\item microwave for $[1...1.5]$ \minute, or
	\item invert \& remove, broil for a few minutes.
\end{inparaenum}

\end{preparation}


\begin{variation}
\item Jane likes red pepper, the original recipe was bacon \& $1 C$ Gruyere.
\item Many ingredients are available, try spinach / feta, basil, chilis, chorizo, potato, tomatoes, etc.
\end{variation}


\begin{experiments}
\item The cream effects the texture of the eggs: milk produces a smooth consistency, cream is fluffier, cream \& cottage cheese ($50/50$) or cream cheese for in-between.
\end{experiments}


\recipeend

\section[Spinach Quiche]{Spinach Artichoke Quiche}\index{egg!Spinach Quiche}\index{quiche!Spinach}


\begin{recipestats}[
	servings=4 people,
	preptime=45 \minute,
	bakingtime=45 \minute,
	source=Mike \& Jane,
	original=\citefield{joyofcooking2006}{title}~\cite{joyofcooking2006},
	]
\end{recipestats}


\begin{recipeabstract}
	A classic dish.
	We like to serve this as a side along with cardamom rice and have leftovers for breakfast.
\end{recipeabstract}
\ragmarpar{See~\ref{basicFlakyPiePastry} for pastry recipe.}


\begin{ingredientcolumns}
	\begin{ingredientblock}
		\ingredient[1][]{pie pastry}\\
		\ingredient[6][large]{eggs}\\
		\ingredient[1][C]{cream}\\
		\ingredient[2\;\onehalf][C]{spinach}\\
		\ingredient[1\;\onehalf][C]{Asiago}\\
		\ingredient[6][\fluidounce]{artichokes}
	\end{ingredientblock}
	\begin{ingredientblock}
		\ingredient[1][]{onion, white}\\
		\ingredient[3][Tbs]{butter}\\
		\ingredient[2][\teaspoon]{garlic}\\
		\ingredient[\onefourth][\teaspoon]{salt}\\
		\ingredient[\onehalf][\teaspoon]{pepper, white}\\
		\ingredient[\onehalf][\teaspoon]{marjoram}
	\end{ingredientblock}
\end{ingredientcolumns}


\begin{preparation}
\item Reserve a large bowl and large pie pan.

\ragmarpar{If using pre-mad pastry, roll it out a little more.}
\item Prepare pastry and chill prior.
	Preheat oven to $400$ \Fahrenheit.

\item Measure spinach, artichokes. Grind salt, pepper, marjoram.

\item Begin to cook the onion.
	Slice onion, cook over medium-low heat with a little olive oil while stirring occasionally.

\ragmarpar{See~\ref{basicFlakyPiePastry} for more partial prebake details.}
\item Begin to partially prebake the pastry.
	Butter pie pan, add pastry, pre-bake for 15 $\minute$ @ $400\;\Fahrenheit$ w/ weights, $[10..12]$  $\minute$ @ $375\;\Fahrenheit$ w/o weights.
	Set aside 1 egg yolk and immediately brush with egg yolk when removed from oven.

\item Meanwhile, roughly chop artichokes \& spinach, shred cheese.

\item Prepare vegetables.
	Add spinach to onions.
	Add garlic, butter, and cook $\approx 1$ \minute, take off the heat.

\item Combine remaining eggs, cream, salt, pepper, marjoram.

\item Layer into pie: cheese, vegetables, egg mixture.
	Optionally garnish with $\onehalf$ $\teaspoon$ paprika for some pizzazz.

\ragmarpar{Check @ $20 \minute$ and add foil to edges if needed to prevent burning.}
\item Bake at $375$ \Fahrenheit~for $30$ \minute, broil briefly, rest $10$ \minute.
\end{preparation}


\begin{variation}
\item Quiche Lorraine uses bacon and Gruyere, quiche Florentine is spinach and Gruyere, and leek and onion appear popular.

\item Other milk varieties are available but heavy cream will make the best texture in my humble opinion.
\end{variation}


\recipeend%


\chapter{Soup}
\clearpage
\begin{recipe}[
    preparationtime = 10 minutes,
    bakingtime = 4 hours 20 minutes,
    portion = 3 quarts (4...6 people),
    source = Pamela Teresi
]
{Turkey Gumbo}
\graph{
    %small = strawberry,
    smallpicturewidth = 0.3\textwidth,
    %big = strawberrycake,
    bigpicturewidth = 0.6\textwidth,
}
%%%%%%%%%%%%%%%%%%%%%%%%%%%%%%%%%%%%%%%%%%%%%%%%%%%%%%
\ingredients{
\unit[1]{lb} & sausage, smoked\\
\unit[2]{C} & turkey\\
\unit[1/4]{C} & vegetable oil\\
\unit[1/4]{C} & flour\\
\unit[5]{stalks} & celery\\
\unit[2]{medium} & onions\\
\unit[2]{C} & turkey\\
\unit[4]{C} & stock $\|$ broth\\
\unit[1]{tsp} & gumbo fil\'{e}\\
\unit[1/2]{tsp} & salt\\
\unit[2 1/2]{C} & cooked rice
}
%%%%%%%%%%%%%%%%%%%%%%%%%%%%%%%%%%%%%%%%%%%%%%%%%%%%%%
\preparation{
\step Brown sausage, dice vegetables \& turkey, set aside.

\step Make roux: combine flour \& oil, stirring constantly on medium heat, $10...15\; min$ or until copper in color.

\step Add celery \& onions to roux, cook $5\; minutes$.

\step Add turkey, sausage, stock. Simmer $\sim 4\; hours$.

\step Meanwhile, prepare rice to serve when gumbo is complete.

\step Add gumbo fil\'{e}, salt, pepper. Serve over rice.
}
%%%%%%%%%%%%%%%%%%%%%%%%%%%%%%%%%%%%%%%%%%%%%%%%%%%%%%
\hint{
\begin{itemize}
\item Andouille sausage works particularly well.
\item Serve with hot sauce \& yeast rolls.
\item Traditionally served the day after Thanksgiving.
\end{itemize}
}
%%%%%%%%%%%%%%%%%%%%%%%%%%%%%%%%%%%%%%%%%%%%%%%%%%%%%%
\setRecipeLengths{
preparationwidth = 0.60\textwidth,
ingredientswidth = 0.35\textwidth,
pictureheight = 6cm,
bigpicturewidth = 0.6\textwidth,
smallpicturewidth = 0.35\textwidth
}
%%%%%%%%%%%%%%%%%%%%%%%%%%%%%%%%%%%%%%%%%%%%%%%%%%%%%%
\setRecipeSizes{
recipename = \fontsize{25pt}{30pt},
ing = \normalsize,
inghead = \normalsize,
prep = \normalsize,
prephead = \normalsize,
hint = \normalsize,
hinthead = \Large
}
%%%%%%%%%%%%%%%%%%%%%%%%%%%%%%%%%%%%%%%%%%%%%%%%%%%%%%
%\setRecipenameFont{
%%pbsi
%%fau
%%fwb
%%fjd % default when using the option handwritten
%cmr % default
%}{T1}{m}{n}
%%%%%%%%%%%%%%%%%%%%%%%%%%%%%%%%%%%%%%%%%%%%%%%%%%%%%%
\setHeadlines{
inghead = Ingredients,
prephead = Preparation,
hinthead = Hint,
calory = Cal,
continuationhead = Continuation,
continuationfoot = Continuation on next page
}
%%%%%%%%%%%%%%%%%%%%%%%%%%%%%%%%%%%%%%%%%%%%%%%%%%%%%%
%\setBackgroundPicture
%[%
%x = 2cm,
%y = -1cm,
%width=\paperwidth-3cm,
%height,
%orientation=pagecenter
%]{pic/bg_transparent} % filepath
\end{recipe}

%\clearpage
%\thispagestyle{empty}
% \begin{tikzpicture}[remember picture,overlay]
   % inelegant way of getting a good image:
   % use [keepaspectratio], trim to an aspect ratio close to the page, then remove [keepaspectraio] to get full page
%   \node at (current page.center) {\includegraphics[width=\pdfpagewidth,height=\pdfpageheight,clip,trim={47px 10px 47px 10px}]{tricolor}};
   % IMAGE: http://www.homemadeitaliancooking.com/italian-rainbow-cookies/
%\end{tikzpicture}
\section[Crab Bisque]{Cajun Crabmeat Bisque}
\begin{recipestats}[
	servings=5 people,
	preptime=45 minutes,
	bakingtime=45 minutes,
	source=Art of the Palate 2006 (Pamela Teresi),
	]
\end{recipestats}

\begin{ingredientcolumns}
	\begin{ingredientblock}
		\ingredient[\nicefrac{3}{4}][C]{butter}\\
		\ingredient[\nicefrac{3}{4}][C]{flour}\\
		\ingredient[3][Tbs]{tomato paste}\\
		\ingredient[\nicefrac{3}{2}][C]{onion, yellow}\\
		\ingredient[1][C]{celery}\\
		\ingredient[\nicefrac{1}{2}][C]{scallions}\\
		\ingredient[4][cloves]{garlic}\\
		\ingredient[\nicefrac{2}{3}][C]{green pepoer}\\
		\ingredient[3][Tbs]{parsley}
	\end{ingredientblock}
	\begin{ingredientblock}
		\ingredient[2][quarts]{stock, chicken}\\
		\ingredient[1][Tbs]{Worcestershire}\\
		\ingredient[1][]{bay leaf}\\
		\ingredient[1][tsp]{thyme, dried}\\
		\ingredient[1][tsp]{salt}\\
		\ingredient[\nicefrac{1}{8}][tsp]{pepper, black}\\
		\ingredient[\nicefrac{1}{8}][tsp]{cayenne}\\
		\ingredient[\nicefrac{1}{2}][tsp]{ketchup}\\
		\ingredient[1][lb]{crabmeat}
	\end{ingredientblock}
\end{ingredientcolumns}

%%%%%%%%%%%%%%%%%%%%%%%%%%%%%%%%%%%%%%%%%%%%%%%%%%%%%%
\begin{preparation}
\item Make roux: melt butter, gradually combine flour, stirring constantly on medium heat, $20...30\; min$ or until golden brown.

\item Add tomato, vegetables finely diced, sweat.

\item Add stock gradually. Add spices, crab meat.

\item Simmer 40 minutes, covered, stirring occasionally. \ragmarpar{Serve w/ hot sauce \& yeast rolls}
\end{preparation}

\begin{experiments}
\item Vary the amount of stock, crab; adding a quart of stock \& 8 $oz$ crab can work.
\end{experiments}

\recipeend

\chapter{Sides}
\clearpage
 \section[Collard Greens]{Thanksgiving Collard Greens}

\begin{recipestats}[
	servings=[4 \ldots 6] people,
	preptime=30 \minute,
	bakingtime=[2 \ldots 6] \hour,
	]
\end{recipestats}

\begin{recipeabstract}
	These are similar to southern-style collards in that they are simmered for a long time with meat.
	Turkey was substituted here for ham as an experiment for thanksgiving, partly due to availability.
	The stock can also be used to make gravy and can be very flavorful.
\end{recipeabstract}


\ragmarpar{smoked turkey necks works best}
\begin{ingredientcolumns}
	\begin{ingredientblock}
		\ingredient[1][lb]{collards}\\
		\ingredient[2][]{turkey necks}\\
		\ingredient[6][C]{water}\\
		\ingredient[2][Tbs]{butter}
	\end{ingredientblock}
	\begin{ingredientblock}
		\ingredient[2][Tbs]{vinegar}\\
		\ingredient[1][tsp]{sugar}\\
		\ingredient[1][tsp]{pepper}\\
		\ingredient[1][tsp]{hot sauce}
	\end{ingredientblock}
\end{ingredientcolumns}


\begin{preparation}
\item Make stock with turkey necks: cover with cold water, cook over high heat, bring to simmer, skim off foam from surface, reduce to $\approx 180$ \Fahrenheit. Simmer for $1 \ldots 4$ \hour.

\item Prepare collards greens ($\approx 15$ \minute~prior to removing turkey necks).
	Clean \& rinse collard greens, cut off the main stem, dice $\|$ tear greens (into $\approx 1x1"$ rectangles).

\item Remove turkey necks and sieve out undesired particulates.

\item Add greens, simmer $\approx 1$ \hour uncovered.

\item Meanwhile, remove and dice meat from turkey necks according to your preference. Add to pot.

\item Add butter, vinegar, spices, serve.
\end{preparation}

\begin{experiments}
	\item This was more of a quick experiment, it would be good to find a more traditional recipe.
	\item The stock was good but removing the meat was quite labor intensive for Thanksgiving day. This recipe might benefit from making the stock ahead of time and giving it a bit more spices \& vegetables.
\end{experiments}


\recipeend

\begin{recipe}[
preparationtime = 25 minutes,
bakingtime = 25 minutes,
source = The Gieskens,
]
{Classic Macaroni and Cheese}

\ingredients {
\unit[3]{C} & macaroni, uncooked\\
\unit[1/4]{C} & butter\\
\unit[3]{T} & flour \\
\unit[2]{C} & milk\\
\unit[8]{oz} & cream cheese\\
\unit[1/4]{tsp} & salt\\
\unit[1/4]{tsp} & pepper\\
\unit[2]{tsp} & mustard, Dijon\\
\unit[2 1/2]{C} & cheddar, sharp
}

\preparation {
\step Preheat oven to $400 \textfahrenheiht$, remove cream cheese from fridge to soften, shred cheese.
\step Cook macaroni al dente, drain, set aside.
\step Meanwhile, melt butter over medium heat then stir in flour and cook until bubbling. Stir in milk, cream cheese, salt, pepper, and mustard until thick. Stir in macaroni and cheese.
\step Add to $\approx9x13"$ or $\approx10x15"$ pan, bake at $400 \textfahrenheiht$ for $[20...25]\; minutes$
}

\hint{
\begin{itemize}
\item Other cheeses are available but sharp cheddar is important to this dish.
\end{itemize}
}

\end{recipe}


\chapter{Entr\`{e}es}
\clearpage
\section[Beef Bowl]{Mexican Beef Bowl}\label{mexican_beef_bowl}
\begin{recipestats}[
	servings=4 people,
	preptime=30 minutes,
	bakingtime=30 minutes,
	source=\citeauthor{blueApronBeefBowl} \cite{blueApronBeefBowl}
	]
\end{recipestats}

\begin{ingredientcolumns}
	\begin{ingredientblock}
		\ingredient[1][lb]{ground beef}\\
		\ingredient[\onehalf][lb]{tomatoes, cherry}\\
		\ingredient[\onehalf][lb]{carrots}\\
		\ingredient[1]{lime}\\
		\ingredient[\onefourth][C]{jalepe\~{n}os, pickled}\\
		\ingredient[\onefourth][C]{mayonnaise}\\
		\ingredient[\approx 8][oz.]{cheese, cotija}\\
		\ingredient[\approx 1][C]{rice}\\
		\ingredient[\approx 3][\Tablespoon]{spice blend}
	\end{ingredientblock}
	\begin{ingredientblock}[spice blend]
		\ingredient[2][\Tablespoon]{guajillo}\\
		\ingredient[1][\Tablespoon]{ancho}\\
		\ingredient[1][\Tablespoon]{paprika, smoked}\\
		\ingredient[1][\Tablespoon]{cumin}\\
		\ingredient[1][\Tablespoon]{marjoram}\\
		\ingredient[\onehalf][\Tablespoon]{garlic powder}\\
		\ingredient[\onehalf][\Tablespoon]{salt}
	\end{ingredientblock}
\end{ingredientcolumns}

\ragmarpar{for dried chilis: remove seeds, toast $\approx3\;min$ \@ $350\; \Fahrenheit$, grind}

\begin{preparation}
\item Prepare ground spice blend; ahead of time if preferred.

\item Begin to brown the beef in large saut\'{e} pan, stirring occasionally. Reach a strong color at end of recipe.

\item Dice carrots, toss with oil and spices. Preheat oven to $450 \Fahrenheit$.

\item Begin cooking rice.

\item Roast carrots in oven on a cookie sheet, $12...14\;min$.

\item Dice tomatoes, finely dice jalepe\~{n}os, combine with juice of 1/2 of lime, pickled jalepe\~{n}o juice to taste.

\item Add spices to beef ($\approx 3/4\; Tbs$) at end, cook $\approx1\; min$.\\Add $1/4\;C$ water, cook $2...3\;min$.

\item Combine mayonnaise and juice of 1/2 lime.\\Add pickled jalepe\~{n}o juice to taste.

\item Serve by layering rice, beef, vegetables, pickled jalepe\~{n}os, cheese, mayonnaise.
\end{preparation}


\begin{variation}
\item Substitute riced cauliflower in place of rice. Cut into small pieces, chop in food processor with garlic, optionally pan fry.
\item Substitute Chorizo in place of beef, reduce spices.
\item Cotija cheese is worth the effort to find it, but mozzarella works.
\item For convenience, regular pre-ground chili powder can replace the guajillo. Similarly ancho powder is readily available.
\end{variation}

\recipeend

\section{Fettuccine Alfredo}\index{pasta!Fettucine Alfredo}


\begin{recipestats}[
	servings=2 people,
	preptime=15 \minute,
	bakingtime=20 \minute,
	source=Mike \& Jane,
	original=\citetitle{newCookBook2014}~\cite{newCookBook2014},
	]
\end{recipestats}

\begin{ingredientcolumns}
	\begin{ingredientblock}
		\ingredient[8][\ounce]{fettuccine, dry}\\
		\ingredient[4][\ounce]{mushrooms}\\
		\ingredient[2][\ounce]{Asiago}\\
		\ingredient[\approx~3][cloves]{garlic}
	\end{ingredientblock}
	\begin{ingredientblock}
		\ingredient[3][\Tablespoon]{butter, unsalted}\\
		\ingredient[1][C]{cream, heavy}\\
		\ingredient[\onehalf][\teaspoon]{salt}\\
		\ingredient[\oneeighth][\teaspoon]{pepper}
	\end{ingredientblock}
\end{ingredientcolumns}


\begin{preparation}
\ragmarpar{Stir mushrooms occasionally, cook to well done.}
\item Slice mushrooms, begin to saut\`{e} on medium w/ dash of oil.

\item Cook pasta in salted water to al dente, then toss with oil.
	Meanwhile grate chese, crush garlic.

\item Saut\`{e} crushed garlic in butter in a large saucepan $\approx1$ \minute~on medium-high.

\ragmarpar{Use a bit of the pasta water to thicken the sauce.}
\item Add cream, salt, pepper to sauce. Bring to boil then reduce heat, simmer uncovered $\approx3$ \minute~or until it begins to thicken.

\item Remove from heat add cheese \& mushrooms.

\item Add pasta to sauce, toss to combine.
\end{preparation}


\begin{variation}
\ragmarpar{Pecorino Romano does not melt well for use in the sauce.}
\item Try Parmesan instead of Asiago, mushrooms are optional.

\item For shrimp alfredo add $\approx8~\ounce$ prior to removing from heat, cook through and continue.
\end{variation}


\recipeend

\section{Chistmas Rib Roast}
\index{beef!Rib~Roast}


\begin{recipestats}[
	servings=$2\;person \; / \; 1 \; lb$,
	preptime=1 \onefourth~ \hour,
	bakingtime=2 \onefourth~ \hour,
	source=Ralph Nelson (Fa),
	]
\end{recipestats}

\ragmarpar{Fa's rule is that you MUST NOT open the oven under and circumstances.}

\begin{recipeabstract}
	A Giesken Christmas tradition.
\end{recipeabstract}

\begin{ingredientcolumns}[1]
	\begin{ingredientblock}
		\ingredient{standing rib roast}\\
		\ingredient[\approx 1][\Tablespoon]{pepper}\\
		\ingredient[\approx 1][\teaspoon]{salt}
	\end{ingredientblock}
\end{ingredientcolumns}

\begin{preparation}
\item Let stand at room temperature for one hour.

\item Rub black pepper \& salt on roast. Preheat oven to $400$ \Fahrenheit.

\item Place roast on pan fat side up. Do not cover or add water.

\item Bake 15 \minute~ at 400 \Fahrenheit.

\item Lower to 375 \Fahrenheit, bake for 45 \minute.

\item Turn off heat, do not open oven. Bake for about 30 \minute.


\item Turn on oven to 375 \Fahrenheit~$[35 \dots 45]$ \minute~before eating.
\end{preparation}

\recipeend

\begin{recipe}[
	preparationtime = 30 minutes,
	bakingtime = 2 hours,
	portion = 4 people,
	source = Rick Martinez; Bon Apetit,
	]
	{Chili Colorado}
	\graph{
		smallpicturewidth = 0.3\textwidth,%
		bigpicturewidth = 0.6\textwidth,%
	}
	%%%%%%%%%%%%%%%%%%%%%%%%%%%%%%%%%%%%%%%%%%%%%%%%%%%%%%
	\ingredients{
		\unit[5]{} & ancho\\
		\unit[2]{} & pasilla\\
		\unit[2]{} & guajillo\\
		\unit[8]{C} & stock, chicken\\
		\unit[2]{lb} & pork shoulder\\
		\unit[6...9]{} & garlic cloves\\
		\unit[2]{} & bay leaves\\
		\unit[1]{Tbs} & cumin, ground\\
		\unit[2]{tsp} & sage, fresh\\
		\unit[2]{tsp} & oregano, Mexican
	}
	%%%%%%%%%%%%%%%%%%%%%%%%%%%%%%%%%%%%%%%%%%%%%%%%%%%%%%
	\preparation{
		\step Measure the spices, chop the sage, and crush the garlic.
		\step Prepare the pork. Cut into $\approx1\;inch$ cubes, toss with salt, pepper.
		\step Brown the pork. Heat a neutral oil almost to smoking point in a $ \geqq 3.5\;quart$ pot. Reduce to medium high. Brown in batches so as to not overcrowd, de-glazing if necessary to prevent burning.
		\step Add spices, stir for about a minute.
		\step Add $5\;Cups$ stock, simmer uncovered for $1\;hour$.
		\step Meanwhile,re-hydrate the chilies. Remove stems, seeds, veins from chilies and roughly chop. Add to large bowl, add $3\;Cups$ boiling stock, cover with plastic wrap. Wait $30\;minutes$, then blend it all.
		\step Add blended chilies to the soup at the end of the first simmer. Simmer for $45\;minutes$ uncovered.
		\step Season with salt pepper to taste.
	}
	%%%%%%%%%%%%%%%%%%%%%%%%%%%%%%%%%%%%%%%%%%%%%%%%%%%%%%
	
	\hint{
		\begin{itemize}
			\item "Chili Colorado" means "chili colored red" rather than from the state of Colorado.
			\item Toss pork with a bit of flour as well to thicken the chili further.
			\item Marjoram can be a substitute for the Mexican oregano if necessary, but not Mediterranean oregano.
			\item Serve with tortillas, and the carrots from 'Mexican Beef Bowl' \ref{mexican_beef_bowl}. Rick Martinez recommends rice, beans a la charra, and tortillas.
		\end{itemize}
	}
	%%%%%%%%%%%%%%%%%%%%%%%%%%%%%%%%%%%%%%%%%%%%%%%%%%%%%%
	\setRecipeLengths{
		preparationwidth = 0.60\textwidth,
		ingredientswidth = 0.35\textwidth,
		pictureheight = 6cm,
		bigpicturewidth = 0.6\textwidth,
		smallpicturewidth = 0.35\textwidth
	}
	%%%%%%%%%%%%%%%%%%%%%%%%%%%%%%%%%%%%%%%%%%%%%%%%%%%%%%
	\setRecipeSizes{
		recipename = \fontsize{25pt}{30pt},
		ing = \normalsize,
		inghead = \normalsize,
		prep = \normalsize,
		prephead = \normalsize,
		hint = \normalsize,
		hinthead = \Large
	}
	%%%%%%%%%%%%%%%%%%%%%%%%%%%%%%%%%%%%%%%%%%%%%%%%%%%%%%
	%\setRecipenameFont{
	%%pbsi
	%%fau
	%%fwb
	%%fjd % default when using the option handwritten
	%cmr % default
	%}{T1}{m}{n}
	%%%%%%%%%%%%%%%%%%%%%%%%%%%%%%%%%%%%%%%%%%%%%%%%%%%%%%
	\setHeadlines{
		inghead = Ingredients,
		prephead = Preparation,
		hinthead = Hint,
		calory = Cal,
		continuationhead = Continuation,
		continuationfoot = Continuation on next page
	}
	%%%%%%%%%%%%%%%%%%%%%%%%%%%%%%%%%%%%%%%%%%%%%%%%%%%%%%
	%\setBackgroundPicture
	%[%
	%x = 2cm,
	%y = -1cm,
	%width=\paperwidth-3cm,
	%height,
	%orientation=pagecenter
	%]{pic/bg_transparent} % filepath
\end{recipe}

%\clearpage
%\thispagestyle{empty}
% \begin{tikzpicture}[remember picture,overlay]
% inelegant way of getting a good image:
% use [keepaspectratio], trim to an aspect ratio close to the page, then remove [keepaspectraio] to get full page
%   \node at (current page.center) {\includegraphics[width=\pdfpagewidth,height=\pdfpageheight,clip,trim={47px 10px 47px 10px}]{tricolor}};
% IMAGE: http://www.homemadeitaliancooking.com/italian-rainbow-cookies/
%\end{tikzpicture}
\begin{recipe}[
	preparationtime = 1 hour,
	bakingtime = 1 hour,
	portion = 4 people,
	]
	{Chili Schwarz}
	\graph {
		smallpicturewidth = 0.3\textwidth,%
		bigpicturewidth = 0.6\textwidth,%
	}

	\ingredients {
		\unit[3]{} & Ancho\\
		\unit[3]{} & Pasilla\\
		\unit[1]{} & Guajillo\\
		\unit[8]{C} & stock / broth\\
		\unit[1...2]{lb} & ground beef\\
		\unit[1]{} & onion, red\\
		\unit[1...2]{} & Pablano\\
		\unit[3]{} & carrots\\
		\unit[3]{} & garlic cloves\\
		\unit[3]{} & black garlic cloves\\
		\unit[2]{} & bay leaves\\
		\unit[1]{Tbs} & cumin, ground\\
		\unit[2]{tsp} & oregano, Mexican
	}
	
	\preparation
	{
		\step Preheat oven to $400\;F^\circ$. Remove seeds / veins / stems of chilies. Squash garlic and cut off the ends. Toast chilies for $\approx 1...2$ on a sheet (make certain not to burn them).
		\step Add chilies \& garlic to a bowl, cover w/ boiling water. Keep submerged, cover w/ plastic wrap, steep for $20...30\;min$.
		\step Meanwhile begin browning the beef. Achieve a strong color and deglaze often. Recommend 1/2 in a dutch oven, 1/2 in a pan, drain and reserve fat from pan. Deglaze with a toasty beer for a bit of flair.
		\step Julienne the onion, lightly saut\`{e} in pan. Meanwhile slice the Pablano, roughly chop the carrots, set aside.
		\step When the chilies are done remove from the water into a food processor. Add black garlic. Blend well for $\approx5\;min$.
		\step Add cumin to beef, stir $\approx 1\;min$, add chili sauce.
		\step Add onion, broth / stock, bay leaf, oregano. Bring to simmer. Simmer for $\approx30...60\;min$.
		\step Meanwhile, saut\`{e} the carrot, add to chili, repeat with Pablano. Season with salt.
	}
	
	\hint{
		\begin{itemize}
			\item Consider adding ground Arbol to leftovers which will mellow out.
		\end{itemize}
	}
	%%%%%%%%%%%%%%%%%%%%%%%%%%%%%%%%%%%%%%%%%%%%%%%%%%%%%%
	\setRecipeLengths{
		preparationwidth = 0.60\textwidth,
		ingredientswidth = 0.35\textwidth,
		pictureheight = 6cm,
		bigpicturewidth = 0.6\textwidth,
		smallpicturewidth = 0.35\textwidth
	}
	%%%%%%%%%%%%%%%%%%%%%%%%%%%%%%%%%%%%%%%%%%%%%%%%%%%%%%
	\setRecipeSizes{
		recipename = \fontsize{25pt}{30pt},
		ing = \normalsize,
		inghead = \normalsize,
		prep = \normalsize,
		prephead = \normalsize,
		hint = \normalsize,
		hinthead = \Large
	}
	%%%%%%%%%%%%%%%%%%%%%%%%%%%%%%%%%%%%%%%%%%%%%%%%%%%%%%
	%\setRecipenameFont{
	%%pbsi
	%%fau
	%%fwb
	%%fjd % default when using the option handwritten
	%cmr % default
	%}{T1}{m}{n}
	%%%%%%%%%%%%%%%%%%%%%%%%%%%%%%%%%%%%%%%%%%%%%%%%%%%%%%
	\setHeadlines{
		inghead = Ingredients,
		prephead = Preparation,
		hinthead = Hint,
		calory = Cal,
		continuationhead = Continuation,
		continuationfoot = Continuation on next page
	}
	%%%%%%%%%%%%%%%%%%%%%%%%%%%%%%%%%%%%%%%%%%%%%%%%%%%%%%
	%\setBackgroundPicture
	%[%
	%x = 2cm,
	%y = -1cm,
	%width=\paperwidth-3cm,
	%height,
	%orientation=pagecenter
	%]{pic/bg_transparent} % filepath
\end{recipe}

%\clearpage
%\thispagestyle{empty}
% \begin{tikzpicture}[remember picture,overlay]
% inelegant way of getting a good image:
% use [keepaspectratio], trim to an aspect ratio close to the page, then remove [keepaspectraio] to get full page
%   \node at (current page.center) {\includegraphics[width=\pdfpagewidth,height=\pdfpageheight,clip,trim={47px 10px 47px 10px}]{tricolor}};
% IMAGE: http://www.homemadeitaliancooking.com/italian-rainbow-cookies/
%\end{tikzpicture}

\section{Hoola Poola}
\begin{recipestats}[
	servings=1 person,
	preptime=10 minutes,
	bakingtime=20 minutes,
	]
\end{recipestats}

\begin{recipeabstract}
	This was created for a cost-effective lunch.
	The ingredients keep a long time and can mostly be prepped before hand.
	The name is a play on the Giesken's spam \& eggs recipe Hunka Punka. % FUTURE add ref to Hunka Punka
\end{recipeabstract}

\begin{ingredientcolumns}
	\begin{ingredientblock}
		\ingredient[4][\ounce]{Spam}\\
		\ingredient[2][C]{collards}\\
		\ingredient[1][C]{chickpeas, cooked}\\
		\ingredient[1][C]{rice, cooked}
	\end{ingredientblock}
	\begin{ingredientblock}
		\ingredient[\approx \onehalf][\teaspoon]{Sa\'zon}\\
		\ingredient[1][\Tablespoon]{butter}\\
		\ingredient[\onehalf][\Tablespoon]{garlic, minced}
	\end{ingredientblock}
\end{ingredientcolumns}


\ragmarpar{Cooking the collards until crispy (instead of stewed) is sometimes referred to as ``Brazillian'' collards}
\begin{preparation}
\item Dice and add Spam to frying pan over medium heat. Optionally add mushrooms, or other customization. Stir occasionally.

\item Meanwhile prepare the collards. Discard woody stems, slice leaves ($\approx1x1"$), let soak in water w/ a dash of vinegar and salt. Set aside remaining ingredients.

\item Once Spam is toasted, drain majority of water from collards, add to pan, add Sa\'zon. Stir often and cook until collards begin to get crispy.

\item Once collards are toasted, reduce heat a bit, expose center of pan by pushing mixture aside and add butter and garlic. Cook to a light toast ($\approx[30\dots 60]\; sec$).

\item Serve with rice. Top with a dash of Sa\'zon and yell ``HOOLA POOLA!'' for a bit of pizzazz.
\end{preparation}

\recipeend


\chapter{Bread}
\clearpage
\section[Focaccia]{Handshake Focaccia}

\begin{recipestats}[
	servings=6 people,
	preptime=3 hours,
	bakingtime=15 minutes,
	source=\citefield{howToBake2013}{title} \cite{howToBake2013},
	]
\end{recipestats}

\begin{recipeabstract}
	A Focaccia worthy of the Paul Hollywood Handshake.
	This is best eaten out of the oven and dipped in olive oil with a little salt and or-A-gaano (as they say in Britain).
\end{recipeabstract}

\begin{ingredientcolumns}
	\begin{ingredientblock}
		\ingredient[500][g]{bread flour}\\
		\ingredient[10][g]{salt}\\
		\ingredient[10][g]{yeast, instant}
	\end{ingredientblock}
	\begin{ingredientblock}
		\ingredient[140][ml]{olive oil}\\
		\ingredient[360][ml]{water}
	\end{ingredientblock}
\end{ingredientcolumns}

\begin{preparation}
\item Mix the dough.
Add flour to mixing bowl, add salt \& yeast on opposite sides.
Add $40ml$ oil \& $\approx 3/4$ water, hand mix.
Continue mixing and gradually add water until all the flour is incorporated; water may be left over.
Aim for a soft / wet dough.

\item Knead the dough.
Add some oil to the working surface.
Add dough to surface, knead $\approx 5...10\;min$.
Knead past the wet stage until the exterior is smooth \& soft.
Refrain from adding more flour.

\item Rise the dough.
Move dough to a lightly oiled $\approx 2..3\; quart$ tub.
Add tea towel on top and rise $\approx 1\;hr$ until at least doubled in size.

\item Separate the dough.
Line baking parchment to two trays, drizzle olive oil on top.
Add olive oil to the working surface, optionally dust w/ fine semolina.
Move dough to working surface slowly as to keep air in the dough.
Divide dough in half and stretch out flat onto the trays.

\item Prove the dough.
Add each tray into a plastic bag and prove for $\approx 1 hr$, until it has doubled in size.
The dough should spring back quickly.
Preheat oven to $430^\circ \; F$.

\item Bake.
Add dimples on top of the dough using your fingers; push all the way to the bottom.
Drizzle each with olive oil, top with flaked salt and oregano.
Bake $\approx 15 \; min$.
The bread should be cooked through so that tapping the bottom will sound hollow.
Drizzle with olive oil, cool.

\end{preparation}

\recipeend

\chapter{Hot Sauce}
\clearpage
{
\section[Brined Hot Sauce]{Base Brined Hot Sauce}
\label{base_lacto_brine_hotsauce}

\begin{recipestats}[
	servings=12 \fluidounce,
	preptime=60 minutes,
	bakingtime=2 weeks,
	source=\citefield{fieryferments2017}{title} \cite{fieryferments2017}; \textit{Pickl-It},
	]
\end{recipestats}

\begin{recipeabstract}
	A base recipe for whole brined peppers.
	Select your peppers, spices and other accoutrement
	Peppers are fermented whole and then pur\`{e}ed rather than mashed prior to fermentation, see ``Mixed-Media Basic Mash'' \cite{fieryferments2017}.
\end{recipeabstract}

\begin{ingredientcolumns}[1]
	\begin{ingredientblock}
		\ingredient[1][32\; \fluidounce]{mason jar}\\
		\ingredient[1]{pickling weight}\\
		\ingredient[1]{air lock}\\
		\ingredient[\approx \onefourth][\fluidounce]{Star San}\\
		\ingredient[\approx48][\fluidounce]{water, un-chlorinated}\\
		\ingredient[\approx70][g]{salt, un-iodized}\\
		\ingredient[\approx8][\ounce]{peppers, fresh $\|$ dried}\\
		\ingredient{spices}\\
		\ingredient[\approx\onefourth][\teaspoon]{achiote}
	\end{ingredientblock}
\end{ingredientcolumns}

\begin{preparation}
\item Sanitize equipement with Star San or equivalent.
\item Prepare a $[5...5.3]\%$ brine by weight, $\approx3/4$ volume of jar.
\item Prepare dried peppers by removing stems, veins, and seeds. Keep flesh largely intact by slicing stem off and cutting once down length-wise, then unrolling the pepper.
\item Prepare fresh peppers by removing stems / seeds and cutting into large portions.
\item Add ingredients to jar. Prevent ingredients from floating up by starting with smaller pieces, then top off with a large dried pepper or cabbage leaf tucked down the sides of the jar. Reserve Achiote for post fermentation.
\item Add a fermentation weight to top of ingredients.
\item Add brine mixture, leaving $\approx1cm$ of head space. Remove ingredients floating on top of brine.
\item Add lid and air lock, making sure the lock vent is not covered in brine.
\item Ferment for $[1...2]$ weeks or up to many months, then place in fridge until ready to pur\`{e}e. Make certain that everything is covered in brine.
\item Separate brine and ingredients, pur\`{e}e with $\approx1/2\;Cup$ brine. Strain if desired; recommended if dried peppers are used. Optionally blend Achiote for color.
\item Add vinegar or brine to sauce to desired consistency, refridgerate.
\end{preparation}

\begin{experiments}
\item Prevent spoilage by keeping ingredients submerged and by not removing the lid (keep it anaerobic). SEE Fiery Ferments for troubleshooting. In general: you should throw out the batch if you see any fuzzy mold.
\item The heat level of the sauce lowers drastically over fermentation. Consider compensating for example with Thai, Pequin, or Arbol.
\item This recipe should probably have some cabbage in order to jump-start fermentation.
\end{experiments}


\recipeend
	\section[Red Hot Sauce]{Fermented Red Hot Sauce}
	\let\section\subsection
	\let\subsection\subsubsection
	
	% use \input (not \include) to remove page breaks
	\section{Red No. 3}

\begin{recipestats}[
	servings=12 \fluidounce,
	preptime=60 minutes,
	bakingtime=4 weeks,
	]
\end{recipestats}
\begin{ingredientcolumns}[1]
	\begin{ingredientblock}
		\ingredient[8]{fresno}\\
		\ingredient[6]{cherry}\\
		\ingredient[6]{thai}\\
		\ingredient[3]{guajillo}\\
		\ingredient[3]{arbol}
	\end{ingredientblock}
	\begin{ingredientblock}
		\ingredient[1]{shallot}\\
		\ingredient[1][\teaspoon]{Indian green pepper}\\
		\ingredient[\onefourth]{oil}\\
		\ingredient[\onefourth][\teaspoon]{achiote}
	\end{ingredientblock}
\end{ingredientcolumns}

\begin{preparation}
\item Follow the basic brine \ref{base_lacto_brine_hotsauce} with a $5.3\%$ brine.
\item Ferment $2$ weeks, refridgerate $2$ weeks, blend with $1/2\;Cup$ brine and white vinegar each, strain.
\end{preparation}

\begin{experiments}
	\item Look into using xantham gum to keep the emulsion in suspension longer
	\item Consider rinsing / re-hydrating the dried chilies
	\item Still needs more heat
	\item Consider adding some cabbage to help fermentation
\end{experiments}

\recipeend
}

\chapter{Ice Cream}
\clearpage
\section{Chocolate Ice Cream}

\begin{recipestats}[
	servings=5 $Cups$,
	preptime=preparation time,
	bakingtime=cooking time,
	source=\citefield{joyofcooking2006}{title} \cite{joyofcooking2006},
	]
\end{recipestats}

\ragmarpar{definitely use dutch process cocoa for a 'dark chocolate' flair}
\begin{ingredientcolumns}
	\begin{ingredientblock}
		\ingredient[2][C]{milk, whole}\\
		\ingredient[\threefourth][C]{sugar}\\
		\ingredient[4]{egg yolks}
	\end{ingredientblock}
	\begin{ingredientblock}
		\ingredient[\onethird][C]{cocoa powder}\\
		\ingredient[1][C]{heavy cream}\\
		\ingredient[1][\teaspoon]{vanilla}
	\end{ingredientblock}
\end{ingredientcolumns}


\begin{preparation}
\item Combine in saucepan over medium low heat the milk and $\onehalf$ cups sugar, bring to simmer stirring occasionally.

\item Whisk egg yolks and $\onefourth$ cups sugar in a medium bowl, whisk in cocoa.

\item Pour slowly while stirring constantly about half of the hot milk into the eggs. Pour back into the saucepan.

\item Cook stirring constantly over low heat until it reaches $175F^\circ$, and do not allow it to boil. Remove from heat.

\item Strain through a fine sieve into a bowl, then add cream \& vanilla. Refrigerate until cold. Proceed with ice cream machine directions.
\end{preparation}

\begin{variation}
	\item Marshmallow Oreo: 1 cup broken Oreos in mixer at end, fold in marshmallow fluff after mixing.
\end{variation}



\chapter{Cookies, Cakes}
\clearpage

\section{Italian Tricolors}

\begin{recipestats}[
	servings=36 cookies,
	preptime=1 hour (1.25 hour chilling),
	bakingtime=10 minutes,
	source=\citefield{goodHousekeeping_2013}{title} \cite{goodHousekeeping_2013},
	]
\end{recipestats}

\begin{recipeabstract}
	A Teresi Christmas tradition.
	The almond, apricot, and chocolate are quite complementary flavors.
	
\end{recipeabstract}

\ragmarpar{fresh almond paste is critical}
\begin{ingredientcolumns}[1]
	\begin{ingredientblock}
		\ingredient[8][\ounce]{almond paste}\\
		\ingredient[\threefourth][C]{butter}\\
		\ingredient[\threefourth][C]{sugar}\\
		\ingredient[\onehalf][\teaspoon]{almond extract}\\
		\ingredient[3][large]{eggs}\\
		\ingredient[1][C]{flour, all purpose}\\
		\ingredient[\onefourth][\teaspoon]{salt}\\
		\ingredient[15][drops]{red food coloring}\\
		\ingredient[15][drops]{green food coloring}\\
		\ingredient[\twothird][C]{apricot preserves}\\
		\ingredient[3][\ounce]{dark chocolate}\\
		\ingredient[2][\teaspoon]{shortening}
	\end{ingredientblock}
\end{ingredientcolumns}

\begin{preparation}
\item Preheat oven to $350$ \Fahrenheit, grease three $8x8"$ pans.
Line bottoms w/ waxed paper, grease and flour the interior.

\ragmarpar{it's ok if a few lumps remain}
\item Blend at medium-high speed: almond paste, butter, sugar, almond extract.
Reduce to medium and add eggs one-at-a-time.
Reduce to low and beat in flour \& salt until just combined.

\item Divide batter into thirds into separate bowls.
Blend green dye into one, red into another.

\item For each mixture. transfer and spread evenly into the pans.

\item Bake on two oven racks $10-12\; min$ rotating between upper/lower halfway through.

\item Cool in pans on wire racks $5\; min$.
Run knife around sides to loosen layers. Invert onto racks and cool completely; removing the paper when done.

\ragmarpar{add more shortening or corn syrup to the chocolate to make it easier to cut}
\item Blend jam in food processor to remove the larger chunks.

\item Assemble layers, green / white / red, with jam between.
Melt on low chocolate / shortening, stirring frequently.
Spread on top then refrigerate $\geq1$ $hour$.

\item Rest at room temperature for $\geq 15\; min$ then trim the edges and cut into squares.
Store cookies in a single layer in a tightly covered container.
Refrigerate $\approx 1\; week$ or freeze $\approx 3\; months$.
\end{preparation}

\begin{variation}
\item Other colors and jams are available. Fourth of July with red / white / blue / cherries works out nicely.
\end{variation}
\recipeend
%\section[Felix Cookies]{Felix Cookies, or,\\ \mbox{Schwarz-Wei\ss-Geb\"{a}ck}}


\begin{recipestats}[
	servings=40 cookies,
	preptime=30 \minute~(+chill),
	bakingtime=12 \minute,
	source=\citetitle{luisaWeiss2016} \cite{luisaWeiss2016},
	]
\end{recipestats}


\begin{recipeabstract}
	A shortbread cookie with a black and white checkerboard.
	Works very well with the addition of chocolate.
	Aliased for our black and white cats, Felix and Frankie.
\end{recipeabstract}


\begin{ingredientcolumns}
	\begin{ingredientblock}
		\ingredient[150][\gram]{butter, unsalted}\\
		\ingredient[75][\gram]{sugar, powdered}\\
		\ingredient[\oneeighth][\teaspoon]{salt}\\
		\ingredient[\onefourth][\teaspoon]{vanilla extract}
	\end{ingredientblock}
	\begin{ingredientblock}
		\ingredient[200][\gram]{flour, all purpose}\\
		\ingredient[1]{egg yolk}\\
		\ingredient[2][\Tablespoon]{whole milk}\\
		\ingredient[2\;\onehalf][\Tablespoon]{cocoa powder}
	\end{ingredientblock}
\end{ingredientcolumns}


\begin{preparation}
\ragmarpar{other shapes are available, like 6 petal flowers, spirals, etc.}
\item Cream butter $\approx 1$ \minute, add sugar, salt, vanilla, then cream.
Add flour and mix until just combined.

\item Divide dough in half, mix cocoa into one half.
Form into disks, wrap in plastic wrap, refrigerate $\approx1$ \hour.

\item Mix milk \& egg yolk in a small bowl.

\ragmarpar{don't over bake, try adding another sheet below to shield the radiation}
\item Make 4 square logs, brush sides w/ egg wash, press together and refrigerate $\approx 30$ \minute.

\item Preheat oven, line baking sheets w/ parchment paper.

\item Slice cookies to $\approx 1$ $cm$, bake $12 \dots 15$ \minute.
\end{preparation}


\begin{variation}
\item Dip into dark chocolate.
\end{variation}


\recipeend

%\begin{recipe}[
	bakingtime = 30 $min$,
	portion = $\sim60$ squares,
	source = Cookie Swap 2003,
	]
	{Peppermint Fudge}
	\graph{
		%small = strawberry,
		smallpicturewidth = 0.3\textwidth,
		%big = strawberrycake,
		bigpicturewidth = 0.6\textwidth,
	}
	%%%%%%%%%%%%%%%%%%%%%%%%%%%%%%%%%%%%%%%%%%%%%%%%%%%%%%
	\ingredients{
		\unit[4]{C} & sugar \\
		\unit[10]{oz} & evaporated milk\\
		\unit[1]{C} & butter\\
		\unit[2]{C} & chocolate chips\\
		\unit[7]{oz} & marshmallow creme\\
		\unit[1/2]{tsp} & peppermint extract\\
		\unit[2/3]{C} & peppermint candy
	}
	%%%%%%%%%%%%%%%%%%%%%%%%%%%%%%%%%%%%%%%%%%%%%%%%%%%%%%
	\preparation{
		\step Line a $13x9\; inch$ pan with foil and butter the interior.\\Crush peppermint candy.
		\step Combine sugar, milk, butter, in a $3\;quart$ saucepan.\\Bring to boil over medium-high heat, stirring constantly.
		\step Reduce to medium, stir to $10\; minutes$.
		\step Remove from heat, add chocolate chips, marshmallow creme, peppermint extract. Stir until chocolate and creme are melted and mixture is smotth.
		\step Pour into pan, sprinkle peppermint on top, cover, refridgerate until set.
	}
	%%%%%%%%%%%%%%%%%%%%%%%%%%%%%%%%%%%%%%%%%%%%%%%%%%%%%%
	\hint{
		\begin{itemize}
			\item 
		\end{itemize}
	}
	%%%%%%%%%%%%%%%%%%%%%%%%%%%%%%%%%%%%%%%%%%%%%%%%%%%%%%
	\setRecipeLengths{
		preparationwidth = 0.60\textwidth,
		ingredientswidth = 0.35\textwidth,
		pictureheight = 6cm,
		bigpicturewidth = 0.6\textwidth,
		smallpicturewidth = 0.35\textwidth
	}
	%%%%%%%%%%%%%%%%%%%%%%%%%%%%%%%%%%%%%%%%%%%%%%%%%%%%%%
	\setRecipeSizes{
		recipename = \fontsize{25pt}{30pt},
		ing = \normalsize,
		inghead = \normalsize,
		prep = \normalsize,
		prephead = \normalsize,
		hint = \normalsize,
		hinthead = \Large
	}
	%%%%%%%%%%%%%%%%%%%%%%%%%%%%%%%%%%%%%%%%%%%%%%%%%%%%%%
	%\setRecipenameFont{
	%%pbsi
	%%fau
	%%fwb
	%%fjd % default when using the option handwritten
	%cmr % default
	%}{T1}{m}{n}
	%%%%%%%%%%%%%%%%%%%%%%%%%%%%%%%%%%%%%%%%%%%%%%%%%%%%%%
	\setHeadlines{
		inghead = Ingredients,
		prephead = Preparation,
		hinthead = Hint,
		calory = Cal,
		continuationhead = Continuation,
		continuationfoot = Continuation on next page
	}
	%%%%%%%%%%%%%%%%%%%%%%%%%%%%%%%%%%%%%%%%%%%%%%%%%%%%%%
	%\setBackgroundPicture
	%[%
	%x = 2cm,
	%y = -1cm,
	%width=\paperwidth-3cm,
	%height,
	%orientation=pagecenter
	%]{pic/bg_transparent} % filepath
\end{recipe}

%\clearpage
%\thispagestyle{empty}
% \begin{tikzpicture}[remember picture,overlay]
% inelegant way of getting a good image:
% use [keepaspectratio], trim to an aspect ratio close to the page, then remove [keepaspectraio] to get full page
%   \node at (current page.center) {\includegraphics[width=\pdfpagewidth,height=\pdfpageheight,clip,trim={47px 10px 47px 10px}]{tricolor}};
% IMAGE: http://www.homemadeitaliancooking.com/italian-rainbow-cookies/
%\end{tikzpicture}

%\section{Seven Layer Brownies}

\begin{recipestats}[
	servings=9 brownies,
%	preptime=preparation time,
%	bakingtime=cooking time,
	source=Rand Pearson,
	]
\end{recipestats}

\begin{recipeabstract}
	A most excellent brownie recipe.
	The original was free-form so these instructions are more like guidelines.
	Try to find your inner muse and capture some of that magic.
\end{recipeabstract}

\begin{ingredientcolumns}
	\begin{ingredientblock}
		\ingredient[1][box]{Betty Crocker Brownie / Cookie combo mix}\\
		\ingredient[4][\Tablespoon]{butter}\\
		\ingredient[4 \dots 6]{Heath Bars}\\
		\ingredient{marshmallows}\\
		\ingredient{graham crackers}\\
		\ingredient{chocolate chips}\\
		\ingredient{peanut butter}
	\end{ingredientblock}
\end{ingredientcolumns}

\begin{preparation}
\item Mix up the brownie and cookie mix as instructed on the box.

\item Place brownie mix in a baking dish as instructed.

\item Coarsely chop Heath bar and spread evenly over brownie batter.

\item Place cookie dough mix on top of Heath bar layer.

\item Crush graham crackers and spread over cookie dough.

\item Melt half a stick of butter and pour over graham crackers.

\item If using chocolate chips spread a layer over graham crackers.
If using peanut butter melt it and pour over graham cracker.

\item Split marshmallows lengthwise and arrange a solid layer over the top of everything.
You might also use mini marshmallows to skip cutting them.

\item Bake as instructed on brownie box. Time may increase due to the extra layers.
Brownies are done when a toothpick comes out clean.
\end{preparation}


\begin{experiments}
\item Keep in mind that there really isn't a recipe since I just made it all up as I went along. What you should take from that is this: Feel free to experiment; any problems can be overcome with enough butter and sugar. I'm pretty sure this is also true in life.

\item In the future I was considering leaving the marshmallows off until I take the brownies out of the oven, then adding them ad hitting them with a torch.
\end{experiments}
\recipeend

%\section{Elegant Wine Cake}
\index{wine!Wine~Cake}


\begin{recipestats}[
	servings=2 loafs,
	preptime=10 \minute,
	bakingtime=45 \minute,
	source=Lucille Steinmiller (Oma),
	]
\end{recipestats}


\begin{recipeabstract}
	A Teresi Christmas tradition from Oma.
	We like to make these as small bundt cakes to share with family and friends.
\end{recipeabstract}


\begin{ingredientcolumns}[1]
	\begin{ingredientblock}
		\ingredient[1][pkg]{cake mix, yellow}\\
		\ingredient[1][pkg]{vanilla pudding, instant}\\
		\ingredient[\onehalf][C]{vegetable}\\
		\ingredient[4][large]{eggs}\\
		\ingredient[\threefourth][C]{sherry, medium}\\
		\ingredient[\threefourth][C]{water}\\
		\ingredient[1][C]{chopped nuts}
	\end{ingredientblock}
\end{ingredientcolumns}


\begin{preparation}
\item Grease and flour 2 loaf pans $\approx 9x5"$, or four smaller bundt pans.

\item Combine all ingredients in a bowl, beat 2 \minute~on medium speed.

\item Bake at 350 \Fahrenheit~for 45 \minute~or until done.

\item Top with powdered sugar.
\end{preparation}

\begin{variation}
\item Chablis can be substituted for the sherry.

\item Mom usually uses pecans, I  omit the nuts.
\end{variation}


\recipeend

%
%\clearpage
%\section{Pie}
%\section{Basic Flaky Pie Pastry}\label{basicFlakyPiePastry}
\index{pastry!Basic~Flaky~Pastry}


\begin{recipestats}[
	servings=1 pastry,
	preptime=45 \minute,
	bakingtime=1 \hour~(chill),
	source=\citefield{pie2004}{title} \cite{pie2004},
	]
\end{recipestats}


\begin{recipeabstract}
	An all purpose pie pastry.
	Very useful to make in large batches, freezing up to a month.
\end{recipeabstract}


\begin{ingredientcolumns}
	\begin{ingredientblock}
		\ingredient[1 \onehalf][C]{flour, all purpose}\\
		\ingredient[1 \onehalf][\teaspoon]{sugar}\\
		\ingredient[\onehalf][\teaspoon]{salt}
	\end{ingredientblock}
	\begin{ingredientblock}
		\ingredient[\onefourth][C]{butter, unsalted}\\
		\ingredient[\onefourth][C]{shortening}\\
		\ingredient[\onefourth][C]{water}
	\end{ingredientblock}
\end{ingredientcolumns}


\begin{preparation}
\item Cut fat into small pieces ($\approx 3/8"$ cubes), place in freezer briefly along with water until cold.

\item Mix flour, sugar, salt, butter, in a large bowl.
	Blend using a pastry cutter $\|$ fork $\|$ fingers, until the butter is pea sized.
	Blend the shortening similarly.

\item Add half the water and toss with fork.
	Add water $\approx 1.5 \dots 2$ \Tablespoon~at a time, and pull all the flour into the dough.
	Continue until the dough can be packed together.

\item Pack dough into a ball, knead once or twice.
	Flatten onto a floured surface into $\approx 3/4"$ disks.
	Wrap in plastic and refrigerate at least 1 \hour~or overnight.

\item Roll pastry onto wax paper, invert onto pan \& shape.
	Freeze for 15 \minute.

\item For a pre-baked crust: preheat to 400 \Fahrenheit, press aluminum foil on top of pastry and fill with pie weights.
	Bake 15 \minute, remove foil \& weights, prick holes into pastry base with fork to prevent bubbles.

\item Lower to 375 \Fahrenheit, bake $10 \dots 12$ \minute~for a partially pre-baked crust or $15 \dots 17$ \minute~for a fully prebaked crust.
\end{preparation}


\recipeend

%\section[Strawberry Rhubarb Pie]{Strawberry Rhubarb Crumb Pie}\index{strawberry!Strawberry~Rhubarb~Pie}\index{rhubarb!Strawberry~Rhubarb~Pie}


\begin{recipestats}[
	servings=1 pie,
	preptime=1 \onehalf~\hour,
	bakingtime=50 \minute,
	source=\citefield{pie2004}{title}~\cite{pie2004},
	]
\end{recipestats}


\begin{recipeabstract}
	I don't always have favorites but when I do it's pretty close to this pie.
	Make sure to capitalize on the spring season when rhubarb is available.
	There is a lot of liquid so I increased the tapioca and maceration time.
	Make sure to use a deep pan and high crust.
\end{recipeabstract}


\begin{ingredientcolumns}
	\begin{ingredientblock}[filling]
		\ingredient[3][C]{rhubarb}\\
		\ingredient[\threefourth][C]{sugar}\\
		\ingredient[1 \onehalf][\Tablespoon]{lemon juice}\\
		\ingredient[1][lemon]{zest}\\
		\ingredient[4][C]{strawberries}\\
		\ingredient[\onehalf][C]{tapioca, quick}
	\end{ingredientblock}
	\begin{ingredientblock}[topping]
		\ingredient[\threefourth][C]{flour, all-purpose}\\
		\ingredient[\onefourth][C]{cornmeal, yellow}\\
		\ingredient[\twothird][C]{sugar, brown}\\
		\ingredient[\onehalf][\teaspoon]{cinnamon}\\
		\ingredient[\onefourth][\teaspoon]{salt}\\
		\ingredient[\onehalf][C]{butter, unsalted}
	\end{ingredientblock}
\end{ingredientcolumns}


\begin{preparation}
\item Prepare~\ref{basicFlakyPiePastry} \nameref{basicFlakyPiePastry} and refrigerate $\geq 1$ \hour.

\ragmarpar{make a high crust to prevent spills}
\item Roll pastry onto wax paper $\approx 13''$ diameter, invert onto pie pan and shape.
	Freeze for 15 \minute.
	Preheat oven to $400$ \Fahrenheit.

\item Prepare filling.
	Slice rhubarb $\approx \onehalf''$ pieces, mix fruit w/ sugar, lemon juice, zest, tapioca.
	Quarter strawberries and mix in.
	Macerate for $\geq 15$ \minute.

\item Add filling evenly into crust, bake on center rack 30 \minute.

\item Meanwhile prepare topping.
	Combine flour, cornmeal, brown sugar, cinnamon, salt.
	Cut butter into pieces and blend in with food processor or pastry cutter.
	Make large crumbs by rubbing the mixture between your hands.
	Refridgerate.

\ragmarpar{place a cookie sheet underneath to catch spills}
\item Remove pie and reduce oven to 375 \Fahrenheit.
	Add crumbs to top of pie.
	Rotate pie $180^\circ$ (to bake evenly) and bake $[30..40]$~\minute.
	Add foil heat shield if needed for last $\approx 10$ \minute.
	Cool $\geq 1$ \hour.
\end{preparation}


\begin{variation}
\item Try a bit of ground green cardamom in the filling.
\end{variation}


\recipeend%

%\section[Lemon Meringue Pie]{Classic Lemon Meringue Pie}


\begin{recipestats}[
	servings=1 pie,
	preptime=1 \hour,
	bakingtime=20 \minute,
	source=\citefield{pie2004}{title} \cite{pie2004},
	]
\end{recipestats}


\begin{recipeabstract}
	A Teresi Thanksgiving \& Christmas tradition.
	This is perfect for large gatherings.
	It can be made the day prior, save for the meringue which is done before serving.
	It is light and tart which is excellent after a large meal.
\end{recipeabstract}


\begin{ingredientcolumns}
	\begin{ingredientblock}[filling]
		\ingredient[1 \onethird][C]{sugar}\\
		\ingredient[\threeeighth][C]{corn starch}\\
		\ingredient[\oneeighth][\teaspoon]{salt}\\
		\ingredient[2][C]{water}\\
		\ingredient[\onehalf][C]{lemon juice}\\
		\ingredient[1][\Tablespoon]{lemon zest}\\
		\ingredient[4][large]{egg yolks}\\
		\ingredient[2][\Tablespoon]{butter, unsalted}
	\end{ingredientblock}
	\begin{ingredientblock}[meringue]
		\ingredient[4][large]{egg whites}\\
		\ingredient[\onefourth][\teaspoon]{cream of tartar}\\
		\ingredient[1][pinch]{salt}\\
		\ingredient[\onehalf][C]{sugar, powdered}\\
		\ingredient[\onehalf][\teaspoon]{vanilla extract}
	\end{ingredientblock}
\end{ingredientcolumns}


\begin{preparation}
\ragmarpar{Fresh lemons are critical, you'll need $\geq 2$. Extra juice / zest is ok.}
\item Prepare pastry \ref{basicFlakyPiePastry} \nameref{basicFlakyPiePastry}, partially pre-bake, and let cool.
	Dice the butter into $\approx\onehalf$ $inch$ pieces.

\item Cook custard.
	Mix sugar, cornstarch, salt, in a saucepan.
	Add water, lemon juice \& zest.
	Whisk in egg yolks.
	Whisk nonstop over medium, heat, for $\approx 5\dots7$ \minute, until it boils.
	Reduce heat and continue whisking for $\approx 60 \dots 90$ \second.

\item Emulsify custard.
	Remove from heat, continue whisking, and add the butter slowly one piece at a time.

\item Cool custard.
	Pour custard into shell, let settle \& cool briefly.
	Press plastic wrap on top to keep air out and let cool to room temperature.
	Refrigerate and use within $1\; day$.

\item Prepare meringue prior to serving.
	Preheat broiler.
	Beat egg whites on medium-high to soft peaks, beat in cream of tartar \& salt.
	Slowly beat in sugar to firm or stiff peaks.
	Beat in vanilla briefly. Top the pie and broil.
\end{preparation}


\recipeend


\clearpage
\addcontentsline{toc}{chapter}{References}
\printbibliography

\end{document}
