% The Teresi family cookbook
%
% Collection of favorites from family / friends / the interwebz
%
% Design:
% - targets a book with a 6x9" trim
% - uses outer marginnotes and single column
%   (worked out better than both inner/outer notes for 6x9)
%   (worked out better than twocolumn for 6x9)
% - provides environments for formatting
% - targets a recipe format of
%   abstract, ingredients (1 | 2 col), steps, variations
% - images not included in this git repository for space
%
% Future:
% - add temperature command w/ compile time switch between Fahrenheit / Centigrade
% - add aliases for Tablespoon, teaspon, fluid oz, minutes etc
% - add keywords / index
% - add 'contents at a glance' (1 level) before contents
% - add a new package and move environments there?
% - add a compile time toggle for adding images
% - add a matrix of images for multiple recipes
% - add tessalation for page backround artwork
%   (SEE http://tex.my/creating-tiled-background-patterns/)
% - add a flag for changing justification on the margin paragraphs (e.g. use justify, use ragged)
% - add an alias for common measurements? e.g to replace \nicefrac{1}{4} etc.?

% Bugs:
% - there may be a line skip on the ingredient tables (on top) when using multicols
% - marginnote used the wrong side of the page, possibly after the preface was added? had to use marginpar
% - in xcookybooky couldn't use \cite[]{} (vs \cite{}) inside a recipe block? :  \begin{recipe}[source=\cite[]{}]{}

% Recipes:
% - can we get a southern style mac and cheese?
% - can we get a canonical southern style collards?

\documentclass[10pt]{book}
\newcommand\ningredientcol{2}  % variable for number of columns for ingredient list
\usepackage[T1]{fontenc}
\usepackage[latin1]{inputenc}

\usepackage{ragged2e}  % for \justify
\usepackage{marginnote}
\usepackage[%
showframe,%
twoside,%
top=0.75in,%
bottom=0.5in,%
marginparsep=3ex,%
marginparwidth=0.8in,%
inner=0.875in,%
outer=1.25in,%
papersize={6in,9in},%
]{geometry}

%\usepackage[cam,letter,center]{crop} % show trim lines


\usepackage{fancyhdr}
\pagestyle{empty}
\usepackage{lmodern}
\usepackage{lipsum}
\usepackage[ngerman, english]{babel}
\usepackage{graphicx}
\usepackage{xcolor}
\usepackage[clock, misc]{ifsym} % symbols
\usepackage{cookingsymbols}     % symbols e.g. \Oven, \Dish etc
\usepackage{tikzsymbols}        % symbols
\usepackage{fontawesome}        % symbols
\usepackage{xkeyval}
\usepackage{lettrine} % a large first letter/number
\usepackage{units}
%\usepackage{eso-pic} % for background pictures
%\usepackage{picture} % for modfifying the position of the bg pictures
\usepackage{tabularx} % line breaks in tabular
\usepackage{tikz}
\usepackage{nicefrac}
\usepackage{amssymb}
\usepackage{paralist}
\usepackage{multicol}
\usepackage{xparse}  % positional arg defaults
%\usepackage{tikzpagenodes}  % for printing marginnotes on correct side
\usepackage{pgfornament}  % symbols
\usepackage{nameref}      % ref section name etc. w/ \nameref
\usepackage{adforn}
\usepackage{enumitem}
\setlist{leftmargin=*}  % don't indent lists
\usepackage{longtable}
\usepackage{ifthen}

% NOTE what are the right settings for `nowidow` to work?
% adding \noclub to variation env doesn't seem to work
\usepackage[]{nowidow}  % wrecks tables so just load macros (no options)

% NOTE what are the right inputs to have sections moved to new page
% if they are too close to the bottom?
\usepackage[nobottomtitles]{titlesec}
%\titlespacing*{\section}{12pt}{48pt}{4pt}
%\usepackage[defaultlines=4,all]{nowidow}  % nowidow breaks the ingredient tables

\usepackage{sectsty}  % access sectional units
\usepackage{fancyhdr}
\usepackage[backend=bibtex]{biblatex}

\bibliography{bib}

\usepackage{hyperref}    % must be the last package
\hypersetup{%
	pdfauthor            = {Michael Teresi},
	pdftitle             = {Teresi Family Cookbook},
	pdfsubject           = {Recipes},
	pdfkeywords          = {recipes, recipe, cookbook, Teresi},
	pdfstartview         = {FitV},
	pdfview              = {FitH},
	pdfpagemode          = {UseNone}, % Options; UseNone, UseOutlines
	bookmarksopen        = {true},
	pdfpagetransition    = {Glitter},
	colorlinks           = {true},
	linkcolor            = {magenta},
	urlcolor             = {blue},
	citecolor            = {cyan},
	filecolor            = {black},
}

%\hbadness=10000	% Ignore underfull boxes

% FUTURE move to env variable?
\graphicspath{}  % use out of source path

% add an ornament to the section numbers
\makeatletter
\def\@seccntformat#1{\adforn{33} \csname the#1\endcsname\quad}
\makeatother

% show servings, time, etc. w/ icons
\ExplSyntaxOn
% keys
\keys_define:nn { mymodule/recipestats }
{
	servings .tl_set:N = \l_mymodule_servings_tl,
	preptime .tl_set:N = \l_mymodule_preptime_tl,
	bakingtime .tl_set:N = \l_mymodule_bakingtime_tl,
	source .tl_set:N = \l_mymodule_source_tl,
}
\NewDocumentEnvironment{recipestats}{O{}}
{
	\keys_set:nn { mymodule/recipestats } { #1 }
	% BUG a multicols doesn't left justify the icons?
	% FUTURE using a multicols may be preferred so that missing icons aren't just blank cells
	\begin{tabular}{clcl}
		\tl_if_empty:NTF \l_mymodule_servings_tl
		{& &}
		{\Dish & \l_mymodule_servings_tl &}
		
		\tl_if_empty:NTF \l_mymodule_source_tl
		{& \\}
		{\PaperLandscape & \l_mymodule_source_tl \\}
		
		\tl_if_empty:NTF \l_mymodule_preptime_tl
		{& &}
		{\Gloves & \l_mymodule_preptime_tl &}
		
		\tl_if_empty:NTF \l_mymodule_bakingtime_tl
		{& \\}
		{\oven & \l_mymodule_bakingtime_tl \\}
}
{
	\end{tabular}
	\newline  % NOTE is there a better way?
}
\ExplSyntaxOff

% italicize the abstract
\newenvironment{recipeabstract}%
{
	\itshape
}
{

}

% a 3 col tabular for ingredients
% can be used to specify which componenet the ingredients are for
% can be used to have multiple tabulars in columns
% Args:
%    (str): name of ingredient 'section', no name if None
\ExplSyntaxOn % for \tl_if_blank:nTF
\NewDocumentEnvironment{ingredientblock}{O{}}
{
	% default args:
	% https://tex.stackexchange.com/questions/29973/more-than-one-optional-argument-for-newcommand
	\tl_if_blank:nTF {#1}
	{}  % empty
	{\textit{#1}\\}  % not empty
	\begin{tabular}[t]{rl>{\bfseries}l}
}
{
	\end{tabular}
}
\ExplSyntaxOff % for \tl_if_blank:nTF

% wrapper around a multicols
% Args:
%     (int): number of columns, 2 if None
\NewDocumentEnvironment{ingredientcolumns}{O{2}}
{
	\setlength\columnsep{1ex}
	% NOTE is there a better way to fix the spacing?
	% The 1col ingredientcolumns has weird spacing so add a skip
	\ifthenelse{\equal{#1}{1}}{\vskip1ex}{}
	\centering
	\noindent
	\begin{multicols}{#1}
}
{
	\end{multicols}
	\par
	\ifthenelse{\equal{#1}{1}}{\vskip1em}{}
}

% ingredient for a 3 column tabular 
% intended for: quantity & unit & food
% default empty quantity and unit:
% e.g. \food[col1][col2]{col3}
% e.g. \food[quantity][unit]{name}
% e.g. \food{<ingredient name>}
\NewDocumentCommand{\ingredient}{O{} O{} m }{
	\ensuremath{#1} & \ensuremath{#2} & #3
}

\newenvironment{preparation}%
{
	% can also be controlled globally using
	% \setlist[enumerate]{<options>}
	\begin{enumerate}[topsep=0pt,itemsep=0ex,partopsep=0ex,parsep=0.5ex,label=\textbf{\arabic*}.]
	\setcounter{enumi}{-1}
	}
{
	\end{enumerate}
	\par
	\vspace{2ex}  % NOTE is there a more canonical solution?
}

\newenvironment{experiments}%
{
	
	\textbf{EXPERIMENTS}%
	\begin{itemize}[label=\faFlask,topsep=0pt,itemsep=0ex,partopsep=1ex,parsep=1ex]
	}
	{
	\end{itemize}
	\par
	\vspace{2ex}  % NOTE is there a more canonical solution?
}

\newenvironment{variation}%
{
	\textbf{VARIATIONS}%
	\begin{itemize}[label=\adforn{28},topsep=0pt,itemsep=0ex,partopsep=1ex,parsep=1ex]
	}
{
	\end{itemize}
	\par
	\vspace{2ex}  % NOTE is there a more canonical solution?
}

% FUTURE wrap an input call to add this?
\newcommand{\recipeend}[0]{
	{\centering \pgfornament[scale=0.55]{88} \par}
}

% marginpar with ragged justification
% NOTE had to use marginpar b/c marginnote produce notes on wrong side (after adding preface?)
% https://latex.org/forum/viewtopic.php?t=6041
\newcounter{pl}
\newcommand\ragmarpar[1]{%
	\stepcounter{pl}\label{pl-\thepl}%
	\ifthenelse{\isodd{\pageref{pl-\thepl}}}%
	{\marginpar{\RaggedRight #1}}
	{\marginpar{\RaggedLeft #1}}
}


% abbreviations / aliases
\newcommand\onehalf{\nicefrac{1}{2}}
\newcommand\onethird{\nicefrac{1}{3}}
\newcommand\onefourth{\nicefrac{1}{4}}
\newcommand\oneeighth{\nicefrac{1}{8}}

\newcommand\twothird{\nicefrac{2}{3}}

\newcommand\threehalf{\nicefrac{3}{2}}
\newcommand\threefourth{\nicefrac{3}{4}}
\newcommand\threeeighth{\nicefrac{3}{8}}

\newcommand\Tablespoon{Tbs}
\newcommand\teaspoon{tsp}
\newcommand\Celsius{\ensuremath{C\^\circ}}
\newcommand\Fahrenheit{\ensuremath{F^\circ}}
\newcommand\ounce{\ensuremath{oz}}
\newcommand\fluidounce{\ensuremath{fl\;oz}}

\frenchspacing
% stop chapters / titles from having an extra blank page
\let\cleardoublepage=\clearpage

\pagestyle{fancy}
\newcommand{\changefont}{%
	\fontsize{9}{11}\selectfont
}
\fancyhf{}
\fancyhead[LE,RO]{\changefont \slshape \rightmark} %section
\fancyhead[RE,LO]{\changefont \slshape \leftmark} %chapter
\fancyfoot[C]{\changefont \thepage} %footer



%% wrapper around a multicols
%% Args:
%%     (int): number of columns, 2 if None
%\newsavebox\ltmcbox
%\NewDocumentEnvironment{ingredientmulticols}{O{2}}
%{
%	\vbox\bgroup
%	\setlength\columnsep{0ex}
%	\centering
%	\begin{multicols}{#1}
%		\setbox\ltmcbox
%			\makeatletter\col@number\@ne
%			\begin{longtable}{rl>{\bfseries}l}
%}
%{
%	\end{longtable}
%	\unskip
%	\unpenalty
%	\unpenalty
%	\egroup
%	\unvbox\ltmcbox
%	\end{multicols}
%}



% box for a multi column long table
% SEE https://tex.stackexchange.com/questions/161827/balanced-longtables-multicol-and-page-breaks
%\makeatletter
%\newsavebox\ltmcbox
%\newsavebox\xxbox
%\newenvironment{multicolslongtable}[1]{
%	\setbox\ltmcbox\vbox\bgroup
%	\col@number\@ne
%	\begin{longtable}{#1}
%	}
%	{
%	\end{longtable}
%	\unskip
%	\unpenalty
%	\unpenalty\egroup
%	\unvbox\ltmcbox
%}
%\makeatother


% environment to accept table input and produce multicol
%\NewDocumentEnvironment{ingredientmulticols}{O{2}}
%{
%	\savebox\xxbox\bgroup
%	\begin{minipage}{\textwidth}
%		\begin{multicols}{#1}
%			\begin{multicolslongtable}{| l | l | l |}
%			}
%			{
%			\end{multicolslongtable}
%		\end{multicols}
%	\end{minipage}
%	\egroup
%	\begin{multicols}{#1}
%		\centering
%		\usebox\xxbox
%	\end{multicols}
%}


% patch multicols to allow for one column
% SEE https://tex.stackexchange.com/questions/233866/one-column-multicol-environment
\let\multicolmulticols\multicols
\let\endmulticolmulticols\endmulticols
\RenewDocumentEnvironment{multicols}{mO{}}
{%
	\ifnum#1=1
	#2%
	\else % More than 1 column
	\multicolmulticols{#1}[#2]
	\fi
}
{%
	\ifnum#1=1
	\else % More than 1 column
	\endmulticolmulticols
	\fi
}


\begin{document}

\frontmatter
\begin{titlepage}
\onecolumn
\pagestyle{empty}
%% temporary titles
% command to provide stretchy vertical space in proportion
\newcommand\nbvspace[1][3]{\vspace*{\stretch{#1}}}
% allow some slack to avoid under/overfull boxes
\newcommand\nbstretchyspace{\spaceskip0.5em plus 0.25em minus 0.25em}
% To improve spacing on titlepages
\newcommand{\nbtitlestretch}{\spaceskip0.6em}
{
	\centering
	\bfseries
	\nbvspace[1]
	\Huge
	{\nbtitlestretch
		TERESI FAMILY COOKERY
	}\\
	\nbvspace[1]
	\footnotesize
	AN ONGOING COLLECTION OF FAVORITES FROM AROUND THE GLOBE\\
	\nbvspace[1]
	\Large MICHAEL TERESI \\ \small \& COMPANY\\
	\nbvspace[1]
	\hspace{1em}  % FUTURE a more legit way of lining it up?
	\IfFileExists{../cookbook_img/stein2.png}{\includegraphics[width=0.33\textwidth]{../cookbook_img/stein2}}{}
	\vfill
}
\end{titlepage}

\tableofcontents
\chapter*{Preface}
\addcontentsline{toc}{chapter}{Preface}
Welcome to the Teresi family cookbook.
This is a curated list of recipes from family, friends, and other sources.
Consider it an invitation for collaboration.

This is designed as an ongoing project.
It's written in \LaTeX\ to separate content from presentation and versioned in \texttt{Git} to track changes over a long period of time.

Said presentation is included below, see \texttt{recipe\_snippet.tex}.
Future changes to the layout will affect all recipes through the use of environments / commands provided there.

\begin{recipe}[
preparationtime = N hours,
bakingtime = N minutes,
source = A. Author
]
{Title}

\ingredients {
\unit[100]{g} & water\\
\unit[1]{} & ice cube
}

\preparation {
\step Step zero

\step Step ...

\step Step N
}

\hint{
\begin{itemize}
\item Hint zero
\item Hint ...
\item Hint N
\end{itemize}
}

\end{recipe}

\nopagebreak

\mainmatter

\chapter{Breakfast}
\clearpage
\begin{recipe}[
    preparationtime = 15 minutes,
    bakingtime = 1 hour,
    portion = 12,
    source = Anova Culinary
]
{Egg Bites, Sous Vide}
\graph{
    %small = strawberry,
    smallpicturewidth = 0.3\textwidth,
    %big = strawberrycake,
    bigpicturewidth = 0.6\textwidth,
}
%%%%%%%%%%%%%%%%%%%%%%%%%%%%%%%%%%%%%%%%%%%%%%%%%%%%%%
\ingredients{
\unit[12]{large} & eggs\\
\unit[1]{C} & Gruyere\\
\unit[1/2]{C} & cream cheese\\
\unit[1/4]{tsp} & salt\\
\unit[6]{slices} & bacon\\
\unit[12]{} & canning jars ($4oz$) \& lids\\
}
%%%%%%%%%%%%%%%%%%%%%%%%%%%%%%%%%%%%%%%%%%%%%%%%%%%%%%
\preparation{
\step Set the water bath for $172 F^\circ\;/\;77.8 C^\circ$.
\step Add a coating of butter / crisco / etc. to the jar interiors to allow egg bites to release.
\step Cook bacon \& cut slices in half. Grate cheese.
\step Add $1/2$ of a bacon slice to each jar.
\step Blend eggs, cheeses, salt. Add egg mixture to each jar.
\step Add lids to each jar and screw on lightly using only your fingertips. The goal is to allow air to release in order to prevent the jars from shattering in the bath.
\step Add jars to water bath and cook for $1hour$. Remove, cool, tighten lids, and refridgerate up to $\approx1 week$.
\step Reheat by:
\begin{inparaenum}[\itshape a)\upshape]
	\item microwave for $[1...1.5\; minutes$]
	\item invert \& remove, broil for a few minutes.
\end{inparaenum}
}
%%%%%%%%%%%%%%%%%%%%%%%%%%%%%%%%%%%%%%%%%%%%%%%%%%%%%%
\hint{
\begin{itemize}
\item The cream effects the texture of the eggs: milk for flan-like, cream for fluffier, cream \& cottage cheese ($50/50$) or cream cheese for in-between (but this will require more experimentation to verify).
\item Other flavors are available, such as tomato / basil, broccoli, red pepper, pickled jalepe\~{n}os. Experiment and add to this recipe.
\item This recipe may need some more salt, other resources recommend a ratio of $300g$ eggs (about 6), $300g$ cream, $3g$ salt, but omit the cheese.
\item This recipe might benefit from butter / olive oil beaten into the eggs.
\end{itemize}
}
%%%%%%%%%%%%%%%%%%%%%%%%%%%%%%%%%%%%%%%%%%%%%%%%%%%%%%
\setRecipeLengths{
preparationwidth = 0.60\textwidth,
ingredientswidth = 0.35\textwidth,
pictureheight = 6cm,
bigpicturewidth = 0.6\textwidth,
smallpicturewidth = 0.35\textwidth
}
%%%%%%%%%%%%%%%%%%%%%%%%%%%%%%%%%%%%%%%%%%%%%%%%%%%%%%
\setRecipeSizes{
recipename = \fontsize{25pt}{30pt},
ing = \normalsize,
inghead = \normalsize,
prep = \normalsize,
prephead = \normalsize,
hint = \normalsize,
hinthead = \Large
}
%%%%%%%%%%%%%%%%%%%%%%%%%%%%%%%%%%%%%%%%%%%%%%%%%%%%%%
%\setRecipenameFont{
%%pbsi
%%fau
%%fwb
%%fjd % default when using the option handwritten
%cmr % default
%}{T1}{m}{n}
%%%%%%%%%%%%%%%%%%%%%%%%%%%%%%%%%%%%%%%%%%%%%%%%%%%%%%
\setHeadlines{
inghead = Ingredients,
prephead = Preparation,
hinthead = Hint,
calory = Cal,
continuationhead = Continuation,
continuationfoot = Continuation on next page
}
%%%%%%%%%%%%%%%%%%%%%%%%%%%%%%%%%%%%%%%%%%%%%%%%%%%%%%
%\setBackgroundPicture
%[%
%x = 2cm,
%y = -1cm,
%width=\paperwidth-3cm,
%height,
%orientation=pagecenter
%]{pic/bg_transparent} % filepath
\end{recipe}

%\clearpage
%\thispagestyle{empty}
% \begin{tikzpicture}[remember picture,overlay]
%    inelegant way of getting a good image:
%    use [keepaspectratio], trim to an aspect ratio close to the page, then remove [keepaspectraio] to get full page
%   \node at (current page.center) {\includegraphics[width=\pdfpagewidth,clip,trim={0px 225px 0px 0px}]{eggbites}};
%\end{tikzpicture}
\section[Spinach Quiche]{Spinach Artichoke Quiche}
\begin{recipestats}[
	servings=4 people,
	preptime=45 minutes,
	bakingtime=45 minutes,
	source=\citefield{joyofcooking2006}{title} \cite{joyofcooking2006},
	]
\end{recipestats}

\begin{recipeabstract}
	A classic dish.
	We like to serve this as a side along with cardamom rice and have leftovers for breakfast.
\end{recipeabstract}

\begin{ingredientcolumns}
	\begin{ingredientblock}
		\ingredient[1][]{pie pastry}\\
		\ingredient[5][large]{eggs}\\
		\ingredient[1][C]{cream}\\
		\ingredient[2][C]{spinach}\\
		\ingredient[\nicefrac{3}{2}][C]{Asiago}\\
		\ingredient[6][\fluidounce]{artichokes}
	\end{ingredientblock}
	\begin{ingredientblock}
		\ingredient[1][]{onion, white}\\
		\ingredient[2][Tbs]{butter}\\
		\ingredient[2][tsp]{garlic}\\
		\ingredient[\onefourth][tsp]{salt}\\
		\ingredient[\onehalf][tsp]{pepper, white}
	\end{ingredientblock}
\end{ingredientcolumns}

\begin{preparation}
\item Remove ingredients from the refrigerator.
\item Slice onion, begin cooking over medium-low heat with a little oil while stirring occasionally.
\item Measure and set aside remaining ingredients. Roughly chop artichokes \& spinach, shred cheese, separate 1 egg yolk, add remaining eggs to cream, grind spices.
\item Butter pie pan, add pastry, partially pre-bake (see \ref{basicFlakyPiePastry}). Immediately brush with egg yolk when the pastry is removed from oven.
\item Meanwhile prepare filling. Whisk egg, cream, spices. Add spinach to onions. Add garlic and remaining butter in the last minute or so.
\item Layer into pie: cheese, vegetables, egg mixture. Optionally garnish w/ cayenne $\|$ chili flakes $\|$ paprika for some pizzazz.
\item Bake at $375\; F^\circ$ for $30\; min$, broil briefly, rest $10\; min$.
\end{preparation}


\begin{variation}
\item This recipe has a lot of leeway. Try marjoram, cayenne, or a different pepper corn for example. Asiago / artichoke / spinach, or, cheddar / spinach both work well.
\item Other milk varieties are available but heavy cream will make the best texture IMHO.
\end{variation}


\recipeend


\chapter{Soup}
\clearpage
\section[Turkey Gumbo]{Thanksgiving Turkey Gumbo}


\begin{recipestats}[
	servings=[4 \ldots 6] people,
	preptime=30 \minute,
	bakingtime=4 \onehalf~\hour,
	source=Pamela Teresi,
]
\end{recipestats}


\begin{recipeabstract}
	A Teresi post Thanksgiving tradition.
\end{recipeabstract}


\ragmarpar{Andoullie sausage works particularly well}
\begin{ingredientcolumns}
	\begin{ingredientblock}
		\ingredient[1][lb]{sausage}\\
		\ingredient[2][C]{Turkey}\\
		\ingredient[\nicefrac{1}{4}][C]{vegetable oil}\\
		\ingredient[\nicefrac{1}{4}][C]{flour}\\
		\ingredient[\nicefrac{3}{2}][C]{celery}
	\end{ingredientblock}
	\begin{ingredientblock}
		\ingredient[2][]{white onions}\\
		\ingredient[4][C]{stock}\\
		\ingredient[1][tsp]{gumbo fil\'{e}}\\
		\ingredient[\nicefrac{1}{2}][tsp]{salt}\\
		\ingredient[\approx 2][C]{rice}
	\end{ingredientblock}
\end{ingredientcolumns}


\begin{preparation}
\item Brown sausage, dice vegetables, shred turkey, set aside.

\item Make roux: combine flour \& oil, stirring constantly on medium heat, $10 \ldots 15$ \minute or until copper in color.

\item Add celery \& onions to roux, cook $5$ \minute.

\item Add turkey, sausage, stock. Simmer $\approx 4$ \hour.

\ragmarpar{Serve w/ hot sauce \& yeast rolls}
\item Meanwhile, prepare rice to serve when gumbo is complete

\item Add gumbo fil\'{e}, salt, pepper.
\end{preparation}


\recipeend

\section[Crab Bisque]{Cajun Crabmeat Bisque}
\begin{recipestats}[
	servings=5 people,
	preptime=45 minutes,
	bakingtime=45 minutes,
	source=Art of the Palate 2006 (Pamela Teresi),
	]
\end{recipestats}

\begin{ingredientcolumns}
	\begin{ingredientblock}
		\ingredient[\nicefrac{3}{4}][C]{butter}\\
		\ingredient[\nicefrac{3}{4}][C]{flour}\\
		\ingredient[3][Tbs]{tomato paste}\\
		\ingredient[\nicefrac{3}{2}][C]{onion, yellow}\\
		\ingredient[1][C]{celery}\\
		\ingredient[\nicefrac{1}{2}][C]{scallions}\\
		\ingredient[4][cloves]{garlic}\\
		\ingredient[\nicefrac{2}{3}][C]{green pepper}\\
		\ingredient[3][Tbs]{parsley}
	\end{ingredientblock}
	\begin{ingredientblock}
		\ingredient[2][quarts]{stock, chicken}\\
		\ingredient[1][Tbs]{Worcestershire}\\
		\ingredient[1][]{bay leaf}\\
		\ingredient[1][tsp]{thyme, dried}\\
		\ingredient[1][tsp]{salt}\\
		\ingredient[\nicefrac{1}{8}][tsp]{pepper, black}\\
		\ingredient[\nicefrac{1}{8}][tsp]{cayenne}\\
		\ingredient[\nicefrac{1}{2}][tsp]{ketchup}\\
		\ingredient[1][lb]{crabmeat}
	\end{ingredientblock}
\end{ingredientcolumns}

\begin{preparation}
\item Make roux: melt butter, gradually combine flour, stirring constantly on medium heat, $20...30\; min$ or until golden brown.

\item Add tomato, vegetables finely diced, sweat.

\item Add stock gradually. Add spices, crab meat.

\ragmarpar{Serve w/ hot sauce \& yeast rolls}
\item Simmer 40 minutes, covered, stirring occasionally. 
\end{preparation}

\begin{experiments}
\item Vary the amount of stock, crab; adding a quart of stock \& 8 $oz$ crab can work.
\end{experiments}

\recipeend

\chapter{Sides}
\clearpage
\section[Collard Greens]{Thanksgiving Collard Greens}
\index{collards!Collard~Greens}


\begin{recipestats}[
	servings=[4 \ldots 6] people,
	preptime=30 \minute,
	bakingtime=[2 \ldots 6] \hour,
	]
\end{recipestats}

\begin{recipeabstract}
	These are similar to southern-style collards in that they are simmered for a long time with meat.
	Turkey was substituted here for ham as an experiment for thanksgiving, partly due to availability.
	The stock can also be used to make gravy and can be very flavorful.
\end{recipeabstract}


\ragmarpar{smoked turkey necks works best}
\begin{ingredientcolumns}
	\begin{ingredientblock}
		\ingredient[1][lb]{collards}\\
		\ingredient[2][]{turkey necks}\\
		\ingredient[6][C]{water}\\
		\ingredient[2][Tbs]{butter}
	\end{ingredientblock}
	\begin{ingredientblock}
		\ingredient[2][Tbs]{vinegar}\\
		\ingredient[1][tsp]{sugar}\\
		\ingredient[1][tsp]{pepper}\\
		\ingredient[1][tsp]{hot sauce}
	\end{ingredientblock}
\end{ingredientcolumns}


\begin{preparation}
\item Make stock with turkey necks: cover with cold water, cook over high heat, bring to simmer, skim off foam from surface, reduce to $\approx 180$ \Fahrenheit. Simmer for $1 \ldots 4$ \hour.

\item Prepare collards greens ($\approx 15$ \minute~prior to removing turkey necks).
	Clean \& rinse collard greens, cut off the main stem, dice $\|$ tear greens (into $\approx 1x1"$ rectangles).

\item Remove turkey necks and sieve out undesired particulates.

\item Add greens, simmer $\approx 1$ \hour uncovered.

\item Meanwhile, remove and dice meat from turkey necks according to your preference. Add to pot.

\item Add butter, vinegar, spices, serve.
\end{preparation}

\begin{experiments}
	\item This was more of a quick experiment, it would be good to find a more traditional recipe.
	\item The stock was good but removing the meat was quite labor intensive for Thanksgiving day. This recipe might benefit from making the stock ahead of time and giving it a bit more spices \& vegetables.
\end{experiments}


\recipeend

\begin{recipe}[
preparationtime = 25 minutes,
bakingtime = 25 minutes,
source = The Gieskens,
]
{Classic Macaroni and Cheese}

\ingredients {
\unit[3]{C} & macaroni, uncooked\\
\unit[1/4]{C} & butter\\
\unit[3]{T} & flour \\
\unit[2]{C} & milk\\
\unit[8]{oz} & cream cheese\\
\unit[1/4]{tsp} & salt\\
\unit[1/4]{tsp} & pepper\\
\unit[2]{tsp} & mustard, Dijon\\
\unit[2 1/2]{C} & cheddar, sharp
}

\preparation {
\step Preheat oven to $400\; F^\circ$, remove cream cheese from fridge to soften, shred cheese.
\step Cook macaroni al dente, drain, set aside.
\step Meanwhile, melt butter over medium heat then stir in flour and cook until bubbling. Stir in milk, cream cheese, salt, pepper, and mustard until thick. Stir in macaroni and cheese.
\step Add to $\approx9x13"$ or $\approx10x15"$ pan, bake at $400 \textfahrenheiht$ for $[20...25]\; minutes$
}

\hint{
\begin{itemize}
\item Other cheeses are available but sharp cheddar is important to this dish.
\end{itemize}
}

\end{recipe}


\chapter{Entr\`{e}es}
\clearpage
\section[Beef Bowl]{Mexican Beef Bowl}\label{mexican_beef_bowl}\index{chili!Beef Bowl}\index{beef!Beef Bowl}


\begin{recipestats}[
	servings=4 people,
	preptime=30 \minute,
	bakingtime=30 \minute,
	source=Mike \& Jane,
	original=\citeauthor{blueApronBeefBowl}~\cite{blueApronBeefBowl}
	]
\end{recipestats}


\begin{recipeabstract}
	Full flavored and complex.
	This recipe introduced us to the chili baked carrots which are excellent on their own.
	The tomatoes, sauce, and cotija balance out the spicy beef.
\end{recipeabstract}


\ragmarpar{Add a chile de \'{a}rbol to taste (or cayenne).}
\begin{ingredientcolumns}
	\begin{ingredientblock}
		\ingredient[1][lb]{ground beef}\\
		\ingredient[\onehalf][lb]{tomatoes, cherry}\\
		\ingredient[\onehalf][lb]{carrots}\\
		\ingredient[1]{lime}\\
		\ingredient[\onefourth][C]{jalepe\~{n}os, pickled}\\
		\ingredient[\onefourth][C]{mayonnaise}\\
		\ingredient[\approx~8][oz.]{cheese, cotija}\\
		\ingredient[\approx~1][C]{rice}\\
		\ingredient[\approx~3][\Tablespoon]{spice blend}
	\end{ingredientblock}
	\begin{ingredientblock}[spice blend]
		\ingredient[2][\Tablespoon]{guajillo}\\
		\ingredient[1][\Tablespoon]{ancho}\\
		\ingredient[1][\Tablespoon]{paprika, smoked}\\
		\ingredient[1][\Tablespoon]{cumin}\\
		\ingredient[1][\Tablespoon]{marjoram}\\
		\ingredient[\onehalf][\Tablespoon]{garlic powder}\\
		\ingredient[\onehalf][\Tablespoon]{salt}
	\end{ingredientblock}
\end{ingredientcolumns}


\ragmarpar{For dried chilis: remove seeds, toast $\approx3$ \minute~\@ $350$ \Fahrenheit, grind.}

\begin{preparation}
\item Prepare ground spice blend; ahead of time if preferred.

\item Begin to brown the beef in large saut\'{e} pan, stirring occasionally. Reach a strong color at end of recipe.

\item Dice carrots, toss with oil and spices. Preheat oven to $450$ \Fahrenheit.

\item Begin cooking rice.

\item Roast carrots in oven on a cookie sheet, $[12..14)$ \minute.

\item Dice tomatoes, finely dice jalepe\~{n}os, combine with juice of 1/2 of lime, pickled jalepe\~{n}o juice to taste.

\item Add spices to beef ($\approx~3/4\ \Tablespoon$) at end, cook $\approx1~\minute$.\\
	Add $\onefourth\;C$ water, cook $[2..3)$ \minute.

\item Combine mayonnaise and juice of 1/2 lime.\\Add pickled jalepe\~{n}o juice to taste.

\item Serve by layering rice, beef, vegetables, pickled jalepe\~{n}os, cheese, mayonnaise.
\end{preparation}

\ragmarpar{We particularly appreciate the cauliflower rice version.}
\begin{variation}
\item Substitute riced cauliflower in place of rice. Cut into small pieces, chop in food processor with garlic, optionally pan fry.
\item Substitute Chorizo in place of beef, reduce spices.
\item Cotija cheese is worth the effort to find it, but mozzarella works.
\item For convenience, regular pre-ground chili powder can replace the guajillo. Similarly ancho powder is readily available.
\end{variation}

\recipeend%

\section{Fettuccine Alfredo}


\begin{recipestats}[
	servings=2 people,
	preptime=10 \minute,
	bakingtime=20 \minute,
	source=\citetitle{newCookBook2014} \cite{newCookBook2014},
	]
\end{recipestats}


\begin{ingredientcolumns}
	\begin{ingredientblock}
		\ingredient[8][\ounce]{fettuccine, dry}\\
		\ingredient[\approx3][cloves]{garlic}\\
		\ingredient[2][\Tablespoon]{butter, unsalted}\\
		\ingredient[1][C]{cream, heavy}
	\end{ingredientblock}
	\begin{ingredientblock}
		\ingredient[\onehalf][\teaspoon]{salt}\\
		\ingredient[\oneeighth][\teaspoon]{pepper}\\
		\ingredient[2][\ounce]{Parmesan}
	\end{ingredientblock}
\end{ingredientcolumns}


\begin{preparation}
\item Prepare pasta according to the packaged directions.\\
	Meanwhile, grate cheese and crush garlic.

\item Saut\`{e} crushed garlic in butter in a large saucepan $\approx1$ \minute~on medium-high.

\item Add cream, salt, pepper to sauce. Bring to boil then reduce heat, simmer uncovered $\approx3$ \minute or until it begins to thicken.

\item Remove from heat add cheese.

\item Drain pasta, add to sauce, toss to combine.
\end{preparation}


\begin{variation}
\item For shrimp alfredo add $\approx8\; \ounce$ prior to removing from heat, cook through and continue.
\item Try adding mushrooms by browning in oil prior to this recipe, remove from pan, add to sauce with the cream.
\item Try other combinations of cheese such as Parmesan and Asiago. Be sure to freshly grate it.
\end{variation}


\recipeend

\begin{recipe}[
    preparationtime = 1 hour,
    bakingtime = 2 hour 15 minutes,
    portion = $1\;person \; / \; 1 \; lb$ rib roast,
    source = Ralph Nelson (Fa)
]
{Christmas Rib Roast}
\graph{
    %small = strawberry,
    smallpicturewidth = 0.3\textwidth,
    %big = strawberrycake,
    bigpicturewidth = 0.6\textwidth,
}
%%%%%%%%%%%%%%%%%%%%%%%%%%%%%%%%%%%%%%%%%%%%%%%%%%%%%%
\ingredients{
\unit[1]{lb} & standing rib roast / person\\
\unit[to taste]{} & black pepper\\
\unit[to taste]{} & salt
}


%%%%%%%%%%%%%%%%%%%%%%%%%%%%%%%%%%%%%%%%%%%%%%%%%%%%%%
\preparation{

\step Let stand at room temperature for one hour.

\step Rub black pepper \& salt on roast. Preheat oven to $400\; F^\circ$.

\step Place roast on pan fat side up. Do not cover or add water.

\step Bake $15\; minutes$ at $400\; F^\circ$.

\step Lower to $375\; F^\circ$, bake for $45\; minutes$.

\step Turn off heat, do not open oven. Bake for about $30\;minutes$.

\step $35...45\; minutes$ before eating, turn on oven to $375\;F^\circ$

}
%%%%%%%%%%%%%%%%%%%%%%%%%%%%%%%%%%%%%%%%%%%%%%%%%%%%%%

\hint{
\begin{itemize}
\item Fa's rule is that you MUST NOT open the oven under and circumstances.
\end{itemize}
}
%%%%%%%%%%%%%%%%%%%%%%%%%%%%%%%%%%%%%%%%%%%%%%%%%%%%%%
\setRecipeLengths{
preparationwidth = 0.60\textwidth,
ingredientswidth = 0.35\textwidth,
pictureheight = 6cm,
bigpicturewidth = 0.6\textwidth,
smallpicturewidth = 0.35\textwidth
}
%%%%%%%%%%%%%%%%%%%%%%%%%%%%%%%%%%%%%%%%%%%%%%%%%%%%%%
\setRecipeSizes{
recipename = \fontsize{25pt}{30pt},
ing = \normalsize,
inghead = \normalsize,
prep = \normalsize,
prephead = \normalsize,
hint = \normalsize,
hinthead = \Large
}
%%%%%%%%%%%%%%%%%%%%%%%%%%%%%%%%%%%%%%%%%%%%%%%%%%%%%%
%\setRecipenameFont{
%%pbsi
%%fau
%%fwb
%%fjd % default when using the option handwritten
%cmr % default
%}{T1}{m}{n}
%%%%%%%%%%%%%%%%%%%%%%%%%%%%%%%%%%%%%%%%%%%%%%%%%%%%%%
\setHeadlines{
inghead = Ingredients,
prephead = Preparation,
hinthead = Hint,
calory = Cal,
continuationhead = Continuation,
continuationfoot = Continuation on next page
}
%%%%%%%%%%%%%%%%%%%%%%%%%%%%%%%%%%%%%%%%%%%%%%%%%%%%%%
%\setBackgroundPicture
%[%
%x = 2cm,
%y = -1cm,
%width=\paperwidth-3cm,
%height,
%orientation=pagecenter
%]{pic/bg_transparent} % filepath
\end{recipe}

%\clearpage
%\thispagestyle{empty}
% \begin{tikzpicture}[remember picture,overlay]
   % inelegant way of getting a good image:
   % use [keepaspectratio], trim to an aspect ratio close to the page, then remove [keepaspectraio] to get full page
%   \node at (current page.center) {\includegraphics[width=\pdfpagewidth,height=\pdfpageheight,clip,trim={47px 10px 47px 10px}]{tricolor}};
   % IMAGE: http://www.homemadeitaliancooking.com/italian-rainbow-cookies/
%\end{tikzpicture}
\section{Chili Colorado}

\label{chili_colorado_goat}

\begin{recipestats}[
	servings=4 people,
	preptime=30 minutes,
	bakingtime=2 hours,
	source=\citeauthor{rMartinez2015} \cite{rMartinez2015},
	]
\end{recipestats}

\ragmarpar{
	Your chilies are fresh if they are pliable.
	Do not use if they are brittle.
}
\begin{recipeabstract}
	``Chili Colorado. It's a traditional Mexican dish of beef or pork stewed in a red chili sauce, chili 'colored red,' not chili from the state of Colorado'' \cite{rMartinez2015}.
	The dried chilies lend a sweetness to the dish not possible with chili powder.
\end{recipeabstract}

\begin{ingredientcolumns}
	\begin{ingredientblock}
		\ingredient[5]{ancho}\\
		\ingredient[2]{pasilla}\\
		\ingredient[2]{guajillo}\\
		\ingredient[8][C]{stock, chicken}\\
		\ingredient[2][lb]{pork shoulder}
	\end{ingredientblock}
	\begin{ingredientblock}
		\ingredient[\approx8][]{garlic cloves}\\
		\ingredient[2][]{bay leaves}\\
		\ingredient[1][\Tablespoon]{cumin, ground}\\
		\ingredient[1][\teaspoon]{sage, fresh}\\
		\ingredient[2][\teaspoon]{oregano, Mexican}
	\end{ingredientblock}
\end{ingredientcolumns}

\begin{preparation}
\item Measure the spices, chop the sage, and crush the garlic.
\item Cut pork into $\approx1\;inch$ cubes, toss with salt, pepper.
\ragmarpar{Toss pork with a bit of flour as well to thicken the chili further.}
\item Brown the pork. Heat a neutral oil almost to smoking point in a $ \geqq 3.5\;quart$ pot. Reduce to medium high. Brown in batches so as to not overcrowd, de-glazing if necessary to prevent burning.
\item Add spices, stir for about a minute.
\item Add $5\;Cups$ stock, simmer uncovered for $1\;hour$.
\item Meanwhile, re-hydrate the chilies. Remove stems, seeds, veins from chilies and roughly chop. Add to large bowl, add $3\;Cups$ boiling stock, cover with plastic wrap. Wait $30\;minutes$, then blend it all.
\item Add blended chilies to the soup at the end of the first simmer. Simmer for $45\;minutes$ uncovered.
\item Season with salt pepper to taste.
\end{preparation}

\begin{variation}
	\item Marjoram can be a substitute for the Mexican oregano if necessary, but not Mediterranean oregano.
	\item Serve with tortillas, and the carrots from \nameref{mexican_beef_bowl} \ref{mexican_beef_bowl}.  \citeauthor{rMartinez2015} recommends rice, beans a la charra, and tortillas.
\end{variation}

\recipeend
\section{Chili Schwarz}
\begin{recipestats}[
	servings=4 people,
	preptime=1 hour,
	bakingtime=1 hour,
	]
\end{recipestats}

\begin{recipeabstract}
	This was an experiment on \nameref{chili_colorado_goat} \ref{chili_colorado_goat} to use more black chilies and reduce cook time.
\end{recipeabstract}

\begin{ingredientcolumns}
	\begin{ingredientblock}
		\ingredient[4]{ancho}\\
		\ingredient[4]{pasilla}\\
		\ingredient[1]{guajillo}\\
		\ingredient[8][C]{stock, beef}\\
		\ingredient[\approx \nicefrac{3}{2}][lb]{ground beef}\\
		\ingredient[1]{onion, red}\\
		\ingredient[2][]{pablano}
	\end{ingredientblock}
	\begin{ingredientblock}
		\ingredient[3][]{carrots}\\
		\ingredient[3][]{garlic cloves}\\
		\ingredient[3][]{black garlic cloves}\\
		\ingredient[2][]{bay leaves}\\
		\ingredient[1][\Tablespoon]{cumin, ground}\\
		\ingredient[2][\teaspoon]{oregano, Mexican}
	\end{ingredientblock}
\end{ingredientcolumns}

\ragmarpar{Leftovers may need some chili d\`{e} arbol to bring back the heat}

\begin{preparation}
\item Preheat oven to $400\;F^\circ$. Remove seeds / veins / stems of chilies. Squash garlic and cut off the ends. Toast chilies for $\approx 1...2$ on a sheet (make certain not to burn them).
\item Add chilies \& garlic to a bowl, cover w/ boiling water. Keep submerged, cover w/ plastic wrap, steep for $20...30\;min$.
\item Meanwhile begin browning the beef. Achieve a strong color and deglaze often. Recommend 1/2 in a dutch oven, 1/2 in a pan, drain and reserve fat from pan. Deglaze with a toasty beer for a bit of flair.
\item Julienne the onion, lightly saut\`{e} in pan w/ pepper. Meanwhile slice the Pablano, roughly chop the carrots, set aside.
\item When the chilies are done remove from the water into a food processor. Add black garlic. Blend well with a bit of the stock for $\approx5\;min$.
\item Add cumin to beef, stir $\approx 1\;min$, add chili sauce.
\item Add onion, broth / stock, bay leaf, oregano. Bring to simmer. Simmer for $\approx30...60\;min$.
\item Meanwhile, saut\`{e} the carrot, add to chili, repeat with Pablano. Season with salt.
\end{preparation}

\begin{variation}
	\item There is not a clear answer on whether or not to use the liquid used to re-hydrate the chilies. Prefer to toss it if it tastes bitter.
\end{variation}

\begin{experiments}
	\item Still working out which chilies to use. Maybe try some mulato or morita?
\end{experiments}

\recipeend

\section{Hoola Poola}
\begin{recipestats}[
	servings=1 person,
	preptime=10 minutes,
	bakingtime=20 minutes,
	]
\end{recipestats}

\begin{recipeabstract}
	This was created for a cost-effective lunch.
	The ingredients keep a long time and can mostly be prepped before hand.
	The name is a play on the Giesken's spam \& eggs recipe Hunka Punka. % FUTURE add ref to Hunka Punka
\end{recipeabstract}

\begin{ingredientcolumns}
	\begin{ingredientblock}
		\ingredient[4][\ounce]{Spam}\\
		\ingredient[2][C]{collards}\\
		\ingredient[1][C]{chickpeas, cooked}\\
		\ingredient[1][C]{rice, cooked}
	\end{ingredientblock}
	\begin{ingredientblock}
		\ingredient[\approx \onehalf][\teaspoon]{Sa\'zon}\\
		\ingredient[1][\Tablespoon]{butter}\\
		\ingredient[\onehalf][\Tablespoon]{garlic, minced}
	\end{ingredientblock}
\end{ingredientcolumns}


\ragmarpar{Cooking the collards until crispy (instead of stewed) is sometimes referred to as ``Brazillian'' collards}
\begin{preparation}
\item Dice and add Spam to frying pan over medium heat. Optionally add mushrooms, or other customization. Stir occasionally.

\item Meanwhile prepare the collards. Discard woody stems, slice leaves ($\approx1x1"$), let soak in water w/ a dash of vinegar and salt. Set aside remaining ingredients.

\item Once Spam is toasted, drain majority of water from collards, add to pan, add Sa\'zon. Stir often and cook until collards begin to get crispy.

\item Once collards are toasted, reduce heat a bit, expose center of pan by pushing mixture aside and add butter and garlic. Cook to a light toast ($\approx[30\dots 60]\; sec$).

\item Serve with rice. Top with a dash of Sa\'zon and yell ``HOOLA POOLA!'' for a bit of pizzazz.
\end{preparation}

\recipeend


\chapter{Bread}
\clearpage
\section[Focaccia]{Handshake Focaccia~\vegan}


\index{bread!Focaccia}


\begin{recipestats}[
	servings=6 people,
	preptime=3 \hour,
	bakingtime=15 \minute,
	source=\citefield{howToBake2013}{title} \cite{howToBake2013},
	]
\end{recipestats}


\begin{recipeabstract}
	A Focaccia worthy of the Paul Hollywood Handshake.
	This is best eaten out of the oven and dipped in olive oil with a little salt and or-A-gaano (as they say in Britain).
\end{recipeabstract}


\begin{ingredientcolumns}
	\begin{ingredientblock}
		\ingredient[500][\gram]{bread flour}\\
		\ingredient[10][\gram]{salt}\\
		\ingredient[10][\gram]{yeast, instant}
	\end{ingredientblock}
	\begin{ingredientblock}
		\ingredient[140][\milliliter]{olive oil}\\
		\ingredient[360][\milliliter]{water}
	\end{ingredientblock}
\end{ingredientcolumns}


\begin{preparation}
\item Mix the dough.
	Add flour to mixing bowl, add salt \& yeast on opposite sides.
	Add 40 \milliliter~oil \& $\approx\threefourth$ water, hand mix.
	Continue mixing and gradually add water until all the flour is incorporated; water may be left over.
	Aim for a soft / wet dough.

\item Knead the dough.
	Add some oil to the working surface, then the dough.
	Knead $5 \dots 10$ \minute.
	Knead past the wet stage until the exterior is smooth \& soft.
	Refrain from adding more flour.

\item Rise the dough.
	Move it to a lightly oiled $\approx 2 \dots 3$ \quart~tub.
	Add tea towel on top, rise $\approx 1 \hour$ until at least doubled in size.

\item Separate the dough.
	Line baking parchment to two trays, drizzle olive oil on top.
	Add olive oil to the working surface, optionally dust w/ fine semolina.
	Move dough to working surface slowly as to keep air in the dough.
	Divide dough in half and stretch out flat onto the trays.

\item Prove the dough.
	Add each tray into a plastic bag and prove for $\approx 1$ \hour, until it has doubled in size.
	The dough should spring back quickly.
	Preheat oven to 430 \Fahrenheit.

\item Bake.
	Add dimples on top of the dough using your fingers; push all the way to the bottom.
	Drizzle each with olive oil, top with flaked salt and oregano.
	Bake $\approx 15$ \minute.
	The bread should be cooked through so that tapping the bottom will sound hollow.
	Drizzle with olive oil, cool.
\end{preparation}


\recipeend


\chapter{Hot Sauce}
\clearpage
{
\section[Brined Hot Sauce]{Base Brined Hot Sauce~\vegan}\label{base_lacto_brine_hotsauce}


\begin{recipestats}[
	servings=12 \fluidounce,
	preptime=1 \hour,
	bakingtime=2 \week,
	source=Teresi Family,
	original=\citefield{fieryferments2017}{title}~\cite{fieryferments2017},
	]
\end{recipestats}


\begin{recipeabstract}
	A base hot sauce recipe with whole brined peppers.
	Select your peppers, spices, and custom additions.
	Peppers are fermented whole and then pur\`{e}ed rather than mashed prior to fermentation, see ``Mixed-Media Basic Mash''~\cite{fieryferments2017}.
\end{recipeabstract}


\ragmarpar{Use un-iodized salt to prevent discoloration.}
\begin{ingredientcolumns}[1]
	\begin{ingredientblock}
		\ingredient[1][]{jar, 32~\fluidounce}\\
		\ingredient[1]{pickle weight}\\
		\ingredient[1]{air lock}\\
		\ingredient[][]{Star San}
	\end{ingredientblock}
	\begin{ingredientblock}
		\ingredient[\approx~600][g]{water, un-chlorinated}\\
		\ingredient[\approx~30][g]{salt, un-iodized}\\
		\ingredient[\approx~10][g]{cabbage}\\
		\ingredient[\approx~8][\ounce]{peppers}
	\end{ingredientblock}
\end{ingredientcolumns}


\begin{preparation}
\item Sanitize equipment.
	Move jar, weight, lid to dish rack to dry.

\item Mix a $[5 \dots 5.3]\%$ brine by weight, $\approx\threefourth$ volume of jar.

\item Wash produce.
	Halve peppers, remove stems and seeds.

\ragmarpar{Floating chunks cause spoilage.}
\item Add fermentables to jar.
	Prevent items from floating; start with smaller pieces, pack tightly, end with cabbage leaf tucked into the sides.
	Top with fermentation weight.

\item Add brine mixture, leaving $\approx1cm$ of head space.
	Ensure all ingredients are submerged.

\item Add lid and air lock.
	Ensure air lock is not submerged.

\item Ferment for $[1..2]$ \week~or up to many months, then refrigerate until ready to pur\`{e}e.

\item Separate brine and ingredients, pur\`{e}e with $\approx\onehalf$ $C$ brine.
	Strain if desired; recommended if dried peppers are used.
	Optionally blend Achiote for color.

\item Add vinegar or brine to desired consistency, refrigerate.
\end{preparation}


\begin{experiments}
\item Prevent spoilage by keeping ingredients submerged and by not removing the lid (keep it anaerobic).
	See \citefield{fieryferments2017}{title}~\cite{fieryferments2017},
for troubleshooting.
	In general: you should throw out the batch if you see any fuzzy mold.
\item The heat level of the sauce lowers drastically over fermentation.
\end{experiments}


\recipeend%

	\section[Red Hot Sauce]{Fermented Red Hot Sauce}
	\let\section\subsection
	\let\subsection\subsubsection
	
	% use \input (not \include) to remove page breaks
	\begin{recipe}[
    preparationtime = 60 minutes,
    bakingtime = 2 weeks,
    portion = 12 fluid ounces,
]
{Red No. 3}
\graph{
    %small = strawberry,
    smallpicturewidth = 0.3\textwidth,
    %big = strawberrycake,
    bigpicturewidth = 0.6\textwidth,
}

%%%%%%%%%%%%%%%%%%%%%%%%%%%%%%%%%%%%%%%%%%%%%%%%%%%%%%
\ingredients{
\unit[8]{} & Fresno \\
\unit[6]{} & Cherry\\
\unit[6]{} & Thai\\
\unit[3]{} & Guajillo\\
\unit[3]{} & Arbol\\
\unit[1]{} & Shallot\\
\unit[1]{tsp} & Indian Green pepper\\
\\
\unit[1/4]{tsp} & Achiote\\
\unit[1]{Tbs} & Oil, vegetable
}

%%%%%%%%%%%%%%%%%%%%%%%%%%%%%%%%%%%%%%%%%%%%%%%%%%%%%%
\preparation{

\step Follow the basic brine \ref{base_lacto_brine_hotsauce} with a $5.3\%$ brine.

\step Ferment $2$ weeks, refridgerate $2$ weeks, blend with $1/2\;Cup$ brine and white vinegar each, strain.
}
%%%%%%%%%%%%%%%%%%%%%%%%%%%%%%%%%%%%%%%%%%%%%%%%%%%%%%

%%%%%%%%%%%%%%%%%%%%%%%%%%%%%%%%%%%%%%%%%%%%%%%%%%%%%%
\setRecipeLengths{
preparationwidth = 0.60\textwidth,
ingredientswidth = 0.35\textwidth,
pictureheight = 6cm,
bigpicturewidth = 0.6\textwidth,
smallpicturewidth = 0.35\textwidth
}
%%%%%%%%%%%%%%%%%%%%%%%%%%%%%%%%%%%%%%%%%%%%%%%%%%%%%%
\setRecipeSizes{
recipename = \fontsize{25pt}{30pt},
ing = \normalsize,
inghead = \normalsize,
prep = \normalsize,
prephead = \normalsize,
hint = \normalsize,
hinthead = \Large
}
%%%%%%%%%%%%%%%%%%%%%%%%%%%%%%%%%%%%%%%%%%%%%%%%%%%%%%
%\setRecipenameFont{
%%pbsi
%%fau
%%fwb
%%fjd % default when using the option handwritten
%cmr % default
%}{T1}{m}{n}
%%%%%%%%%%%%%%%%%%%%%%%%%%%%%%%%%%%%%%%%%%%%%%%%%%%%%%
\setHeadlines{
inghead = Ingredients,
prephead = Preparation,
hinthead = Hint,
calory = Cal,
continuationhead = Continuation,
continuationfoot = Continuation on next page
}
%%%%%%%%%%%%%%%%%%%%%%%%%%%%%%%%%%%%%%%%%%%%%%%%%%%%%%
%\setBackgroundPicture
%[%
%x = 2cm,
%y = -1cm,
%width=\paperwidth-3cm,
%height,
%orientation=pagecenter
%]{pic/bg_transparent} % filepath
\end{recipe}

%\clearpage
%\thispagestyle{empty}
% \begin{tikzpicture}[remember picture,overlay]
   % inelegant way of getting a good image:
   % use [keepaspectratio], trim to an aspect ratio close to the page, then remove [keepaspectraio] to get full page
%   \node at (current page.center) {\includegraphics[width=\pdfpagewidth,height=\pdfpageheight,clip,trim={47px 10px 47px 10px}]{tricolor}};
   % IMAGE: http://www.homemadeitaliancooking.com/italian-rainbow-cookies/
%\end{tikzpicture}
}

\chapter{Ice Cream}
\clearpage
\begin{recipe}[
    portion = $5$ Cups,
    source = \citefield{joyofcooking2006}{title}\cite{joyofcooking2006},
]
{Chocolate Ice Cream}
\graph{
    %small = strawberry,
    smallpicturewidth = 0.3\textwidth,
    %big = strawberrycake,
    bigpicturewidth = 0.6\textwidth,
}

%%%%%%%%%%%%%%%%%%%%%%%%%%%%%%%%%%%%%%%%%%%%%%%%%%%%%%
\ingredients{
\unit[2]{C} & milk, whole \\
\unit[1/2]{C} & sugar\\
\\
\unit[4]{} & egg yolks, large\\
\unit[1/4]{C} & sugar\\
\unit[1/3]{C} & cocoa powder, dutch\\
\\
\unit[1]{C} & cream, heavy\\
\unit[1]{tsp} & vanilla
}

%%%%%%%%%%%%%%%%%%%%%%%%%%%%%%%%%%%%%%%%%%%%%%%%%%%%%%
\preparation{

\step Combine in saucepan over medium low heat the milk and sugar, bring to simmer stirring occasionally.

\step Whisk egg yolks and the second sugar volume in a medium bowl, whisk in cocoa.

\step Pour slowly while stirring constantly about half of the hot milk into the eggs. Pour back into the saucepan.

\step Cook stirring constantly over low heat until it reaches $175F^\circ$, and do not allow it to boil. Remove from heat.

\step Strain through a fine sieve into a bowl, then add cream \& vanilla. Refrigerate until cold. Proceed with ice cream machine directions.
}
%%%%%%%%%%%%%%%%%%%%%%%%%%%%%%%%%%%%%%%%%%%%%%%%%%%%%%
\hint{
\begin{itemize}
	\item Try marshmallow oreo: 1 cup broken oreos in mixer at end, fold in marshmallow fluff after mixing.
\end{itemize}
}
%%%%%%%%%%%%%%%%%%%%%%%%%%%%%%%%%%%%%%%%%%%%%%%%%%%%%%
\setRecipeLengths{
preparationwidth = 0.60\textwidth,
ingredientswidth = 0.35\textwidth,
pictureheight = 6cm,
bigpicturewidth = 0.6\textwidth,
smallpicturewidth = 0.35\textwidth
}
%%%%%%%%%%%%%%%%%%%%%%%%%%%%%%%%%%%%%%%%%%%%%%%%%%%%%%
\setRecipeSizes{
recipename = \fontsize{25pt}{30pt},
ing = \normalsize,
inghead = \normalsize,
prep = \normalsize,
prephead = \normalsize,
hint = \normalsize,
hinthead = \Large
}
%%%%%%%%%%%%%%%%%%%%%%%%%%%%%%%%%%%%%%%%%%%%%%%%%%%%%%
%\setRecipenameFont{
%%pbsi
%%fau
%%fwb
%%fjd % default when using the option handwritten
%cmr % default
%}{T1}{m}{n}
%%%%%%%%%%%%%%%%%%%%%%%%%%%%%%%%%%%%%%%%%%%%%%%%%%%%%%
\setHeadlines{
inghead = Ingredients,
prephead = Preparation,
hinthead = Hint,
calory = Cal,
continuationhead = Continuation,
continuationfoot = Continuation on next page
}
%%%%%%%%%%%%%%%%%%%%%%%%%%%%%%%%%%%%%%%%%%%%%%%%%%%%%%
%\setBackgroundPicture
%[%
%x = 2cm,
%y = -1cm,
%width=\paperwidth-3cm,
%height,
%orientation=pagecenter
%]{pic/bg_transparent} % filepath
\end{recipe}




\chapter{Cookies, Cakes}
\clearpage

\begin{recipe}
        [
preparationtime = 1 hour (plus cooling/ chilling),
bakingtime = 10 minutes,
bakingtemperature = {\unit[350]{$F^\circ$}},
portion = $\sim$ 36 cookies,
source = Good Housekeeping 2013 (magazine pg. 69-70)
]{Italian Tricolors}
        \graph{
%small = strawberry,
smallpicturewidth = 0.3\textwidth,
%big = strawberrycake,
bigpicturewidth = 0.6\textwidth,
}
%%%%%%%%%%%%%%%%%%%%%%%%%%%%%%%%%%%%%%%%%%%%%%%%%%%%%%
\ingredients{
\unit[7 to 8]{oz} & almond paste \\
\unit[3/4]{C} & butter or margarine, softened\\
\unit[3/4]{C} & sugar\\
\unit[1/2]{tsp} & almond extract\\
\unit[3]{L} & eggs\\
\unit[1]{C} & all purpose flour\\
\unit[1/4]{tsp} & salt\\
\unit[15]{drops} & red food coloring\\
\unit[15]{drops} & green food coloring\\
\unit[2/3]{C} & apricot preserves\\
\unit[3]{oz} & semisweet chocolate\\
\unit[1]{tsp} & vegetable shortening\\
}
%%%%%%%%%%%%%%%%%%%%%%%%%%%%%%%%%%%%%%%%%%%%%%%%%%%%%%
\preparation{
\step Preheat oven ($350F^\circ$), grease three $8"\;x\;8"$ (square) pans.
Line bottoms w/ waxed paper, grease and flour paper.

\step In large bowl w/ mixer at medium-high speed, blend: almond paste, butter, sugar, almond extract.
There will be small lumps remaining.
Reduce to medium and add eggs one-at-a-time.
Reduce to low and beat in flour \& salt until just combined.

\step Divide batter into thirds into separate bowls.
Blend green dye into one, red into another, leaving one un-tinted.
\step For each mixture, transfer into prepared metal pan and spread evenly (e.g. with an offset spatula).

\step Bake on two oven racks $10-12\; minutes$ rotating pans between upper/lower racks halfway through.
Ensure layers are set such that a toothpick inserted in the center comes out clean.

\step Cool in pans on wire racks $5\; minutes$.
Run knife around sides to loosen layers. Invert onto racks, leaving wax paper attached, and cool completely.

\step When layers are cool, use a food processor or sieve to remove large chunks from the fruit preserves.
Remove the waxed paper from the layers.
Assemble by inverting the layers and spreading the preserves in the following order: green, 1/2 of the preserves, white, 1/2 of the preserves, red.
Heat the chocolate and shortening on low in a 1-quart saucepan, stirring frequently until melted.
Spread on top of the red layer, but not the sides, and refrigerate at least 1 hour.

\step To serve, let rest at room temperature for at least 5 minutes then trim the edges and cut into squares (about 36 pieces).
Store cookies in a single layer in a tightly covered container in the refrigerator up to 1 week or in the freezer up to 3 months.

}
%%%%%%%%%%%%%%%%%%%%%%%%%%%%%%%%%%%%%%%%%%%%%%%%%%%%%%
\hint{
\begin{itemize}
\item Use fresh almond paste, try dark chocolate, try adding liqueur to the fruit, try adding zest in the cake layer.
\item Reduce cracking in chocolate by increasing shortening or adding corn syrup or raising temperature.
\item Try other combinations of colors and preserves for different events, such as red/white/blue with cherries for $4^{th}$ of July, and orange/white/black for Halloween.
\item A double batch may prevent breaking and make it easier to spread in the pan.
\end{itemize}
}
%%%%%%%%%%%%%%%%%%%%%%%%%%%%%%%%%%%%%%%%%%%%%%%%%%%%%%
\setRecipeLengths{
preparationwidth = 0.60\textwidth,
ingredientswidth = 0.35\textwidth,
pictureheight = 6cm,
bigpicturewidth = 0.6\textwidth,
smallpicturewidth = 0.35\textwidth
}
%%%%%%%%%%%%%%%%%%%%%%%%%%%%%%%%%%%%%%%%%%%%%%%%%%%%%%
\setRecipeSizes{
recipename = \fontsize{25pt}{30pt},
ing = \normalsize,
inghead = \normalsize,
prep = \normalsize,
prephead = \normalsize,
hint = \normalsize,
hinthead = \Large
}
%%%%%%%%%%%%%%%%%%%%%%%%%%%%%%%%%%%%%%%%%%%%%%%%%%%%%%
%\setRecipenameFont{
%%pbsi
%%fau
%%fwb
%%fjd % default when using the option handwritten
%cmr % default
%}{T1}{m}{n}
%%%%%%%%%%%%%%%%%%%%%%%%%%%%%%%%%%%%%%%%%%%%%%%%%%%%%%
\setHeadlines{
inghead = Ingredients,
prephead = Preparation,
hinthead = Hint,
calory = Cal,
continuationhead = Continuation,
continuationfoot = Continuation on next page
}
%%%%%%%%%%%%%%%%%%%%%%%%%%%%%%%%%%%%%%%%%%%%%%%%%%%%%%
%\setBackgroundPicture
%[%
%x = 2cm,
%y = -1cm,
%width=\paperwidth-3cm,
%height,
%orientation=pagecenter
%]{pic/bg_transparent} % filepath
\end{recipe}

%\clearpage
%\thispagestyle{empty}
% \begin{tikzpicture}[remember picture,overlay]
%   % inelegant way of getting a good image:
%   % use [keepaspectratio], trim to an aspect ratio close to the page, then remove [keepaspectraio] to get full page
%   \node at (current page.center) {\includegraphics[width=\pdfpagewidth,height=\pdfpageheight,clip,trim={47px 10px 47px 10px}]{tricolor}};
%   % IMAGE: http://www.homemadeitaliancooking.com/italian-rainbow-cookies/
%\end{tikzpicture}
%\section[Felix Cookies]{Felix Cookies, or,\\ \mbox{Schwarz-Wei\ss-Geb\"{a}ck}}

\begin{recipestats}[
	servings=40 cookies,
	preptime=30 minutes (+90 min chill),
	bakingtime=12 minutes / batch,
	source=\citetitle{luisaWeiss2016} \cite{luisaWeiss2016},
	]
\end{recipestats}

\begin{recipeabstract}
	A shortbread cookie with a black and white checkerboard.
	Works very well with the addition of chocolate.
	Aliased for our black and white cats, Felix and Frankie.
\end{recipeabstract}

\begin{ingredientcolumns}
	\begin{ingredientblock}
		\ingredient[150][g]{butter, unsalted}\\
		\ingredient[75][g]{sugar, powdered}\\
		\ingredient[\oneeighth][\teaspoon]{salt}\\
		\ingredient[\onefourth][\teaspoon]{vanilla extract}
	\end{ingredientblock}
	\begin{ingredientblock}
		\ingredient[200][g]{flour, all purpose}\\
		\ingredient[1]{egg yolk}\\
		\ingredient[2][\Tablespoon]{whole milk}\\
		\ingredient[2\;\onehalf][\Tablespoon]{cocoa powder}
	\end{ingredientblock}
\end{ingredientcolumns}

\begin{preparation}
\ragmarpar{other shapes are available, like 6 petal flowers, spirals, etc.}
\item Cream butter $\approx 1$ $minute$, add sugar, salt, vanilla, then cream.
Add flour and mix until just combined.

\item Divide dough in half, mix cocoa into one half.
Form into disks, wrap in plastic wrap, refrigerate $\approx1$ $hour$.

\item Mix milk \& egg yolk in a small bowl.

\item Make 4 square logs, brush sides w/ egg wash, press together and refrigerate $\approx 30$ $minutes$.

\ragmarpar{don't over bake, try adding another sheet below to shield the radiation}
\item Meanwhile preheat oven \& line baking sheets with parchment paper.

\item Slice off cookies to $1cm$ or other desired thickness, bake $12...15$ $minutes$.
\end{preparation}

\begin{variation}
\item Dip into dark chocolate.
\end{variation}
%\section{Peppermint Fudge}

\begin{recipestats}[
	servings=60 squares,
	preptime=15 minutes,
	bakingtime=30 minutes,
	source=Cookie Swap 2003,
	]
\end{recipestats}

\begin{recipeabstract}
	A Giesken Christmas tradition.
\end{recipeabstract}

\begin{ingredientcolumns}
	\begin{ingredientblock}
		\ingredient[4][C]{sugar}\\
		\ingredient[10][\fluidounce]{evaporated milk}\\
		\ingredient[1][C]{butter}\\
		\ingredient[2][C]{chocolate chips}
	\end{ingredientblock}
	\begin{ingredientblock}
		\ingredient[7][\fluidounce]{marshmallow creme}\\
		\ingredient[\onehalf][\teaspoon]{peppermint extract}\\
		\ingredient[\twothird][C]{peppermint candy}
	\end{ingredientblock}
\end{ingredientcolumns}

\begin{preparation}
\item Line a $13x9$ $inch$ pan with foil and butter the interior.
Crush peppermint candy.

\item Combine sugar, milk, butter, in a $3$ $quart$ saucepan.
Bring to boil over medium-high heat, stirring constantly.

\item Reduce to medium, stir to $10$ $minutes$.

\item Remove from heat, add chocolate chips, marshmallow creme, peppermint extract.
Stir until chocolate and creme are melted and mixture is smooth.

\item Pour into pan, sprinkle peppermint on top, cover, refrigerate until set.
\end{preparation}
\recipeend
%\begin{recipe}[
source = Rand Pearson
]
{7-Layer Brownies}

\ingredients {
\unit[1]{box} & Betty Crocker brownie/cookie combo mix\\
\unit[4]{Tbs} & butter\\
\unit[4\dots 6]{} & Heath bars\\
\unit[]{} & marshmallows\\
\unit[]{} & graham crackers\\
\unit[]{} & chocolate chips\\
\unit[]{} & peanut butter\\
}

\preparation {
\step Mix up the brownie and cookie mix as instructed on the back of the box.

\step Place brownie mix in a baking dish as instructed.

\step Corsely chop Heath bar and spead evenly over brownie batter.

\step Place cookie dough mix on top of Heath bar layer.

\step Crush graham crackers and spread over cookie dough.

\step Melt half a stick of butter and pour over graham crackers.

\step If using chocolate chips spread a layer over graham crackers. If using peanut butter melt it and pour over graham cracker.

\step Split marxhmallows lengthwise and arrange a asolid layer over the top of everything. You might also use mini marshamallows to skip cutting them.

\step Bake as instructed on brownie box. Time may increase due to the extra layers. Brownies are done when a toothpick comes out clean.
}

\hint{
\begin{itemize}
\item Keep in mind that there really isn't a recipe since I just made it all up as I went along. What you should take from that is this: Feel free to experiment; any problems can be overcome with enough butter and sugar. I'm pretty sure this is also true in life.
\item In the future I was considering leaving the marshmallows off until I take the brownies out of the oven, then adding them ad hitting them with a torch.
\end{itemize}
}

\end{recipe}

%\section{Elegant Wine Cake}

\begin{recipestats}[
	servings=2 loafs,
%	preptime=,  % TODO add time
%	bakingtime=,% TODO add time
	source=Lucille Steinmiller (Oma),
	]
\end{recipestats}

\begin{recipeabstract}
	A Teresi Christmas tradition from Oma.
	Truly a refined and flavorful cake, perfect for sharing.
\end{recipeabstract}

\begin{ingredientcolumns}[1]
	\begin{ingredientblock}
		\ingredient[1][pkg]{cake mix, yellow}\\
		\ingredient[1][pkg]{vanilla pudding, instant}\\
		\ingredient[\onehalf][C]{vegetable}\\
		\ingredient[4][large]{eggs}\\
		\ingredient[\threefourth][C]{sherry, medium}\\
		\ingredient[\threefourth][C]{water}\\
		\ingredient[1][C]{chopped nuts}
	\end{ingredientblock}
\end{ingredientcolumns}


\begin{preparation}
\item Combine all ingredients in a bowl, beat for 2 minutes on medium speed.
\item Bake in 2 loaf pans $\approx 9x5"$ (greased and floured) at $350\; F^\circ$ for $45\; minutes$ or until done.
\item Top with powdered sugar.
\end{preparation}

\begin{variation}
\item Chablis can be substituted for the sherry.
\item Mom usually uses pecans, I  omit the nuts.
\item Using four smaller cake or a bundt pans also works well.
\end{variation}
\recipeend

%
%\clearpage
%\section{Pie}
%\section{Basic Flaky Pie Pastry}\label{basicFlakyPiePastry}


\begin{recipestats}[
	servings=1 pastry,
	preptime=45 \minute,
	bakingtime=1 \hour~(chill),
	source=\citefield{pie2004}{title} \cite{pie2004},
	]
\end{recipestats}


\begin{recipeabstract}
	An all purpose pie pastry.
	Very useful to make in large batches, freezing up to a month.
\end{recipeabstract}


\ragmarpar{works very well as a double batch}
\begin{ingredientcolumns}
	\begin{ingredientblock}
		\ingredient[1 \onehalf][C]{flour, all purpose}\\
		\ingredient[1 \onehalf][\teaspoon]{sugar}\\
		\ingredient[\onehalf][\teaspoon]{salt}
	\end{ingredientblock}
	\begin{ingredientblock}
		\ingredient[\onefourth][C]{butter, unsalted}\\
		\ingredient[\onefourth][C]{shortening}\\
		\ingredient[\onefourth][C]{water}
	\end{ingredientblock}
\end{ingredientcolumns}


\begin{preparation}
\item Cut fat into small pieces ($\approx 3/8"$ cubes), place in freezer briefly along with water until cold.

\item Mix flour, sugar, salt, butter, in a large bowl.
	Blend using a pastry cutter $\|$ fork $\|$ fingers, until the butter is pea sized.
	Blend the shortening similarly.

\item Add half the water and toss with fork.
	Add water $\approx 1.5 \dots 2$ \Tablespoon~at a time, and pull all the flour into the dough.
	Continue until the dough can be packed together.

\item Pack dough into a ball, knead once or twice.
	Flatten onto a floured surface into $\approx 3/4"$ disks.
	Wrap in plastic and refrigerate at least 1 \hour~or overnight.

\item Roll pastry onto wax paper, invert onto pie pan \& shape.
	Freeze for 15 \minute.

\item For a pre-baked crust: preheat to 400 \Fahrenheit, press aluminum foil on top of pastry and fill with pie weights.
	Bake 15 \minute, remove foil \& weights, prick holes into pastry base with fork to prevent bubbles.

\item Lower to 375 \Fahrenheit, bake $10 \dots 12$ \minute~for a partially pre-baked crust or $15 \dots 17$ \minute~for a fully prebaked crust.
\end{preparation}


\recipeend

%\section[Strawberry Rhubarb Pie]{Strawberry Rhubarb Crumb Pie}


\begin{recipestats}[
	servings=1 pie,
	preptime=1 \onehalf~\hour,
	bakingtime=50 \minute,
	source=\citefield{pie2004}{title} \cite{pie2004},
	]
\end{recipestats}


\begin{recipeabstract}
	I don't always have favorites but when I do it's pretty close to this pie.
	Make sure to capitalize on the spring season when rhubarb is available.
	There is a lot of liquid so I increased the tapioca and maceration time.
	Make sure to use a deep pan and high crust.
\end{recipeabstract}


\ragmarpar{don't eat the rhubarb leaves}

\begin{ingredientcolumns}
	\begin{ingredientblock}[filling]
		\ingredient[3][C]{rhubarb}\\
		\ingredient[\threefourth][C]{sugar}\\
		\ingredient[1 \onehalf][\Tablespoon]{lemon juice}\\
		\ingredient[1][lemon]{zest}\\
		\ingredient[4][C]{strawberries}\\
		\ingredient[\onehalf][C]{tapioca, quick}
	\end{ingredientblock}
	\begin{ingredientblock}[topping]
		\ingredient[\threefourth][C]{flour, all-purpose}\\
		\ingredient[\onefourth][C]{cornmeal, yellow}\\
		\ingredient[\twothird][C]{sugar, brown}\\
		\ingredient[\onehalf][\teaspoon]{cinnamon}\\
		\ingredient[\onefourth][\teaspoon]{salt}\\
		\ingredient[\onehalf][C]{butter, unsalted}
	\end{ingredientblock}
\end{ingredientcolumns}


\begin{preparation}
\item Prepare \ref{basicFlakyPiePastry} \nameref{basicFlakyPiePastry} and refrigerate $\geq 1$ \hour.

\ragmarpar{make a high crust to prevent spills}
\item Roll pastry onto wax paper $\approx 13"$ diameter, invert onto pie pan and shape.
	Freeze for 15 \minute.
	Preheat oven to $400$ \Fahrenheit.

\item Prepare filling.
	Slice rhubarb $\approx 1/2"$ pieces, mix fruit w/ sugar, lemon juice, zest, tapioca.
	Quarter strawberries and mix in.
	Macerate for $\geq 15$ \minute.

\item Add filling evenly into crust, bake on center rack 30 \minute.

\item Meanwhile prepare topping.
	Combine flour, cornmeal, brown sugar, cinnamon, salt.
	Cut butter into pieces and blend in with food processor or pastry cutter.
	Make large crumbs by rubbing the mixture between your hands.
	Refridgerate.

\ragmarpar{place a cookie sheet underneath to catch spills}
\item Remove pie and reduce oven to 375 \Fahrenheit.
	Add crumbs to top of pie.
	Rotate pie $180^\circ$ (to bake evenly) and bake $[30...40]$ \minute.
	Add foil heat shield if needed for last $\approx 10$ \minute.
	Cool $\geq 1$ \hour.
\end{preparation}


\begin{variation}
\item Try a bit of ground green cardamom in the filling.
\end{variation}


\recipeend

%\section[Lemon Meringue Pie]{Classic Lemon Meringue Pie}
\index{lemon!Lemon~Meringue}


\begin{recipestats}[
	servings=1 pie,
	preptime=1 \hour,
	bakingtime=20 \minute,
	source=\citefield{pie2004}{title} \cite{pie2004},
	]
\end{recipestats}


\begin{recipeabstract}
	A Teresi Thanksgiving \& Christmas tradition.
	This is perfect for large gatherings.
	It can be made the day prior, save for the meringue which is done before serving.
	It is light and tart which is excellent after a large meal.
\end{recipeabstract}

\pagebreak[4]
\begin{ingredientcolumns}
	\begin{ingredientblock}[filling]
		\ingredient[1 \onethird][C]{sugar}\\
		\ingredient[\threeeighth][C]{corn starch}\\
		\ingredient[\oneeighth][\teaspoon]{salt}\\
		\ingredient[2][C]{water}\\
		\ingredient[\onehalf][C]{lemon juice}\\
		\ingredient[1][\Tablespoon]{lemon zest}\\
		\ingredient[4][large]{egg yolks}\\
		\ingredient[2][\Tablespoon]{butter, unsalted}
	\end{ingredientblock}
	\begin{ingredientblock}[meringue]
		\ingredient[4][large]{egg whites}\\
		\ingredient[\onefourth][\teaspoon]{cream of tartar}\\
		\ingredient[1][pinch]{salt}\\
		\ingredient[\onehalf][C]{sugar, powdered}\\
		\ingredient[\onehalf][\teaspoon]{vanilla extract}
	\end{ingredientblock}
\end{ingredientcolumns}


\begin{preparation}
\ragmarpar{Fresh lemons are critical, you'll need $\geq 2$. Extra juice / zest is ok.}
\item Prepare pastry \ref{basicFlakyPiePastry} \nameref{basicFlakyPiePastry}, partially pre-bake, and let cool.
	Dice the butter into $\approx\onehalf$ $inch$ pieces.

\item Cook custard.
	Mix sugar, cornstarch, salt, in a saucepan.
	Add water, lemon juice \& zest.
	Whisk in egg yolks.
	Whisk nonstop over medium, heat, for $\approx 5\dots7$ \minute, until it boils.
	Reduce heat and continue whisking for $\approx 60 \dots 90$ \second.

\item Emulsify custard.
	Remove from heat, continue whisking, and add the butter slowly one piece at a time.

\item Cool custard.
	Pour custard into shell, let settle \& cool briefly.
	Press plastic wrap on top to keep air out and let cool to room temperature.
	Refrigerate and use within $1\; day$.

\item Prepare meringue prior to serving.
	Preheat broiler.
	Beat egg whites on medium-high to soft peaks, beat in cream of tartar \& salt.
	Slowly beat in sugar to firm or stiff peaks.
	Beat in vanilla briefly. Top the pie and broil.
\end{preparation}


\recipeend


\clearpage
\addcontentsline{toc}{chapter}{References}
\printbibliography

\end{document}
